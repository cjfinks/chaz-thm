\chapter{Appendix: Proof of Main Lemma}\label{proofs}

We prove Lem.~\ref{MainLemma} after the following auxiliary lemmas.  % stating and proving some auxiliary lemmas. 
%First, note that since $\|\mathbf{A}\mathbf{x}\|_2 \leq \max_j\|\mathbf{A}_j\|_2\|\mathbf{x}\|_1$ and $\|\mathbf{x}\|_1 \leq \sqrt{k} \|\mathbf{x}\|_2$ for $k$-sparse $\mathbf{x}$, by \eqref{Ldef} we have the following frequently applied inequality:
%\begin{align}\label{delrho}
%L_{k'}(\mathbf{A}) \leq  \max_j\|\mathbf{A}_j\|_2 \ \ \text{for all $k' \leq 2k$}.
%\end{align}

% and then sketch the proofs of Thm.~\ref{robustPolythm} and Cor.~\ref{ProbabilisticCor}.

%\begin{lemma}\label{spanIntersectionLemma}
%Let $\mathbf{M} \in \mathbb{R}^{n \times m}$. If every $2k$ columns of $\mathbf{M}$ are linearly independent, then for any $F \subseteq {[m] \choose k}$:
%\begin{align*}
%\text{\rmfamily span}\{\mathbf{M}_{\cap F}\}  = \bigcap_{S \in F} \text{\rmfamily span}\{\mathbf{M}_S\}.
%\end{align*}
%\end{lemma}
%\begin{proof}
%Using induction on $|F|$, it is enough to prove the lemma when $|F| = 2$; but this case follows directly from the assumption.
%\end{proof}

\begin{lemma}\label{spanIntersectionLemma}
If $f: V \to W$ is injective, then $f\left(\cap_{i=1}^\ell V_i \right) =  \cap_{i=1}^\ell f\left(V_i\right)$ for any $V_1, \ldots, V_\ell \subseteq V$. ($f(\emptyset):=\emptyset$.)
%In particular, if for some $E \in 2^{[m]}$ the map $M \in \mathbb{R}^{n \times m}$ is injective on $\cup_{S \in \mathcal{H}} \text{span}\{e_i\}_{i \in S}$ then $\text{span}\{ M_{\cap_{S \in \mathcal{H}} S} \} = \cap_{S \in \mathcal{H}} \text{span}\{M_S\}$.
\end{lemma}
\begin{proof}
By induction, it is enough to prove the case $\ell = 2$. Clearly, for any map $f$, if $w \in f(U \cap V)$ then $w \in f(U)$ and $w \in f(V)$; hence, $w \in f(U) \cap f(V)$. If $w \in f(U) \cap f(V)$, then $w \in f(U)$ and $w \in f(V)$; thus, $w = f(u) = f(v)$ for some $u \in U$ and $v \in V$, implying $u = v$ by injectivity of $f$. It follows that $u \in U \cap V$ and $w \in f(U \cap V)$.
\end{proof}
In particular, if a matrix $\mathbf{A}$ satisfies $L_{2\mathcal{H}}(\mathbf{A}) > 0$, then taking $V$ to be the union of subspaces consisting of vectors with supports in $2\mathcal{H}$, we have $\bm{\mathcal{A}}_{\cap \mathcal{G}} = \cap \bm{\mathcal{A}}_\mathcal{G}$ for all $\mathcal{G} \subseteq \mathcal{H}$.
% \vspace{-.4 cm}

\begin{lemma}\label{DistanceToIntersectionLemma}
Let $\mathcal{V} = \{V_i\}_{i=1}^k$ be a set of two or more subspaces of $\mathbb{R}^m$, and set $V = \cap \mathcal{V} $. For  $\mathbf{u} \in \mathbb{R}^m$, we have (recall Defs.~\ref{dDef}~\&~\ref{FriedrichsDefinition}):
\begin{align}\label{DTILeq}
\text{\rmfamily dist}(\mathbf{u}, V) \leq \frac{1}{1 - \xi(\mathcal{V})} \sum_{i=1}^k \text{\rmfamily dist}(\mathbf{u}, V_i).
\end{align}
\end{lemma}
\begin{proof} 
% When $V = \{\mathbf{0}\}$, the result is trivial, so suppose otherwise.  
Recall the projection onto the subspace $V \subseteq \mathbb{R}^m$ is the mapping $\Pi_V: \mathbb{R}^m \to V$ that associates with each $\mathbf{u}$ its unique nearest point in $V$; i.e., $\|\mathbf{u} - \Pi_V\mathbf{u}\|_2 = \text{\rmfamily dist}(\mathbf{u}, V)$.
By repeatedly applying the triangle inequality, we have:
\begin{align}\label{f}
\|\mathbf{u} - &\Pi_V\mathbf{u}\|_2 
\leq \|\mathbf{u} - \Pi_{V_k} \mathbf{u}\|_2 + \|\Pi_{V_k}  \mathbf{u} - \Pi_{V_k}\Pi_{V_{k-1}}\mathbf{u}\|_2 \nonumber \\
&\ \ \ \ \ \ \ \ \ \ \ + \cdots + \|\Pi_{V_k} \Pi_{V_{k-1}}\cdots \Pi_{V_1} \mathbf{u} - \Pi_V \mathbf{u}\|_2 \nonumber \\
&\leq  \sum_{\ell=1}^k \|\mathbf{u} - \Pi_{V_{\ell}} \mathbf{u}\|_2 
+ \|(\Pi_{V_k}\cdots\Pi_{V_{1}} - \Pi_V) \mathbf{u}\|_2,
\end{align}
where we have also used that the spectral norm of the orthogonal projections $\Pi_{V_{\ell}}$ satisfies $\|\Pi_{V_{\ell}}\|_2 \leq 1$ for all $\ell$. 

It remains to bound the second term in \eqref{f} by $\xi(\mathcal{V}) \|\mathbf{u} - \Pi_V\mathbf{u}\|_2$. First, note that $\Pi_{V_\ell} \Pi_V = \Pi_V$ and $\Pi_V^2 = \Pi_V$, so we have $\|(\Pi_{V_k} \cdots \Pi_{V_1} - \Pi_V) \mathbf{u} \|_2 
= \| ( \Pi_{V_k} \cdots\Pi_{V_1} - \Pi_V ) (\mathbf{u} - \Pi_V\mathbf{u})\|_2$. % \leq z\|\mathbf{u} - \Pi_V\mathbf{u}\|_2$.
Consequently, inequality \eqref{DTILeq} follows from \cite[Thm.~9.33]{Deutsch12}:
\begin{align}
\|\Pi_{V_k}\Pi_{V_{k-1}}\cdots\Pi_{V_1} \mathbf{x} - \Pi_V\mathbf{x}\|_2 \leq z \|\mathbf{x}\|_2, \ \ \text{for all } \mathbf{x},
\end{align}
with \mbox{$z^2= 1 - \prod_{\ell =1}^{k-1}(1-z_{\ell}^2)$} and \mbox{$z_{\ell} = \cos\theta\left(V_{\ell}, \cap_{s=\ell+1}^k V_s\right)$} (recall $\theta$ from Def.~\ref{FriedrichsDefinition}), after substituting $\xi(\mathcal{V})$ for $z$ and rearranging terms.
\end{proof}
\begin{lemma}\label{NonEmptyLemma} 
Fix an $r$-regular hypergraph $\mathcal{H} \subseteq 2^{[m]}$ satisfying the SIP. If the map $\pi: \mathcal{H} \to 2^{[\overline m]}$ has $\sum_{S \in \mathcal{H}} |\pi(S)| \geq \sum_{S \in \mathcal{H}} |S|$ and:
\begin{align}\label{cond}
	|\cap \pi(\mathcal{G})| \leq |\cap \mathcal{G} |,\ \ \   \text{for } \mathcal{G} \in {\mathcal{H} \choose r} \cup {\mathcal{H} \choose r+1},
\end{align}
%
then $\overline m \geq m$; and if $\overline m  (r-1) < mr$, the map $i \mapsto \cap_{S \in \sigma(i)} \pi(S)$ is an injective function to $[\overline m]$ from some $J \subseteq [m]$ of size $m - (r-1)(\overline m - m)$ (recall $\sigma$ from Def.~\ref{sip}).  %In particular, if $\overline m = m$ then $\pi$ is induced by a permutation on $[m]$.
\end{lemma}

\begin{proof}
Consider the following set: $T_1 := \{(i, S): i \in \pi(S), S \in \mathcal{H}\}$, which numbers $|T_1| = \sum_{S \in \mathcal{H}} |\pi(S)| \geq \sum_{S \in \mathcal{H}} |S| = \sum_{i \in [m]} \deg_\mathcal{H}(i) = mr$ by $r$-regularity of $\mathcal{H}$. Note that $|T_1| \leq \overline m r$; otherwise, pigeonholing the tuples of $T_1$ with respect to their $\overline m$ possible first elements would imply that more than $r$ of the tuples in $T_1$ share the same first element. This cannot be the case, however, since then some $\mathcal{G} \in {\mathcal{H} \choose r+1}$ formed from any $r+1$ of their second elements would satisfy $\cap \pi(\mathcal{G}) \neq 0$; hence, $|\cap \mathcal{G}| \neq 0$ by \eqref{cond}, contradicting $r$-regularity of $\mathcal{H}$. It follows that $\overline m \geq m$.

Suppose now that $\overline m (r-1) < mr$, so that $p := mr - \overline m (r-1)$ is positive and $|T_1| \geq \overline m (r - 1) + p$. Pigeonholing $T_1$ into $[\overline m]$ again, there are at least $r$ tuples in $T_1$ sharing some first element; that is, for some $\mathcal{G}_1 \subseteq \mathcal{H}$ of size $|\mathcal{G}_1| \geq r$, we have $|\cap \pi(\mathcal{G}_1)| \geq 1$ and (by \eqref{cond}) $|\cap \mathcal{G}_1| \geq 1$. Since no more than $r$ tuples of $T_1$ can share the same first element, we in fact have $|\mathcal{G}_1| = r$. It follows by $r$-regularity that $\mathcal{G}_1$ is a star of $\mathcal{H}$; hence, $|\cap \mathcal{G}_1| = 1$ by the SIP and $|\cap \pi(\mathcal{G}_1)|  = 1$ by \eqref{cond}.

If $p=1$, then we are done. Otherwise, define $T_2 := T_1 \setminus \{(i,S) \in T_1: i = \cap \pi(\mathcal{G}_1)\}$, which contains $|T_2| = |T_1| - r \geq (\overline m - 1)(r-1) + (p-1)$ ordered pairs having $\overline m - 1$ distinct first indices. Pigeonholing $T_2$ into $[\overline m - 1]$ and repeating the above arguments produces the star $\mathcal{G}_2 \in {\mathcal{H} \choose r}$ with intersection $\cap \mathcal{G}_2$ necessarily distinct (by $r$-regularity) from $\cap \mathcal{G}_1$. Iterating this procedure $p$ times in total yields the stars $\mathcal{G}_i$ for which $\cap\mathcal{G}_i \mapsto \cap \pi(\mathcal{G}_i)$ defines an injective map to $[\overline m]$ from $J = \{\cap \mathcal{G}_1, \ldots, \cap \mathcal{G}_p\} \subseteq [m]$.
\end{proof}

\begin{proof}[Proof of Lem.~\ref{MainLemma}]
%We will show that the bound \eqref{GapUpperBound} trickles through the intersection semi-lattices of $\{\bm{\mathcal{A}}_S\}_{S \in \mathcal{H}}$ and $\{\bm{\mathcal{B}}_{\overline S}\}_{\overline S \in \pi(\mathcal{H})}$ to yield \eqref{MainLemmaBPD} by virtue of the SIP.  
We begin by showing that $\dim(\bm{\mathcal{B}}_{\pi(S)}) = \dim(\bm{\mathcal{A}}_S)$ for all $S \in \mathcal{H}$. Note that since $\|\mathbf{A}\mathbf{x}\|_2 \leq \max_j\|\mathbf{A}_j\|_2\|\mathbf{x}\|_1$ and $\|\mathbf{x}\|_1 \leq \sqrt{k} \|\mathbf{x}\|_2$ for all $k$-sparse $\mathbf{x}$, by \eqref{Ldef} we have $L_2(\mathbf{A}) \leq \max_j\|\mathbf{A}_j\|_2$ and therefore (as $0 \leq \xi < 1$) the right-hand side of \eqref{GapUpperBound} is less than one. From \eqref{dimLem}, we have $|\pi(S)| \geq \dim(\bm{\mathcal{B}}_{\pi(S)}) \geq \dim(\bm{\mathcal{A}}_S) = |S|$, the final equality holding by injectivity of $\mathbf{A}_S$. As $|\pi(S)| = |S|$, the claim follows. Note, therefore, that $\mathbf{B}_{\pi(S)}$ has full-column rank for all $S \in \mathcal{H}$.

We next demonstrate that \eqref{cond} holds. Fixing $\mathcal{G} \in {\mathcal{H} \choose r} \cup {\mathcal{H} \choose r+1}$, it suffices to show that $d(\bm{\mathcal{B}}_{\cap \pi(\mathcal{G})}, \bm{\mathcal{A}}_{\cap \mathcal{G}} ) < 1$, since by \eqref{dimLem} we then have $|\cap \pi(\mathcal{G})| = \dim(\bm{\mathcal{B}}_{\cap \pi(\mathcal{G})}) \leq \dim(\bm{\mathcal{A}}_{\cap \mathcal{G}}) = |\cap \mathcal{G}|$, with equalities from the full column-ranks of $\mathbf{A}_{S}$ and $\mathbf{B}_{\pi(S)}$ for all $S \in \mathcal{H}$.\footnote{Note that if ever $\bm{\mathcal{B}}_{\cap \pi(\mathcal{G})} \neq /bfseries 0$ while $\cap \mathcal{G} = \emptyset$, we would have $d(\bm{\mathcal{B}}_{\cap \pi(\mathcal{G})}, \bm 0 ) = 1$. However, that leads to a contradiction.} Observe that $d(\bm{\mathcal{B}}_{\cap \pi(\mathcal{G})}, \bm{\mathcal{A}}_{\cap \mathcal{G}}  ) 
\leq d\left( \cap \bm{\mathcal{B}}_{\pi(\mathcal{G})}, \cap \bm{\mathcal{A}}_\mathcal{G} \right)$ by \eqref{UsubU}, since trivially $\bm{\mathcal{B}}_{\cap \pi(\mathcal{G})} \subseteq \cap \bm{\mathcal{B}}_{\pi(\mathcal{G})}$ and also $\bm{\mathcal{A}}_{\cap \mathcal{G}} = \cap \bm{\mathcal{A}}_\mathcal{G}$ by Lem.~\ref{spanIntersectionLemma}. Recalling Def.~\ref{dDef} and applying Lem.~\ref{DistanceToIntersectionLemma} yields:
\begin{align}
d\left( \cap \bm{\mathcal{B}}_{\pi(\mathcal{G})}, \cap \bm{\mathcal{A}}_\mathcal{G} \right)
&\leq \max_{\mathbf{u} \in \cap \bm{\mathcal{B}}_{\pi(\mathcal{G})}, \ \|\mathbf{u}\|_2 \leq 1} \sum_{S \in \mathcal{G}} \frac{ \text{\rmfamily dist}\left( \mathbf{u},\bm{\mathcal{A}}_{S} \right) }{ 1 - \xi( \bm{\mathcal{A}}_\mathcal{G} ) } \nonumber \\
&= \sum_{S \in \mathcal{G}} \frac{ d\left( \cap \bm{\mathcal{B}}_{\pi(\mathcal{G})},\bm{\mathcal{A}}_{S} \right) }{ 1 - \xi( \bm{\mathcal{A}}_\mathcal{G} ) }, \nonumber
\end{align}
passing the maximum through the sum.
Since $\cap \bm{\mathcal{B}}_{\pi(\mathcal{G})} \subseteq \bm{\mathcal{B}}_{\pi(S)}$ for all $S \in \mathcal{G}$, by \eqref{UsubU} the numerator of each term in the sum above is bounded by \mbox{$d\left( \bm{\mathcal{B}}_{\pi(S)},\bm{\mathcal{A}}_{S} \right) = d\left(\bm{\mathcal{A}}_{S}, \bm{\mathcal{B}}_{\pi(S)} \right) \leq \varepsilon$}, with the equality from \eqref{eqdim} since $\dim(\bm{\mathcal{B}}_{\pi(S)}) = \dim(\bm{\mathcal{A}}_S)$. Thus, altogether:
\begin{align}\label{last}
d(\bm{\mathcal{B}}_{\cap \pi(\mathcal{G})}, \bm{\mathcal{A}}_{\cap \mathcal{G}} )
\leq \frac{|\mathcal{G}| \varepsilon}{1 - \xi( \bm{\mathcal{A}}_\mathcal{G} )}
\leq \frac{C_2 \varepsilon}{\max_j\|\mathbf{A}_j\|_2},
\end{align}
recalling the definition of $C_2$ in \eqref{Cdef2}. Lastly, since $C_2 \varepsilon < L_2(\mathbf{A}) \leq \max_j\|\mathbf{A}_j\|_2$, we have $d(\bm{\mathcal{B}}_{\cap \pi(\mathcal{G})}, \bm{\mathcal{A}}_{\cap \mathcal{G}} ) \leq 1$ and therefore \eqref{cond} holds.

%Keeping in mind that $d(U',V) \leq d(U,V)$ whenever $U' \subseteq U$ and applying (in order) Lem.~\ref{spanIntersectionLemma}, Lem.~\ref{DistanceToIntersectionLemma}, \eqref{eqdim}, \eqref{GapUpperBound}, and \eqref{Cdef2} gives:
%\begin{align}\label{randoml}
%d(&\bm{\mathcal{B}}_{\cap \pi(\mathcal{G})}, \bm{\mathcal{A}}_{\cap \mathcal{G}}  ) 
%\leq d\left( \cap \bm{\mathcal{B}}_{\pi(\mathcal{G})}, \cap \bm{\mathcal{A}}_\mathcal{G} \right)
%\leq \sum_{T \in \mathcal{G}} \frac{ d\left( \cap \bm{\mathcal{B}}_{\pi(\mathcal{G})},\bm{\mathcal{A}}_{T} \right) }{ 1 - \xi( \bm{\mathcal{A}}_\mathcal{G} ) } \nonumber \\
%&\leq \sum_{T \in \mathcal{G}} \frac{ d\left( \bm{\mathcal{B}}_{\pi(T)},\bm{\mathbf{A}}_{T} \right) }{ 1 - \xi( \bm{\mathcal{A}}_\mathcal{G} ) }
%\leq \frac{|\mathcal{G}| \varepsilon}{1 - \xi( \bm{\mathcal{A}}_\mathcal{G} )} 
%\leq \frac{C_2 \varepsilon}{\max_j\|\mathbf{A}_j\|_2},
%\end{align}
%
%which is less than one, since $C_2 \varepsilon < L_2(\mathbf{A}) \leq \max_j\|\mathbf{A}_j\|_2$.

Applying Lem.~\ref{NonEmptyLemma}, the association $i \mapsto \cap_{S \in \sigma(i)} \pi(S)$ is an injective map $\overline \pi: J \to [\overline m]$ for some $J \subseteq [m]$ of size $m - (r-1)(\overline m - m)$, and $\mathbf{B}_{\overline \pi(i)} \neq \mathbf{0}$ for all $i \in J$ since the columns of $\mathbf{B}_{\pi(S)}$ are linearly independent for all $S \in \mathcal{H}$. Letting $\overline \varepsilon := C_2 \varepsilon / \max_i \|\mathbf{A}_i\|_2$, it follows from \eqref{eqdim} and \eqref{last} that $d\left( \bm{\mathcal{A}}_i, \bm{\mathcal{B}}_{\overline \pi(i)} \right) = d\left(\bm{\mathcal{B}}_{\overline \pi(i)},  \bm{\mathcal{A}}_i \right)  \leq \overline \varepsilon$ for all $i \in J$. %Fixing $\overline \varepsilon = C_2\varepsilon$ and letting $c_i = \|\mathbf{A}_i\|_2^{-1}$, we thus have that for each $\mathbf{e}_i \in \mathbb{R}^m$ with $i \in J$ there exists some $\overline{c}_i \in \mathbb{R}$ such that $\|c_i\mathbf{A}\mathbf{e}_i - \overline{c}_i \mathbf{B}\mathbf{e}_{\overline \pi(i)}\|_2 \leq \overline \varepsilon < L_2(\mathbf{A}) \min_{i\in J} |c_i|$. But this is exactly the supposition in \eqref{1D}, and the result follows from the case $k=1$ in Sec.~\ref{DUT} applied to the submatrix $\mathbf{A}_J$. 
Setting $c_i := \|\mathbf{A}_i\|_2^{-1}$ so that $\|c_i\mathbf{Ae}_i\|_2 = 1$, by Def.~\ref{dDef} for all $i \in J$:
\begin{align*}
\min_{\overline c_i \in \mathbb{R}} \|c_i\mathbf{Ae}_i - \overline c_i \mathbf{Be}_{\overline \pi(i)} \|_2
\leq d\left( \bm{\mathcal{A}}_i, \bm{\mathcal{B}}_{\overline \pi(i)} \right)
\leq \overline \varepsilon,
%&= \max_{\bm{u} \in \bm{\mathcal{A}}_i, \|\bm{u}\|_2 \leq 1} \text{dist}\left(\bm{u}, \bm{\mathcal{B}}_{\overline \pi(i)} \right) \\
%&\geq  \text{dist}\left(c_i\bm{Ae}_i, \mathbf{Be}_{\overline \pi(i)} \right) \\
%&= \min_{\overline c_i \in \mathbb{R}} \|c_i\bm{Ae}_i - \overline c_i \mathbf{Be}_{\overline \pi(i)} \|_2
\end{align*}
%
for $\overline \varepsilon < L_2(\mathbf{A})\min_{i \in [m]}|c_i|$. But this is exactly the supposition in \eqref{1D}, with $J$ and $\overline \varepsilon$ in place of $[m]$ and $\varepsilon$, respectively. The same arguments of the case $k=1$ in Sec.~\ref{DUT} can then be made to show that for any $\overline m \times \overline m$ permutation and invertible diagonal matrices $\mathbf{P}$ and $\mathbf{D}$ with, respectively, columns $\mathbf{e}_{\pi(i)}$ and $\frac{\overline{c}_i}{c_i}\mathbf{e}_i$ for $i \in J$ (otherwise, $\mathbf{e}_{i}$ for $i \in [\overline{m}] \setminus J$), we have $\|\mathbf{A}_i - (\mathbf{B}\mathbf{PD})_i \|_2 \leq \overline  \varepsilon / |c_i|  \leq C_2 \varepsilon$ for all $i \in J$.
\end{proof}
