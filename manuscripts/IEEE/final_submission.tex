% TODO: cite Fritz l0 paper in introduction for seemingly characteristic structure? Use those mentioned in PNAS response?
% Make L_H a definition..? Change def of L to be 1-norm so we don't need these sqrt k's?
% When do we have polynomial complexity? Is it worth saying something about this for cyclic order (e.g. convolutions?)
% mention regularity is probably not necessary
% have i talked about making deterministic A? (see reviewer comment, and comment about easily meeting thm 1 conditions w deterministic construction). also after f(A). we make L_2k(a) with vandermonde, clarify this is way more than necessary...but it is necessary if we care about sparse coefficient recovery too, which is part of df of stable sparse representation.
% IS THEOREM 3 AND ITS COROLLARY CORRECTLY STATED??
% mention use of coherence instead of L_k? for practical criteria?

% Notes:
%
% Can we prove in a line or two that the inverse problem is stable by arguing that the function is bijective and closed therefore a homeomorphism?
% Should we mention the notion of Grassmannian manifolds and that Theta is a metric on this space?
% Change robustness to stability?
%
\documentclass[journal, twocolumn]{IEEEtran}

% *** MATH PACKAGES ***
\usepackage{amsmath, amssymb, amsthm} 
\newtheorem{theorem}{Theorem}
\newtheorem{lemma}{Lemma}
\newtheorem{conjecture}{Conjecture}
\newtheorem{problem}{Problem}
\newtheorem{question}{Question}
\newtheorem{proposition}{Proposition}
\newtheorem{definition}{Definition}
\newtheorem{corollary}{Corollary}
\newtheorem{remark}{Remark}
\newtheorem{example}{Example}


\usepackage[pdftex]{graphicx}

% *** ALIGNMENT PACKAGES ***
\usepackage{array}
\usepackage{cite}
\usepackage{bm}

% correct bad hyphenation here
%\hyphenation{op-tical net-works semi-conduc-tor}

\begin{document}

%\title{When can a generating dictionary and sparse codes be recovered from noisy data?}
%\title{When is the dictionary learning problem well-posed?}
\title{On the uniqueness and stability of dictionaries for sparse representation of noisy signals}
%\title{When is the dictionary learning problem for sparse coding well-posed?}
%\title{When is the sparsest representation unique?}
%\title{When are sparse codes identifiable from their unknown noisy compressive measurements?}
% FRITZ2: The title doesn't reflect the amount of new stuff, it seems incremental from the last one. It's not very meaningful.
% FRITZ: Not everyone knows what well-posed means. You should mention recovery from noisy data in the title.

\author{Charles~J.~Garfinkle and Christopher~J.~Hillar \\
Redwood Center for Theoretical Neuroscience, Berkeley, CA, USA
%\thanks{%e-mails: cjg@berkeley.edu, chillar@msri.org.  
%CG and CH supported, in part, by NSF grant IIS-1219212 and the Statistical and Applied Mathematical Sciences Institute, under NSF DMS-1127914.}
}

\maketitle

% \keywords{Sparse coding, dictionary learning, matrix factorization, compressed sensing, inverse problems, blind source separation} 

\begin{abstract}
Learning optimal dictionaries for sparse coding has exposed characteristic sparse features of many natural signals. However, there are few universal conditions guaranteeing the stability of such features in the presence of measurement noise. Here, we prove very generally that optimal dictionaries and sparse codes are unique up to measurement error for all diverse enough datasets generated by the sparse coding model. Applications are given to data analysis, engineering, and neuroscience. 
\end{abstract}

% Optional adjustment to line up main text (after abstract) of first page with line numbers, when using both lineno and twocolumn options.
% You should only change this length when you've finalised the article contents.


\section{Introduction}\label{Intro}
%\IEEEPARstart{L}{earning} sparse representations of data has become an important step in modern solutions to problems in signal processing and pattern analysis. 
% \cite{sato1975method}
\IEEEPARstart{A}{}common modern approach to pattern analysis in signal processing is to view each of $N$ observed $n$-dimensional signal samples as a (noisy) linear combination of at most $k$ elementary waveforms drawn from some unknown ``dictionary" of size $m \ll N$ (see \cite{Zhang15} for a comprehensive review). 
Optimizing dictionaries subject to this and related sparsity constraints has revealed seemingly characteristic sparse structure in several signal classes of current interest (e.g., in vision \cite{wang2015sparse}). 
Of particular note are the seminal works in the field \cite{Olshausen96, hurri1996image, bell1997independent, van1998independent}, which discovered that dictionaries optimized for coding small patches of ``natural" images share qualitative similarities with linear filters estimated from response properties of simple-cell neurons in mammalian visual cortex. Curiously, these waveforms (e.g., ``Gabor'' wavelets) appear in dictionaries learned by a variety of algorithms trained over different natural image datasets; in particular, sparse coding of  latent features in natural signals may, in some sense, be canonical \cite{donoho2001can}.

Motivated by these discoveries and more recent work relating ``compressed sensing" to a theory of information transmission through random wiring bottlenecks in the brain \cite{Isely10}, we address when optimal dictionaries for sparse representation are indeed identifiable from data. Answers to this question may also have implications in practice wherever an appeal is made to latent sparse structure of data (e.g., forgery detection \cite{hughes2010, olshausen2010applied}; brain recordings \cite{jung2001imaging, agarwal2014spatially, lee2016sparse}; and gene expression \cite{wu2016stability}). While several algorithms have recently been proposed that provably recover unique dictionaries under specific conditions (see \cite[Sec.~I-E]{Sun16} for a summary of the state-of-the-art), few theorems can be invoked to justify the consistency of inference under this model of data more broadly. To our knowledge, a universal guarantee of the uniqueness and stability of learned dictionaries and the sparse representations they induce when data is noisy has yet to appear in the literature.
% in justifying the identification of forged paintings by their anomalous sparse representations \cite{hughes2010},

Here, we prove very generally that uniqueness and stability is a typical property of sparse dictionary learning. More specifically,  we show that dictionaries injective on a sparse domain are identifiable from \mbox{$N = m(k-1){m \choose k} + m$} noisy sparse linear combinations of their columns up to an error that is linear in the noise (Thm.~\ref{DeterministicUniquenessTheorem}). In fact, provided $n \geq \min(2k,m)$, in almost all cases the problem is well-posed (as per Hadamard \cite{Hadamard1902}) given enough data (Cor.~\ref{ProbabilisticCor}). These guarantees also hold for a related (and perhaps more commonly posed) optimization problem which seeks a dictionary minimizing the average number of elementary waveforms required to reconstruct each sample of the dataset (Thm.~\ref{SLCopt}). Importantly, in both cases our explicit, universal guarantees apply without the imposition of \emph{any} constraints on learned dictionaries  (e.g. that they, too, be injective over some domain) beyond an upper bound on dictionary size, which is necessary in any case to avoid a trivial solution (e.g., allowing for $m = N$ elementary waveforms). %That is, every pair of solutions of comparable size has some number of dictionary elements in common (up to noise), and similarly so for the coefficients of sparse codes they induce.

More precisely, let $\mathbf{A} \in \mathbb R^{n \times m}$ be a matrix with columns $\mathbf{A}_j$ ($j = 1,\ldots,m$) and let dataset $Z$ consist of measurements:
\begin{align}\label{LinearModel}
\mathbf{z}_i = \mathbf{A}\mathbf{x}_i + \mathbf{n}_i,\ \ \  \text{$i=1,\ldots,N$},
\end{align}
for $k$-\emph{sparse} $\mathbf{x}_i \in \mathbb{R}^m$ having at most $k<m$ nonzero entries and \emph{noise} $\mathbf{n}_i \in \mathbb{R}^n$, with bounded norm $\| \mathbf{n}_i \|_2 \leq  \eta$ representing our worst-case uncertainty in measuring the product $\mathbf{A}\mathbf{x}_i$. The first mathematical problem we consider is the following.

%
 
\begin{problem}\label{InverseProblem}
Find a real matrix $\mathbf{B}$ and $k$-sparse $\mathbf{\overline x}_1, \ldots, \mathbf{\overline x}_N$ that satisfy $\|\mathbf{z}_i - \mathbf{B}\mathbf{\overline x}_i\|_2 \leq \eta$ for all $i$.
\end{problem}

Note that every solution to Prob.~\ref{InverseProblem} represents infinitely many equivalent alternatives $\mathbf{BPD}$ and $\mathbf{D}^{-1}\mathbf{P}^{\top}\mathbf{\overline x}_1, \ldots, \mathbf{D}^{-1}\mathbf{P}^{\top}\mathbf{\overline x}_N$ parametrized by a choice of permutation matrix $\mathbf{P}$ and invertible diagonal matrix $\mathbf{D}$. Identifying these ambiguities (labelling and scale) yields a set of elementary waveforms (the columns of $\mathbf{B}$) and their associated sparse coefficients (the entries of $\mathbf{\overline x}_i$) for reconstructing each data point $\mathbf{z}_i$. 

Previous theoretical work addressing the noiseless case $\eta =0$ (e.g., \cite{li2004analysis, Georgiev05, Aharon06, Hillar15}) for matrices $\mathbf{B}$ having exactly $m$ columns has shown that a solution to Prob.~\ref{InverseProblem}, when it exists, is unique up to such relabeling and rescaling provided the $\mathbf{x}_i$ are sufficiently diverse and $\mathbf{A}$ satisfies the \textit{spark condition}:
\begin{align}\label{SparkCondition}
\mathbf{A}\mathbf{x}_1 = \mathbf{A}\mathbf{x}_2 \implies \mathbf{x}_1 = \mathbf{x}_2, \ \ \ \text{for all $k$-sparse } \mathbf{x}_1, \mathbf{x}_2,
\end{align}
%
which in any case is necessary to guarantee the uniqueness of arbitrary $k$-sparse $\mathbf{x}_i$. We generalize these results to the practical setting  $\eta > 0$ by considering the following natural notion of stability with respect to measurement error.

\begin{definition}\label{maindef}
Fix $Y = \{ \mathbf{y}_1, \ldots, \mathbf{y}_N\} \subset \mathbb{R}^n$. We say $Y$ has a \textbf{$k$-sparse representation in $\mathbb{R}^m$} if there exists a matrix $\mathbf{A}$ and $k$-sparse $\mathbf{x}_1, \ldots, \mathbf{x}_N \in \mathbb{R}^m$ such that $\mathbf{y}_i = \mathbf{A}\mathbf{x}_i$ for all $i$. 
This representation is \textbf{stable} if for every $\delta_1, \delta_2 \geq 0$, there exists some $\varepsilon = \varepsilon(\delta_1, \delta_2)$ that is strictly positive for positive $\delta_1$ and $\delta_2$ such that if $\mathbf{B}$ and $k$-sparse $\mathbf{\overline x}_1, \ldots, \mathbf{\overline x}_N \in \mathbb{R}^m$ satisfy:
\begin{align*}
	\|\mathbf{A}\mathbf{x}_i - \mathbf{B}\mathbf{\overline x}_i\|_2 \leq \varepsilon(\delta_1, \delta_2),\ \ \   \text{for all $i=1,\ldots,N$},
\end{align*}
then there is some permutation matrix $\mathbf{P}$ and invertible diagonal matrix $\mathbf{D}$ such that for all $i, j$:
\begin{align}\label{def1}
\|\mathbf{A}_j - \mathbf{BPD}_j\|_2 \leq \delta_1 \ \ \text{and} \ \ \|\mathbf{x}_i - \mathbf{D}^{-1}\mathbf{P}^{\top}\mathbf{\overline x}_i\|_1 \leq \delta_2.
\end{align}
\end{definition}

To see how Prob. \ref{InverseProblem} motivates Def. \ref{maindef}, suppose that $Y$ has a stable $k$-sparse representation in $\mathbb{R}^m$ and fix $\delta_1, \delta_2$ to be the desired accuracies of recovery in \eqref{def1}. Consider any dataset $Z$ generated as in \eqref{LinearModel} with $\eta \leq \frac{1}{2} \varepsilon(\delta_1, \delta_2)$. Using the triangle inequality, it follows that any $n \times m$ matrix $\mathbf{B}$ and $k$-sparse $\mathbf{\overline x}_1, \ldots, \mathbf{\overline x}_N$ solving Prob.~\ref{InverseProblem} are necessarily within $\delta_1$ and $\delta_2$ of the original dictionary $\mathbf{A}$ and codes $\mathbf{x}_1, \ldots, \mathbf{x}_N$, respectively.\footnote{We mention that the different norms in \eqref{def1} reflect the distinct meanings typically ascribed to the dictionary and sparse codes in modeling data.}

The main result of this work is a very general uniqueness theorem for sparse coding (Thm.~\ref{DeterministicUniquenessTheorem}) directly 
implying (Cor.~\ref{DeterministicUniquenessCorollary}) that stable representations of a dataset $Z$ are guaranteed whenever generating dictionaries $A$ satisfy a spark condition on supports and the original sparse codes $\mathbf{x}_i$ are sufficiently diverse.  Moreover, we provide an explicit, computable formula for $\varepsilon(\delta_1, \delta_2)$ in (\ref{epsdel}) that is linear in desired accuracy $\delta_1, \delta_2$.

In the next section, we give precise statements of these findings.  We then extend the same guarantees (Thm.~\ref{SLCopt}) to the following alternate formulation of dictionary learning, which minimizes the number of nonzero entries in sparse codes.

\begin{problem}\label{OptimizationProblem}
Find real $\mathbf{B}$ and \mbox{$\mathbf{\overline x}_1, \ldots, \mathbf{\overline x}_N$} that solve:
\begin{align}\label{minsum}
\min \sum_{\ell = 1}^N \|\mathbf{\overline x}_{\ell}\|_0 \ \
\text{subject to} \ \ \|\mathbf{z}_i - \mathbf{B}\mathbf{\overline x}_i\|_2 \leq \eta, \ \text{for all $i$}.
\end{align}
\end{problem}

Our formulation of Thm.~\ref{DeterministicUniquenessTheorem} is general enough to provide uniqueness stability even when generating $A$ do not satisfy (\ref{SparkCondition}) and recovery dictionaries $B$ have more columns than $m$.  Moreover, our approach incorporates a combinatorial theory for designing generating codes that should be of independent interest.
We also give brief arguments adapting our results to dictionaries and codes drawn from probability distributions (Cor.~\ref{ProbabilisticCor}). We defer to Sec.~\ref{DUT} the technical proofs of Thms.~\ref{DeterministicUniquenessTheorem} and ~\ref{SLCopt}, following some necessary definitions and a useful fact in combinatorial matrix analysis (Lem.~\ref{MainLemma}; proved in the Appendix). Finally, we discuss in Sec.~\ref{Discussion} theoretical and practical applications of our mathematical observations. 
%The Appendix contains a proof of Lem.~\ref{MainLemma}.

\section{Results}\label{Results}

%Denote by $\mathbf{x}^J$ the subvector formed from the entries of $\mathbf{x}$ indexed by $J$. 
%A set of $k$-sparse vectors is said to be in \emph{general linear position} when any $k$ of them are linearly independent. 

A detailed statement of our results requires that we first identify some combinatorial criteria on the supports\footnote{Recall a vector $\mathbf{x}$ is said to be \emph{supported} in $S$ when $\mathbf{x} \in \text{\rm span}\{\mathbf{e}_j: j\in S\}$, with $\mathbf{e}_j$ forming the standard column basis.} of sparse vectors. Let $\{1, \ldots, m\}$ be denoted $[m]$, its power set $2^{[m]}$, and ${[m] \choose k}$ the set of subsets of $[m]$ of size $k$.  A \emph{hypergraph} on vertices $[m]$  is simply any subset $\mathcal{H} \subseteq 2^{[m]}$. We say $\mathcal{H}$ is \textit{$k$-uniform} when in fact $\mathcal{H} \subseteq {[m] \choose k}$. The \emph{degree} $\deg_\mathcal{H}(i)$ of a node $i \in [m]$ is the number of sets in $\mathcal{H}$ that contain $i$, and we say $\mathcal{H}$ is \emph{regular} when for some $r$ we have $\deg_\mathcal{H}(i) = r$ for all $i$ (given such an $r$, we say $\mathcal{H}$ is \textit{$r$-regular}). We also write $2\mathcal{H} := \{ S \cup S': S, S' \in \mathcal{H}\}$.

\begin{definition}\label{sip}
Given $\mathcal{H} \subseteq 2^{[m]}$, the \textbf{star} $\sigma(i)$ is the collection of sets in $\mathcal{H}$ containing $i$. We say $\mathcal{H}$ has the \textbf{singleton intersection property} (\textbf{SIP}) when $\cap \sigma(i) = \{i\}$ for all $i \in [m]$. 
\end{definition}

Next, we describe a quantitative generalization of the spark condition. The \emph{lower bound} of a real $n \times m$ matrix $\mathbf{M}$ is the largest $\alpha$ with \mbox{$\|\mathbf{M}\mathbf{x}\|_2 \geq \alpha\|\mathbf{x}\|_2$} for all $\mathbf{x} \in \mathbb{R}^m$ \cite{Grcar10}. By compactness of the unit sphere, every injective linear map has a positive lower bound; hence, if $\mathbf{M}$ satisfies \eqref{SparkCondition} then each submatrix formed from $2k$ of its columns or less has a strictly positive lower bound. 

We generalize the matrix lower bound for our purposes by considering only the spans of certain submatrices\footnote{See \cite{vidal2005generalized} for an overview of the closely associated ``union of subspaces" model.} associated with a hypergraph $\mathcal{H} \subseteq {[m] \choose k}$ over column indices. Letting $\mathbf{M}_S$ denote the submatrix formed by the columns of $\mathbf{M}$ indexed by $S \subseteq [m]$, with $\mathbf{M}_\emptyset := \mathbf{0}$ (in the sections that follow, we will write $\bm{\mathcal{M}}_S$ to denote the column-span of a submatrix $\mathbf{M}_S$, and $\bm{\mathcal{M}}_\mathcal{G}$ to denote $\{\bm{\mathcal{M}}_S\}_{S \in \mathcal{G}}$), we define: 
%\begin{align*} 
%L_\mathcal{H}(\mathbf{M}) := \min \left\{ \frac{ \|\mathbf{M}(\mathbf{x}_1-\mathbf{x}_2)\|_2 }{ \sqrt{2k} \|\mathbf{x}_1-\mathbf{x}_2\|_2} : \mathbf{x}_1, \mathbf{x}_2 \in \cup_{S \in \mathcal{H}} \bm{\mathcal{M}}_S \right\},
%\end{align*} 
%
%where we write $L_{2k}$ in place of $L_\mathcal{H}$ when $\mathcal{H} = {[m] \choose k}$. Note that if $\mathcal{H}$ covers $[m]$, then $L_2 > L_\mathcal{H}$.\footnote{The reader should beware that $L_2 = L_{[m]}$, whereas $L = L_{\{[m]\}}$. For even $k$, the quantity $1 - \sqrt{k} L_k(\mathbf{M})$ is also known in the compressed sensing literature as the (asymmetric) lower restricted isometry constant \cite{Blanchard2011}.} Clearly, for any $\mathbf{M}$ satisfying \eqref{SparkCondition}, we have $L_{k'}(\mathbf{M}) > 0$ for  $k' \leq 2k$.
%\begin{align*}
%L_\mathcal{H}(\mathbf{M}) := \min \left\{ \frac{ \|\mathbf{M}_S\mathbf{x}\|_2 }{ \sqrt{k} \|\mathbf{x}\|_2} : S \in \mathcal{H} \right\},
%\end{align*} 
%\begin{align*}
%L_\mathcal{H}(\mathbf{M}) := \max \left\{ \ falpha: \|\mathbf{M}_S\mathbf{x}\|_2 \geq \alpha \sqrt{k} \|\mathbf{x}\|_2 : S \in \mathcal{H}, \ \ \mathbf{x} \in \mathbb{R}^m \right\},
%\end{align*} 
\begin{align}\label{Ldef}
L_\mathcal{H}(\mathbf{M}) := \frac{1}{\sqrt{k}} \min \left\{ \frac{\|\mathbf{M}_S\mathbf{x}\|_2}{ \|\mathbf{x}\|_2} : S \in \mathcal{H}, \ \ \mathbf{x} \in \mathbb{R}^{|S|} \right\},
\end{align} 
%
writing also $L_{k}$ in place of $L_\mathcal{H}$ when $\mathcal{H} = {[m] \choose k}$.\footnote{We note that \mbox{$1 - \sqrt{k} L_k(\mathbf{M})$} is known in the compressive sensing literature as the asymmetric lower restricted isometry constant for matrices $\mathbf{M}$ with unit $\ell_2$-norm columns \cite{Blanchard2011}}  As explained previously, by compactness we have $L_{2k}(\mathbf{M}) > 0$ for all $\mathbf{M}$ satisfying \eqref{SparkCondition}. Clearly, $L_{\mathcal{H}'}(\mathbf{M}) \geq L_\mathcal{H}(\mathbf{M})$ whenever $\mathcal{H}' \subseteq \mathcal{H}$ and, by the same token, any $k$-uniform $\mathcal{H}$ satisfying $\cup \mathcal{H} = [m]$ has $L_2 \geq L_{2\mathcal{H}} \geq L_{2k}$.

Finally, we mention that a set of vectors sharing a support $S$ are said to be in \emph{general linear position} when any $|S|$ of them form a linearly independent set.

We are now in a position to state our main result, though for expository purposes we will leave the quantity $C_1$ (a function of $\mathbf{A}$, $\mathbf{x}_1, \ldots, \mathbf{x}_N$, and $\mathcal{H}$) undefined until Eq.~\eqref{Cdef1} of Sec.~\ref{DUT}. We state here once that all of our theorems assume matrices and vectors consist of real numbers. 

%eps-tightness COUNTER-EXAMPLE:
%Consider the alternate dictionary $B = \left(\mathbf{0}, \frac{1}{2}(\mathbf{e}_1 + \mathbf{e}_2), \mathbf{e}_3, \ldots, \mathbf{e}_{m} \right)$ and sparse codes $\mathbf{b}_i = \mathbf{e}_2$ for $i = 1, 2$ and $\mathbf{b}_i = \mathbf{e}_i$ for $i = 3, \ldots, m$. Then $|A\mathbf{a}_i - B\mathbf{b}_i| = 1/\sqrt{2}$ for $i = 1, 2$ (and $0$ otherwise). If there were permutation and invertible diagonal matrices $P \in \mathbb{R}^{m \times m}$ and $D \in \mathbb{R}^{m \times m}$ such that $|(A-BPD)\mathbf{e}_i| \leq C\varepsilon$ for all $i \in [m]$, then we would reach the contradiction $1 = |P^{-1}\mathbf{e}_1|_2 = |(A-BPD)P^{-1}\mathbf{e}_1|_2 \leq 1/\sqrt{2}$. 

\begin{theorem}\label{DeterministicUniquenessTheorem}
Fix integers $n, k, m$ and $\overline m$. Let the $n \times m$ matrix $\mathbf{A}$ satisfy $L_{2\mathcal{H}}(\mathbf{A}) > 0$ for some $r$-regular $\mathcal{H} \subseteq {[m] \choose k}$ with the SIP. If $k$-sparse \mbox{$\mathbf{x}_1, \ldots, \mathbf{x}_N$} include more than $(k-1){\overline m \choose k}$ vectors in general linear position supported in each $S \in \mathcal{H}$, then there is $C_1 > 0$ (given by \eqref{Cdef1}) for which the following holds for all%\footnote{The condition $\varepsilon < L_2(\mathbf{A}) /C_1$ is necessary; otherwise,  with \mbox{$\mathbf{A}$ = $\mathbf{I}$} and $\mathbf{x}_i = \mathbf{e}_i$, there is a matrix $\mathbf{B}$ and $1$-sparse $\mathbf{\overline{x}}_i$ with $\|\mathbf{A}\mathbf{x}_i - \mathbf{B}\mathbf{\overline{x}}_i \|_2 \leq \varepsilon$ that nonetheless violates \eqref{Cstable}.} $\varepsilon < L_{2}(\mathbf{A}) / C_1$:
\footnote{Note that the condition $\varepsilon < L_2(\mathbf{A}) /C_1$ is necessary; otherwise, with \mbox{$\mathbf{A}$ = $\mathbf{I}$} (the identity matrix) and $\mathbf{x}_i = \mathbf{e}_i$, the matrix $\mathbf{B} = \left[\mathbf{0}, \frac{1}{2}(\mathbf{e}_1 + \mathbf{e}_2), \mathbf{e}_3, \ldots, \mathbf{e}_{m} \right]$ and sparse codes $\mathbf{\overline x}_i = \mathbf{e}_2$ for $i = 1, 2$ and $\mathbf{\overline x}_i = \mathbf{e}_i$ for $i \geq 3$ satisfy $\|\mathbf{A}\mathbf{x}_i - \mathbf{B}\mathbf{\overline{x}}_i \|_2 \leq \varepsilon$ but nonetheless violate \eqref{Cstable}.} $\varepsilon < L_{2}(\mathbf{A}) / C_1$:

Every $n \times \overline m$ matrix $\mathbf{B}$ for which there are $k$-sparse $\mathbf{\overline x}_1, \ldots, \mathbf{\overline x}_N$ satisfying \mbox{$\|\mathbf{A}\mathbf{x}_i - \mathbf{B}\mathbf{\overline x}_i\|_2 \leq \varepsilon$} for $i = 0, \ldots, N$ has $\overline m \geq m$, and provided $\overline m (r-1) < mr$, there is a permutation matrix $\mathbf{P}$ and an invertible diagonal matrix $\mathbf{D}$ such that:
\begin{align}\label{Cstable}
\|\mathbf{A}_j- \mathbf{BPD}_j\|_2 \leq C_1 \varepsilon \ \ \text{for all } j \in J
\end{align}
%
for some $J$ of size \mbox{$m - (r-1)(\overline m - m)$}. 
% $\overline m - r(\overline m - m)$

Moreover, if $\mathbf{A}$ satisfies \eqref{SparkCondition} and $\varepsilon < L_{2k}(\mathbf{A}) / C_1$, then $\mathbf{B}$ also satisfies \eqref{SparkCondition} with $L_{2k}(\mathbf{B}\mathbf{PD}) \geq L_{2k}(\mathbf{A}) - C_1 \varepsilon$ and for all $i \in [N]$:
\begin{align}\label{b-PDa}
%\|\mathbf{x}^J_i - \mathbf{D}^{-1}\mathbf{P}^{\top}\mathbf{\overline x}^{\overline J}_i\|_1 &\leq  \left( \frac{ 1+C_1 \|\mathbf{x}^{J}_i\|_1 }{ L_{2k}(\mathbf{A}) -  C_1\varepsilon } \right) \varepsilon \ \  \text{for $i \in [N]$}.
\|(\mathbf{x}_i)_J - (\mathbf{D}^{-1}\mathbf{P}^{\top} \mathbf{\overline x}_i)_J\|_1 &\leq  \left( \frac{ 1+C_1 \|\mathbf{x}_i\|_1 }{ L_{2k}(\mathbf{A}) -  C_1\varepsilon } \right) \varepsilon,
\end{align}
%
where $(\mathbf{x}_i)_J$ here represents the subvector formed from restricting to coefficients indexed by $J$. 
\end{theorem}

%We delay defining the explicit constant $C_1$ until Section \ref{DUT} (\eqref{Cdef1}).
%To be clear, the implication of Thm.~\ref{DeterministicUniquenessTheorem} is that $Y = \{\mathbf{Ax}_1, \ldots, \mathbf{Ax}_N\}$ has a stable $k$-sparse representation in $\mathbb{R}^m$, with \eqref{def1} guaranteed provided $\varepsilon$ in Def.~\ref{maindef} does not exceed: 
In words, Thm.~\ref{DeterministicUniquenessTheorem} says that the smaller the regularity $r$ of $\mathcal{H}$ or the difference $\overline m - m$ between the assumed and actual number of columns in the latent dictionary, the more columns and coefficients of the original dictionary $\mathbf{A}$ and sparse codes $\mathbf{x}_i$ are guaranteed to be contained (up to noise) in the appropriately scaled recovered dictionary $\mathbf{B}$ and codes $\mathbf{\overline x}_i$, respectively. In the particular case when $\overline m = m$, the theorem directly implies that  $Y = \{\mathbf{Ax}_1, \ldots, \mathbf{Ax}_N\}$ has a stable $k$-sparse representation in $\mathbb{R}^m$, with inequalities \eqref{def1} guaranteed for $\varepsilon$ in Def.~\ref{maindef} given by: 
\begin{align}\label{epsdel}
\varepsilon(\delta_1, \delta_2) := \min \left\{ \frac{\delta_1}{ C_1 }, \frac{ \delta_2 L_{2k}(\mathbf{A})}{ 1 + C_1 \left( \delta_2 + \max_{i \in [N]} \|\mathbf{x}_i\|_1  \right) } \right\}.
\end{align}

We claim that the assumptions of Thm.~\ref{DeterministicUniquenessTheorem} are easily met with deterministic constructions. In particular, sparse codes in general linear position are straightforward to produce using a ``Vandermonde'' matrix construction (i.e. use the columns of the matrix $[\gamma_{i}^j]_{i,j=1}^{k,N}$, for distinct nonzero $\gamma_i$).   % (Prob.~\ref{InverseProblem}).
%=======
%An immediate practical implication of this result is that there exists a practical procedure to affirm if one's proposed solution $(\mathbf{B}, \mathbf{\overline x}_1, \ldots, \mathbf{x}_N)$ to Prob.~\ref{InverseProblem} is indeed unique (up to noise and inherent ambiguities): simply check that $\mathbf{B}$ and the $\mathbf{\overline x}_i$ satisfy the assumptions on $\mathbf{A}$ and the $\mathbf{x}_i$ in Thm.~\ref{DeterministicUniquenessTheorem}.
%
%%In fact, a more general result (stated clearly in the next section) can be gleaned from our method of proving Thm.~\ref{DeterministicUniquenessTheorem}. Briefly, in cases where $\mathbf{B}$ has $\overline m \neq m$ columns, or $\mathcal{H}$ is not regular or only partially satisfying the SIP, a relation between $\overline m$ and the degree sequence of nodes in $\mathcal{H}$ gives indices $J \subseteq [m]$ defining a submatrix $\mathbf{A}_J$ and subvectors $\mathbf{x}_i^J$ that are recoverable in the sense of \eqref{Cstable} and \eqref{b-PDa}. For example, if $\mathcal{H}$ is $\ell$-regular with the SIP but $m \leq \overline m < m\ell/(\ell - 1)$ then we have nonzero $|J| = \overline m - \ell(\overline m - m)$. The implication here is that the smaller the difference $\overline m - m$, the more columns and code entries of the original $n \times m$ dictionary $\mathbf{A}$ and codes $\mathbf{x}_i$ contained (up to noise) in the appropriately scaled $n \times \overline m$ dictionary $\mathbf{B}$ and codes $\mathbf{\overline x}_i$. When $\overline m = m$, we recover Thm.~\ref{DeterministicUniquenessTheorem}.
%
%%In fact, even if $\mathcal{H}$ is not regular or only partially satisfies the SIP, a relation between $\overline m$ and the degree sequence of nodes in $\mathcal{H}$ may give the indices $J \subseteq [m]$. For sake of brevity, we delay to the next section a clear statement of this more general result.
%
%Regarding the assumptions of Thm.~\ref{DeterministicUniquenessTheorem}, it so happens that sparse codes $\mathbf{x}_i$ in general linear position are straightforward to produce with a ``Vandermonde'' matrix construction \cite{Hillar15}, leading to the following.
%>>>>>>> d1407a275dc391dabad7bb62d3e659f9a4ac4624

\begin{corollary}\label{DeterministicUniquenessCorollary}
Given any regular hypergraph $\mathcal{H} \subseteq {[m] \choose k}$ with the SIP, there are $N =  |\mathcal{H}| \left[ (k-1){m \choose k} + 1  \right]$ vectors \mbox{$\mathbf{x}_1, \ldots, \mathbf{x}_N \in \mathbb{R}^m$} such that every matrix $\mathbf{A}$ with $L_{2\mathcal{H}}(\mathbf{A}) > 0$ generates a dataset $Y = \{\mathbf{A}\mathbf{x}_1, \ldots, \mathbf{A}\mathbf{x}_N\}$ with a stable $k$-sparse representation in $\mathbb{R}^m$, with $\varepsilon(\delta_1,\delta_2)$ as in \eqref{epsdel}.
\end{corollary}
% Needs to be L_{2k}(A) > 0 to guarantee sparse vector recovery	

%We also have the following refinement of a result in \cite{Hillar15}:
%\begin{corollary}
%Those square matrices $\mathbf{A}$ that satisfy \mbox{$L_{2\mathcal{H}}(\mathbf{A}) > 0$} for some regular $k$-uniform hypergraph $\mathcal{H}$ with the SIP and have the property that $\mathbf{Ax}$ is $k$-sparse for all $k$-sparse $\mathbf{x}$ with support in $\mathcal{H}$ are the matrices $\mathbf{PD}$, where $\mathbf{P}$ and $\mathbf{D}$ run over permutation and invertible diagonal matrices, respectively.
%\end{corollary}

One can also easily verify that for every $k < m$ there are regular $k$-uniform hypergraphs with the SIP besides the obvious $\mathcal{H} = {[m] \choose k}$. For instance, take $\mathcal{H}$ to be the $k$-regular set of consecutive intervals of length $k$ in some cyclic order on $[m]$. In this case, a direct consequence of Cor.~\ref{DeterministicUniquenessCorollary} is rigorous verification of the lower bound \mbox{$N = m(k-1){m \choose k} + m$} for sufficient sample size from the introduction. Special cases allow for even smaller hypergraphs, however. For example, if $k = \sqrt{m}$ then a 2-regular $k$-uniform hypergraph with the SIP can be constructed as the $2k$ rows and columns formed by arranging the elements of $[m]$ into a square grid.

%\cite{li2004analysis, Georgiev05, Aharon06, Hillar15}
We should stress the point here that framing the problem in terms of hypergraphs has allowed us to show, unlike in previous works on the subject, that the matrix $\mathbf{A}$ need not necessarily satisfy \eqref{SparkCondition} to be recoverable from data. As an example, let $\mathbf{A} = [ \mathbf{e}_1, \ldots, \mathbf{e}_5, \mathbf{v}]$ with $\mathbf{v} = \mathbf{e}_1 + \mathbf{e}_3 + \mathbf{e}_5$ and take $\mathcal{H}$ to be all consecutive pairs of indices $1, \ldots ,6$ arranged in cyclic order. Then for $k=2$, the matrix $\mathbf{A}$ fails to satisfy \eqref{SparkCondition} while still satisfying the assumptions of Thm.~\ref{DeterministicUniquenessTheorem}.%, hence guaranteeing \eqref{Cstable} . %(since $\{ \mathbf{A}_1, \mathbf{A}_3, \mathbf{A}_5, \mathbf{A}_6\}$ is not a linearly independent set).
%This weakening allows for a practical (polynomial) amount of data to still guarantee a stable dictionary in the case where the set of sparse supports for the $\mathbf{\overline x}_i$ is known to have a size that grows polynomially in $m$ and $k$.

A practical implication of Thm.~\ref{DeterministicUniquenessTheorem} is the following: there is an effective procedure sufficient to affirm if a proposed solution to Prob.~\ref{InverseProblem} is indeed unique (up to noise and inherent ambiguities). One need simply check that the matrix and codes satisfy the (computable) assumptions of Thm.~\ref{DeterministicUniquenessTheorem} on $\mathbf{A}$ and the $\mathbf{x}_i$. In general, however, there is no efficient such procedure. We defer a brief discussion on this point to the next section.
%Another practical implication of Thm.\ref{DeterministicUniquenessTheorem} is the following: there is an effective procedure sufficient to affirm if a proposed solution $(\mathbf{B}, \mathbf{\overline x}_1, \ldots, \mathbf{x}_N)$ to Prob.~\ref{InverseProblem} is indeed unique (up to noise and inherent ambiguities). One simply checks that $\mathbf{B}$ and the $\mathbf{\overline x}_i$ satisfy the respective assumptions of Thm.~\ref{DeterministicUniquenessTheorem}.  
%. on $\mathbf{A}$ and the $\mathbf{x}_i$, respectively.

%In fact, a more general result (stated clearly in the next section) can be gleaned from our method of proving Thm.~\ref{DeterministicUniquenessTheorem}. Briefly, in cases where $\mathbf{B}$ has $\overline m \neq m$ columns, or $\mathcal{H}$ is not regular or only partially satisfying the SIP, a relation between $\overline m$ and the degree sequence of nodes in $\mathcal{H}$ gives indices $J \subseteq [m]$ defining a submatrix $\mathbf{A}_J$ and subvectors $\mathbf{x}_i^J$ that are recoverable in the sense of \eqref{Cstable} and \eqref{b-PDa}. For example, if $\mathcal{H}$ is $\ell$-regular with the SIP but $m \leq \overline m < m\ell/(\ell - 1)$ then we have nonzero $|J| = \overline m - \ell(\overline m - m)$. The implication here is that the smaller the difference $\overline m - m$, the more columns and code entries of the original $n \times m$ dictionary $\mathbf{A}$ and codes $\mathbf{x}_i$ contained (up to noise) in the appropriately scaled $n \times \overline m$ dictionary $\mathbf{B}$ and codes $\mathbf{\overline x}_i$. When $\overline m = m$, we recover Thm.~\ref{DeterministicUniquenessTheorem}.

%In fact, even if $\mathcal{H}$ is not regular or only partially satisfies the SIP, a relation between $\overline m$ and the degree sequence of nodes in $\mathcal{H}$ may give the indices $J \subseteq [m]$. For sake of brevity, we delay to the next section a clear statement of this more general result.

%=======
%One can also verify that for every $k < m$, there is a regular $k$-uniform hypergraph that satisfies the SIP; for instance, take $\mathcal{H}$ to be the consecutive intervals of length $k$ in some cyclic order on $[m]$, for which Cor.~\ref{DeterministicUniquenessCorollary} implies the lower bound for sample size $N$ from the introduction. In many cases, however, the SIP is achievable with far fewer supports; for example, when $k = \sqrt{m}$, take $\mathcal{H}$ to be the $2k$ rows and columns formed by arranging $[m]$ in a square grid. 
%>>>>>>> d1407a275dc391dabad7bb62d3e659f9a4ac4624

%As mentioned above, there exist $k$-uniform regular hypergraphs $\mathcal{H}$ with the SIP having cardinality $|\mathcal{H}| = m$, 

A less direct consequence of Thm.~\ref{DeterministicUniquenessTheorem} is the following uniqueness and stability guarantee for solutions to Prob.~\ref{SLCopt}, the usual optimization problem of interest for those applying dictionary learning to their data.

\begin{theorem}\label{SLCopt}
Fix a matrix $\mathbf{A}$ and vectors $\mathbf{x}_i$ satisfying the assumptions of Thm.~\ref{DeterministicUniquenessTheorem}, only now with over $(k-1)\left[ {\overline m \choose k} + |\mathcal{H}|k{\overline m \choose k-1}\right]$ vectors supported in general linear position in each $S \in \mathcal{H}$. Every solution to Prob.~\ref{OptimizationProblem} (with $\eta = \varepsilon/2$) satisfies recovery guarantees \eqref{Cstable} and \eqref{b-PDa} when the corresponding bounds on $\eta$ are met.
\end{theorem}

Another extension of Thm.~\ref{DeterministicUniquenessTheorem} can be derived from the following analytic characterization of the spark condition.  Letting $\mathbf{A}$ be the $n \times m$ matrix of $nm$ indeterminates $A_{ij}$, the reader may work out why substituting real numbers for the $A_{ij}$ yields a matrix satisfying \eqref{SparkCondition} if and only if the following polynomial also evaluates to a nonzero number:
\begin{align*}
f(\mathbf{A}) := \prod_{S \in {[m] \choose 2k}} \sum_{S' \in {[n] \choose 2k}} (\det \mathbf{A}_{S',S})^2,
\end{align*}
%
where for any $S' \in {[n] \choose 2k}$ and $S \in {[m] \choose 2k}$, the symbol $\mathbf{A}_{S',S}$ denotes the submatrix of entries $A_{ij}$ with $(i,j) \in S' \times S$.\footnote{The large number of terms in this product is likely necessary given that deciding whether or not a matrix satisfies the spark condition is NP-hard \cite{tillmann2014computational}.}

Since $f$ is analytic, having a single substitution of a real matrix $\mathbf{A}$ satisfy $f(\mathbf{A}) \neq 0$ implies that the zeroes of $f$ form a set of (Borel) measure zero. Such a matrix is easily constructed by adding rows of zeroes to any $\min(2k,m) \times m$ Vandermonde matrix as described previously, so that every sum in the product defining $f$ above is strictly positive. Almost every $n \times m$ matrix with $n \geq \min(2k,m)$ thus satisfies \eqref{SparkCondition}.

%Another extension of Thm.~\ref{DeterministicUniquenessTheorem} arises from the following analytic characterization of the spark condition.  Let $\mathbf{A}$  be the $n \times m$ matrix of $nm$ indeterminates $A_{ij}$. When real numbers are substituted for $A_{ij}$, the resulting matrix satisfies \eqref{SparkCondition} if and only if the following polynomial is nonzero:
%\begin{align*}
%f(\mathbf{A}) := \prod_{S \in {[m] \choose k}} \sum_{S' \in {[n] \choose k}} (\det \mathbf{A}_{S',S})^2,
%\end{align*}
%
%where for any $S' \in {[n] \choose k}$ and $S \in {[m] \choose k}$, the symbol $\mathbf{A}_{S',S}$ denotes the submatrix of entries $A_{ij}$ with $(i,j) \in S' \times S$.   We note that the large number of terms in this product is likely necessary due to the NP-hardness of deciding whether a given matrix $\mathbf{A}$ satisfies the spark condition \cite{tillmann2014computational}.

%Since $f$ is analytic, having a single substitution of a real matrix $\mathbf{A}$ with $f(\mathbf{A}) \neq 0$ necessarily implies that the zeroes of $f$ form a set of measure zero. Fortunately, such a matrix $\mathbf{A}$ is easily constructed by adding rows of zeroes to any $\min(2k,m) \times m$ Vandermonde matrix as described above (so that each term in the product above for $f$ is nonzero). Hence, almost every $n \times m$ matrix with $n \geq \min(2k,m)$ satisfies \eqref{SparkCondition}.

We claim that a similar phenomenon applies to datasets of vectors with a stable sparse representation. Briefly, following the same procedure as in \cite[Sec.~IV]{Hillar15}, for $k < m$ and $n \geq \min(2k, m)$ we may consider the ``symbolic'' dataset $Y = \{\mathbf{A}\mathbf{x}_1,\ldots,\mathbf{A} \mathbf{x}_N\}$ generated by an indeterminate $n \times m$ matrix $\mathbf{A}$ and $m$-dimensional $k$-sparse vectors $\mathbf{x}_1, \ldots, \mathbf{x}_N$ indeterminate within their supports, which form a regular hypergraph $\mathcal{H} \subseteq {[m] \choose k}$ satisfying the SIP. Restricting \mbox{$(k-1){m \choose k} + 1$} indeterminate $\mathbf{x}_i$ to each support in $\mathcal{H}$, and letting $\textbf{M}$ be the $n \times N$ matrix with columns $\mathbf{A}\mathbf{x}_i$, it can be checked that when $f(\mathbf{M}) \neq 0$ for a substitution of real numbers for the indeterminates, all of the assumptions on $\mathbf{A}$ and $\mathbf{x}_i$ in Thm.~\ref{DeterministicUniquenessTheorem} are satisfied; in particular, $\mathbf{A}$ satisfies \eqref{SparkCondition}. We therefore have the following result: 

\begin{theorem}\label{robustPolythm}
%Let $\mathbf{A}$ and $\mathbf{x}_1, \ldots, \mathbf{x}_N$ be an indeterminate $n \times m$ matrix and sequence of $m$-dimensional vectors, respectively. 
There is a polynomial in the entries of $\mathbf{A}$ and the $\mathbf{x}_i$ that evaluates to a nonzero number only when $Y$ has a stable $k$-sparse representation in $\mathbb{R}^m$. In particular, almost all substitutions impart to $Y$ this property.
\end{theorem}

Note moreover that if a set of measure spaces $\{(X_{\ell}, \Sigma_{\ell}, \nu_{\ell})\}_{\ell=1}^p$ has $\nu_{\ell}$ absolutely continuous with respect to $\mu$ for all $\ell \in [p]$, where $\mu$ is the standard Borel measure on $\mathbb{R}$, then the product measure $\prod_{\ell=1}^p \nu_{\ell}$ is absolutely continuous with respect to the standard Borel product measure on $\mathbb{R}^p$. % for space:  (e.g.,  \cite{folland2013real}). 
By Thm.~\ref{robustPolythm}, there is then a polynomial that is nonzero when $Y$ has a stable $k$-sparse representation in $\mathbb{R}^m$; in particular, stability holds almost surely and we have the following:

\begin{corollary}\label{ProbabilisticCor}
If the indeterminate entries of $\mathbf{A}$ and the $\mathbf{x}_i$ are drawn independently from probability measures absolutely continuous with respect to the standard Borel measure, then $Y$ has a stable $k$-sparse representation in $\mathbb{R}^m$ with probability one.
\end{corollary}



Thus, drawing the dictionary and supported sparse coefficients from any continuous probability distribution almost always generates data with a stable sparse representation.

%\cite{rehnsommer2007, rozell2007neurally, ganguli2012compressed, hu2014hebbian}

We close this section with some comments on the optimality of our results.  The linear scaling for $\varepsilon$ in \eqref{epsdel} is essentially optimal (e.g., see \cite{arias2013fundamental}), but a basic open problem remains: how many samples are necessary to determine the sparse coding model? Our results demonstrate that sparse codes $\mathbf{x}_i$ drawn from only a polynomial number of $k$-dimensional subspaces permit stable identification of the generating dictionary $\mathbf{A}$. This lends some legitimacy to the use of the sparse coding model in practice, where data in general are unlikely (if ever) to exhibit the exponentially many possible $k$-wise combinations of dictionary elements required by (to our knowledge) all previously published results. Consequently, if $k$ is held fixed or if the size of the support set of reconstructing codes is known to be polynomial in $\overline m$ and $k$, then a practical (polynomial) amount of data suffices to identify the dictionary.\footnote{In the latter case, a reexamination of the pigeonholing argument in the proof of Thm.~\ref{DeterministicUniquenessTheorem} requires a polynomial number of samples distributed over a polynomial number of supports.} Reasons to be skeptical that this holds in general, however, can be found in \cite{tillmann2014computational, Tillmann15}. Even so, in the next section we discuss how probabilistic guarantees can in fact be made for any number of available samples.


\section{Proofs}\label{DUT} % of Theorems~\ref{DeterministicUniquenessTheorem} and ~\ref{SLCopt}}\label{DUT}

%%%%%%%%%%%%%%
% need to cut maybe: 
%As mentioned in the previous section, Thm.~\ref{DeterministicUniquenessTheorem} is a particular case of a more general result that forgoes the assumption that $\mathcal{H} \subseteq {[m] \choose k}$ is regular and satisfies the SIP. If instead we require only that the stars $\cap \sigma(i)$ intersect at singletons for all $i \leq q$ (assuming that the nodes of $\mathcal{H}$ are labeled in some order of non-increasing degree), we have that $\overline m \geq k|\mathcal{H}| / \deg(1)$ and, provided $\overline m < k|\mathcal{H}| / (\deg(1) - 1)$, the nonempty submatrix $J$ is of size equal to the largest number $p$ satisfying:
%\begin{align}\label{pcond}
%\sum_{i=\ell}^{m} \deg(i) > (\overline m + 1 - \ell) (\deg(\ell) - 1) \ \ \text{for all } \ell \leq p \leq q.
%\end{align}
%Specifically, $J$ contains all nodes of degree exceeding $\deg(p)$ and some subset of those with degreee equal to $\deg(p)$. For the benefit of the reader, we do not prove explicitly this more general result below; it can be discerned from how exactly Lemma \ref{NonEmptyLemma} is incorporated into the proof of Lem.~\ref{MainLemma}.
%%%%%%%%%%%%%%%%


%As mentioned in the previous section, Thm.~\ref{DeterministicUniquenessTheorem} is a particular case of a more general result, which requires a looser set of constraints on the hypergraph $\mathcal{H}$ and which applies to $n \times \overline m$ matrices $\mathbf{B}$ (and $\overline m$-dimensional codes $\mathbf{\overline x}_i$) with $\overline m \neq m$:

%Fix $\overline m$ and suppose the assumptions of Thm.~\ref{DeterministicUniquenessTheorem} hold, only now with the constraints on $\mathcal{H} \subseteq {[m] \choose k}$ being just that $|\cap H(i)| = 1$ for all $i \leq q$ (assuming w.l.o.g. that the nodes of $\mathcal{H}$ are labeled in some order of non-increasing degree), and with more than $(k-1){\overline m \choose k}$ vectors $\mathbf{x}_i$ supported in g.l.p. on each $S \in \mathcal{H}$. Then we must have $\overline m \geq k|\mathcal{H}| / \deg(1)$ and, provided $\overline m < k|\mathcal{H}| / (\deg(1) - 1)$, the guarantee \eqref{Cstable} holds for a submatrix $\mathbf{A}_J$, where $J \subseteq[m]$ is nonempty and of a size equal to the largest number $p$ satisfying:
%\begin{align}\label{pcond}
%\sum_{i=\ell}^{m} \deg(i) > (\overline m + 1 - \ell) (\deg(\ell) - 1) \ \ \text{for all } \ell \leq p \leq q.
%\end{align}
%Specifically, $J$ consists of the union of the set of all nodes of degree exceeding $\deg(p)$ and some subset of those nodes with degrees equal to $\deg(p)$. 

%For the benefit of the reader, we prove below the case where we forgo only the constraint that $\overline m = m$. This yields the implication $\overline m \geq m$ and \eqref{pcond} reduces to $|J| = \overline m - r(\overline m - m)$. The extension to the general result above can be seen by examining how exactly Lemma \ref{NonEmptyLemma} is incorporated into the overall proof. 

% ======== b - PDa =========

We begin our proof of Thm.~\ref{DeterministicUniquenessTheorem} by showing how dictionary recovery \eqref{Cstable} already implies sparse code recovery \eqref{b-PDa} when $\mathbf{A}$ satisfies \eqref{SparkCondition} and \mbox{$\varepsilon < L_{2k}(\mathbf{A}) / C_1$}, assuming for this purpose that $\overline m = m$ without loss of generality. Consider any $2k$-sparse $\mathbf{x} \in \mathbb{R}^m$ and note that $\|\mathbf{x}\|_1 \leq \sqrt{2k}  \|\mathbf{x}\|_2$. By definition of $L_{2k}$ in \eqref{Ldef}, we have:
\begin{align}\label{stuff}
\|\mathbf{x}_i - \mathbf{D}^{-1}\mathbf{P}^{\top}\mathbf{\overline x}_i \|_1 \nonumber
&\leq \frac{\|\mathbf{BPD}(\mathbf{x}_i - \mathbf{D}^{-1}\mathbf{P}^{\top}\mathbf{\overline x}_i)\|_2}{L_{2k}(\mathbf{BPD})} \\ \nonumber
&\leq \frac{\|(\mathbf{BPD} - \mathbf{A})\mathbf{x}_i\|_2 + \|\mathbf{A}\mathbf{x}_i - \mathbf{B}\mathbf{\overline x}_i\|_2}{L_{2k}(\mathbf{BPD})} \\
&\leq \frac{C_1\varepsilon \|\mathbf{x}_i\|_1 + \varepsilon}{L_{2k}(\mathbf{BPD})},
\end{align}
where the term $C_1\varepsilon \|\mathbf{x}\|_1$ in the numerator above follows from the triangle inequality applied to \eqref{Cstable}.

It remains for us to bound the denominator.  By the reverse triangle inequality:
\begin{align*}
\|\mathbf{BPD}\mathbf{x}\|_2 
&\geq | \|\mathbf{A}\mathbf{x}\|_2 - \|(\mathbf{A}-\mathbf{BPD})\mathbf{x}\|_2 | \\
&\geq \sqrt{2k} (L_{2k}(\mathbf{A}) -  C_1\varepsilon) \|\mathbf{x}\|_2,
\end{align*}
%
wherein removal of the absolute value is justified because of $C_1\varepsilon < L_{2k}(\mathbf{A})$. We therefore have that $L_{2k}(\mathbf{BPD}) \geq L_{2k}(\mathbf{A}) - C_1\varepsilon  > 0$, and \eqref{b-PDa} then follows from \eqref{stuff}.


The heart of the matter is therefore \eqref{Cstable}, which we now establish beginning with the important special case of $k = 1$. 

\begin{proof}[Proof of Thm.~\ref{DeterministicUniquenessCorollary} for $k=1$]
Since the only 1-uniform hypergraph with the SIP is $[m]$, which is obviously regular, we require only $\mathbf{x}_i = c_i \mathbf{e}_i$ for $i \in [m]$, with $c_i \neq 0$ to guarantee  linear independence. While we have yet to define $C_1$ in general, in this case we may set $C_1 = 1/ \min_{\ell \in [m]} |c_{\ell}|$ so that $\varepsilon < L_2(\mathbf{A})  \min_{\ell \in [m]} |c_{\ell}|$. 

Fix $\mathbf{A} \in \mathbb{R}^{n \times m}$ satisfying $L_2(\mathbf{A}) > 0$, since here we have $2\mathcal{H} = {[m] \choose 2}$, and suppose that for some $\mathbf{B}$ and $1$-sparse $\mathbf{\overline x}_i \in \mathbb{R}^{\overline m}$ we have  $\|\mathbf{A}\mathbf{x}_i - \mathbf{B}\mathbf{\overline x}_i\|_2 \leq \varepsilon < L_2(\mathbf{A}) / C_1$ for all $i$. Then there exist $\overline{c}_1, \ldots, \overline{c}_m \in \mathbb{R}$ and a map $\pi: [m] \to [\overline m]$ such that:
\begin{align}\label{1D}
\|c_j\mathbf{A}_j - \overline{c}_j\mathbf{B}_{\pi(j)}\|_2 \leq \varepsilon,\ \ \text{for $j \in [m]$}.
\end{align} 
Note that $\overline{c}_j \neq 0$, since otherwise we would reach the contradiction $\|\mathbf{A}_j \|_2 \leq C_1 |c_j| \|\mathbf{A}_j \|_2  \leq C_1\varepsilon < L_2(\mathbf{A}) \leq L_1(\mathbf{A}) = \min_{j \in [m]} \|\mathbf{A}_{j}\|_2$. %implies by \eqref{delrho} the contradiction $|c_j| < \min_{\ell \in [m]} | c_\ell |$.

We now show that $\pi$ is injective (in particular, a permutation if $\overline m = m$). Suppose that $\pi(i) = \pi(j) = \ell$ for some $i \neq j$ and $\ell$. Then $\|c_{j}\mathbf{A}_{j} - \overline{c}_{j}\mathbf{B}_{\ell}\|_2 \leq \varepsilon$ and $\|c_{i}\mathbf{A}_{i} - \overline{c}_{i} \mathbf{B}_{\ell}\|_2  \leq \varepsilon$, and we have: %Scaling and summing these inequalities by $|\overline{c}_{i}|$ and $|\overline{c}_{j}|$, respectively, and applying the triangle inequality to annihilate the terms in $\mathbf{B}_\ell$, we obtain:
\begin{align*}
(|\overline{c}_{i}| + |\overline{c}_{j}|) \varepsilon
&\geq |\overline{c}_{i}| \|c_{j}\mathbf{A}_{j} - \overline{c}_{j}\mathbf{B}_{\ell}\|_2  + |\overline{c}_{j}| \|c_{i}\mathbf{A}_{i} - \overline{c}_{i} \mathbf{B}_{\ell}\|_2 \nonumber \\
&\geq \|\mathbf{A}(\overline{c}_{i}c_{j} \mathbf{e}_{j} - \overline{c}_{j}c_{i}\mathbf{e}_{i})\|_2 \nonumber \\ 
&\geq  L_2(\mathbf{A}) \|\overline{c}_{i}c_{j} \mathbf{e}_{j} - \overline{c}_{j}c_{i}\mathbf{e}_{i}\|_2 \nonumber \\
&\geq  L_2(\mathbf{A}) \left( |\overline{c}_{i}| + |\overline{c}_{j}| \right) \min_{\ell \in [m]} |c_\ell |,
\end{align*}
contradicting our assumed upper bound on $\varepsilon$. Hence, the map $\pi$ is injective and it must be the case that $\overline m \geq m$. %Setting $\overline J = \pi([m])$ and 
Letting $\mathbf{P}$ and $\mathbf{D}$ be the $\overline m \times \overline m$ permutation and invertible diagonal matrices with, respectively, columns $\mathbf{e}_{\pi(j)}$ and $\frac{\overline{c}_j}{c_j}\mathbf{e}_j$ for $j \in [m]$ (and, say, $\mathbf{e}_j$ otherwise), we may rewrite \eqref{1D} to see that for all $j \in [m]$:
\begin{align*}
\|\mathbf{A}_j - \mathbf{BPD}_j\|_2 
= \|\mathbf{A}_j - \frac{\overline{c}_j}{c_j}\mathbf{B}_{\pi(j)}\|_2 
\leq \frac{\varepsilon}{|c_j|} 
\leq C_1\varepsilon.
\end{align*}
%\vspace{-.2 cm}
\end{proof}

An extension of the proof to the general case $k < m$ requires that we introduce some additional tools which we have used to derive  our general expression \eqref{Cdef1} for the constant $C_1$. These including a generalized notion of distance (Def.~\ref{dDef}) and angle (Def.~\ref{FriedrichsDefinition}) between subspaces as well as a stability result in combinatorial matrix analysis (Lem.~\ref{MainLemma}). 

\begin{definition}\label{dDef}
For $\mathbf{u} \in \mathbb R^m$ and vector spaces $U,V \subseteq \mathbb{R}^m$, let $\text{\rm dist}(\mathbf{u}, V) := \min \{\| \mathbf{u}-\mathbf{v} \|_2: \mathbf{v} \in V\}$ and define:
\begin{align}
d(U,V) := \max_{\mathbf{u} \in U, \ \|\mathbf{u}\|_2 \leq 1} \text{\rm dist}(\mathbf{u},V).
\end{align}
\end{definition}

We note the following facts about $d$. Clearly, 
\begin{align}\label{UsubU}
U' \subseteq U \implies d(U',V) \leq d(U,V).
\end{align}
From \cite[Ch.~4 Cor.~2.6]{Kato2013}, we have: %Kato p.223
\begin{align}\label{dimLem}
d(U,V) < 1 \implies \dim(U) \leq \dim(V),
\end{align}
and from \cite[Lem.~3.2]{Morris10}:
\begin{align}\label{eqdim}
\dim(U) = \dim(V) \implies d(U,V) = d(V,U).
\end{align}

Our result in combinatorial matrix analysis, which hides most of the complexity in our proof of Thm.~\ref{DeterministicUniquenessTheorem}, is the following. 

\begin{lemma}\label{MainLemma}
Suppose the $n \times m$ matrix $\mathbf{A}$ has $L_{2\mathcal{H}}(\mathbf{A}) > 0$ for some $r$-regular $\mathcal{H} \subseteq {[m] \choose k}$ with the SIP. There exists $C_2 > 0$ for which the following holds for all $\varepsilon < L_2(\mathbf{A}) / C_2$:

If for some $n \times \overline m$ matrix $\mathbf{B}$ and map $\pi: \mathcal{H} \mapsto {[\overline m] \choose k}$,
\begin{align}\label{GapUpperBound}
d(\bm{\mathcal{A}}_S, \bm{\mathcal{B}}_{\pi(S)}) \leq \varepsilon \ \  \text{for $S \in \mathcal{H}$},
\end{align}
then $\overline m \geq m$, and provided $\overline m (r-1) < mr$, there is a permutation matrix $\mathbf{P}$ and invertible diagonal $\mathbf{D}$ such that:
\begin{align}\label{MainLemmaBPD}
\|\mathbf{A}_j - \mathbf{B}\mathbf{PD}_j\|_2 \leq C_2 \varepsilon \ \  \text{for } j \in J
\end{align}
for some $J$ of size \mbox{$m - (r-1)(\overline m - m)$}.
\end{lemma}

%\begin{lemma}\label{MainLemma}
%Fix $\mathbf{A} \in \mathbb{R}^{n \times m}$ with $L_\mathcal{H}(\mathbf{A}) > 0$ for some regular $\mathcal{H} \subseteq {[m] \choose k}$ with the SIP. There exists a constant $C_2 > 0$ for which the following holds for all $\varepsilon < L_2(\mathbf{A}) / C_2$:

%If a matrix $\mathbf{B} \in \mathbb{R}^{n \times m}$ and map $\pi: E \mapsto {m \choose k}$ satisfy:
%\begin{align}\label{GapUpperBound}
%d(\text{\rm span}\{\mathbf{A}_{S}\}, \bm{\mathcal{B}}_{\pi(S)}) \leq \varepsilon, \ \ \text{for $S \in \mathcal{H}$},
%\end{align}
%then there exist a permutation matrix $\mathbf{P}$ and invertible diagonal matrix $\mathbf{D}$ such that:
%\begin{align}\label{MainLemmaBPD}
%\|\mathbf{A}_j - \mathbf{BPD}_j\|_2 \leq C_2 \varepsilon, \ \  \text{for } j \in [m],
%\end{align}
%\end{lemma}

We present the constant $C_2$ (a function of $\mathbf{A}$ and $\mathcal{H}$) relative to a quantity used in \cite{Deutsch12} to analyze the convergence of the ``alternating projections" algorithm for projecting a point onto the intersection of subspaces. We incorporate this quantity into the following definition, which we refer to in our proof of Lem.~\ref{DistanceToIntersectionLemma} in the Appendix; specifically, we use it to bound the distance between a point and the intersection of subspaces given an upper bound on its distance from each subspace individually.

\begin{definition}\label{FriedrichsDefinition}
For a collection of real subspaces $\mathcal{V} = \{V_i\}_{i=1}^\ell$, define $\xi(\mathcal{V}) := 0$ when $|\mathcal{V}| = 1$, and otherwise:
\begin{align}\label{xi}
\xi^2(\mathcal{V}) := 1 -  \max \prod_{i=1}^{\ell-1} \sin^2  \theta \left(V_i, \cap_{j>i} V_j \right) ,
\end{align} 
%
where the maximum is taken over all ways of ordering 
%\footnote{We modify the quantity in \cite{Deutsch12} in this way since the subspace ordering is irrelevant to our purpose.} 
the $V_i$ and the angle $\theta \in (0,\frac{\pi}{2}]$ is defined implicitly as \cite[Def.~9.4]{Deutsch12}:
\begin{align*}
\cos{\theta(U,W)} := \max\left\{ |\langle \mathbf{u}, \mathbf{w} \rangle|: \substack{ \mathbf{u} \in U \cap (U \cap W)^\perp, \ \|\mathbf{u}\|_2 \leq 1 \\ \mathbf{w} \in W \cap (U \cap W)^\perp, \  \|\mathbf{w}\|_2 \leq 1 } \right\}.
\end{align*}
\end{definition}
Note that $\theta \in (0,\frac{\pi}{2}]$ implies $0 \leq \xi < 1$, and that $\xi(\mathcal{V}') \leq \xi(\mathcal{V})$ when $\mathcal{V}' \subseteq \mathcal{V}$.\footnote{We acknowledge the potentially counter-intuitive property that $\theta =  \pi/2$ when $U \subseteq W$.}  

The constant $C_2 > 0$ of Lem.~\ref{MainLemma} is then given by:  
\begin{align}\label{Cdef2}
	C_2(\mathbf{A}, \mathcal{H}) := \frac{ (r+1) \max_{j \in [m]} \|\mathbf{A}_j\|_2}{ 1- \max_{\mathcal{G} \in {\mathcal{H} \choose r+1}} \xi( \bm{\mathcal{A}}_\mathcal{G} ) }.
\end{align}

We now define the constant $C_1 > 0$ of Thm.~\ref{DeterministicUniquenessTheorem} in terms of $C_2$. Given vectors $\mathbf{x}_1, \ldots, \mathbf{x}_N \in \mathbb{R}^m$, let $\mathbf{X}$ denote the $m \times N$ matrix with columns $\mathbf{x}_i$ and let $I(S)$ denote the set of indices $i$ for which $\mathbf{x}_i$ is supported in $S$.
\begin{align}\label{Cdef1}
C_1(\mathbf{A}, \mathcal{H}, \{\mathbf{x}_i\}_{i=1}^N) := \frac{ C_2(\mathbf{A}, \mathcal{H}) } { \min_{S \in \mathcal{H}} L_k(\mathbf{AX}_{I(S)}) }.
\end{align}
We remark that given the assumptions of Thm.~\ref{DeterministicUniquenessTheorem} on $\mathbf{A}$ and the $\mathbf{x}_i$, this expression for $C_1$ is well-defined\footnote{\label{note1}Fixing $S \in \mathcal{H}$ and $k$-sparse $\mathbf{c}$, we have $\|\mathbf{AX}_{I(S)}\mathbf{c}\|_2 \geq \sqrt{k} L_\mathcal{H}(\mathbf{A})\|\mathbf{X}_{I(S)}\mathbf{c}\|_2 \geq k L_\mathcal{H}(\mathbf{A}) L_k(\mathbf{X}_{I(S)})\|\mathbf{c}\|_2$. Hence, $L_k(\mathbf{AX}_{I(S)}) \geq \sqrt{k} L_\mathcal{H}(\mathbf{A}) L_k(\mathbf{X}_{I(S)}) > 0$, since $L_{\mathcal{H}}(\mathbf{A}) \geq L_{2\mathcal{H}}(\mathbf{A})> 0$ and $L_k(\mathbf{X}_{I(S)}) > 0$ by general linear position of the $\mathbf{x}_i$.} and yields an upper bound on $\varepsilon$ consistent with that proven sufficient in the case $k=1$ considered at the beginning of this section.\footnote{When $\mathbf{x}_i = c_i\mathbf{e}_i$ we have $C_2 \geq 2\|\mathbf{A}_i\|_2$ and the denominator in \eqref{Cdef1} becomes $\min_{i \in [m]} |c_i| \|\mathbf{A}_i\|_2$, hence $C_1 \geq 2 / \min_{i \in [m]} |c_i|$.}
% C_2(\mathbf{A}, \mathcal{H}) := \frac{ (r+1) \max_{j \in [m]} \|\mathbf{A}_j\|_2}{ \min_{\mathcal{G} \in {\mathcal{H} \choose r} \cup {\mathcal{H} \choose r+1}} \xi( \{ \bm{\mathcal{A}}_S \}


The practically-minded reader should note that the explicit constants $C_1$ and $C_2$ are effectively computable: the quantity $L_k$ may be calculated as the smallest singular value of a certain matrix, while the quantity $\xi$ involves computing ``canonical angles'' between subspaces, which reduce again to an efficient singular value decomposition. There is no known fast computation of $L_k$ in general, however, since even $L_{k} > 0$ is NP-hard \cite{tillmann2014computational}, although 
%efficiently computable bounds have been proposed (e.g. the ``mutual coherence" of a matrix \cite{donoho2003optimally}); alternatively, 
fixing $k$ yields polynomial complexity. Moreover, calculating $C_2$ requires an exponential number of queries to $\xi$ unless $r$ is held fixed, too (e.g., the ``cyclic order'' hypergraphs described above have $r=k$).  Thus, as presented, $C_1$ and $C_2$ are not efficiently computable in general, but efficient sub-cases do exist and may be of practical relevance.

\begin{proof}[Proof of Thm.~\ref{DeterministicUniquenessCorollary} for $k < m$] 
We find a map $\pi: \mathcal{H} \to {[m] \choose k}$ for which the distance $d(\bm{\mathcal{A}}_S, \bm{\mathcal{B}}_{\pi(S)})$ is controlled by $\varepsilon$ for all $S \in \mathcal{H}$. Applying Lem.~\ref{MainLemma} then completes the proof.
%We shall show that for every $S \in \mathcal{H}$ there is some $\overline S \in {[\overline m] \choose k}$ for which the distance $d(\bm{\mathcal{A}}_S, \bm{\mathcal{B}}_{\overline S})$ is controlled by $\varepsilon$. Applying Lem.~\ref{MainLemma} with the map $\pi$ defined by $S \mapsto \overline S$ then completes the proof.

Fix $S \in \mathcal{H}$. Since there are more than $(k-1){\overline m \choose k}$ vectors $\mathbf{x}_i$ supported in $S$, by the pigeonhole principle there must be some $\overline S \in {[\overline m] \choose k}$ and a set of $k$ indices $K \subseteq I(S)$ for which all $\mathbf{\overline x}_i$ with $i \in K$ are supported in $\overline S$.
It also follows\footnotemark[\ref{note1}] from $L_{2\mathcal{H}}(\mathbf{A}) > 0$ and the general linear position of the $\mathbf{x}_i$ that $L_k(\mathbf{AX}_{K}) > 0$; that is, the columns of the $n \times k$ matrix $\mathbf{AX}_K$ form a basis for $\bm{\mathcal{A}}_S$. 

Fixing $\mathbf{y} \in \bm{\mathcal{A}}_S \setminus \{\mathbf{0}\}$, there then exists $\mathbf{c} = (c_1, \ldots, c_k) \in \mathbb{R}^k \setminus \{\mathbf{0}\}$ such that $\mathbf{y} = \mathbf{AX}_K\mathbf{c}$. Setting \mbox{$\mathbf{\overline{y}} = \mathbf{B\overline{X}}_K\mathbf{c}$, which is in $\bm{\mathcal{B}}_{\overline S}$}, we have by triangle inequality:
\begin{align*}
\|\mathbf{y} - \mathbf{\overline{y}}\|_2 
%&= \|\sum_{i=1}^k c_i(\mathbf{AX}_K - \mathbf{B\overline{X}}_K)_i\|_2
&= \|(\mathbf{AX}_K - \mathbf{B\overline{X}}_K)\mathbf{c}\|_2
%\leq \varepsilon \sum_{i=1}^k |c_i| \\
\leq \varepsilon \|\mathbf{c}\|_1
\leq \varepsilon \sqrt{k}  \|\mathbf{c}\|_2  \\
&\leq \frac{\varepsilon}{L_k(\mathbf{AX}_K)} \|\mathbf{y}\|_2,
\end{align*}
where the last inequality follows directly from \eqref{Ldef}. From Def.~\ref{dDef}:
\begin{align}\label{rhs222}
d(\bm{\mathcal{A}}_S, \bm{\mathcal{B}}_{\overline S}) 
\leq \frac{\varepsilon}{  L_k(\mathbf{AX}_{K}) } 
\leq \frac{\varepsilon}{  L_k(\mathbf{AX}_{I(S)}) } 
\leq \varepsilon \frac{C_1}{C_2}.
\end{align}
%
Now, apply Lem.~\ref{MainLemma} with $\varepsilon < L_2(\mathbf{A})/C_1$ and $\pi(S) := \overline S$. % L_2(A) > 0 by def of L, since L_2H(A) > 0 and H covers [m]
\end{proof}

Before moving on to our proof of Thm.~\ref{SLCopt}, we briefly revisit here our discussion on sample complexity from the end of the previous section. While an exponential number of samples may very well prove to be necessary in the deterministic or almost-certain case, our proof of Thm.~\ref{DeterministicUniquenessTheorem} can be extended to hold with some probability for \emph{any} number of samples by alternative appeal to a probabilistic pigeonholing at the point early in the proof where the (deterministic) pigeonhole principle is applied to show that for every $S \in \mathcal{H}$, there exist $k$ vectors $\mathbf{x}_i$ supported on $S$ whose corresponding $\mathbf{\overline x}_i$ all share the same support.\footnote{A famous example of such an argument is the counter-intuitive ``birthday paradox", which demonstrates that the probability of two people having the same birthday in a room of twenty-three is in fact greater than 50\%.} Given insufficient samples, this argument has some less-than-certain probability of being valid for each $S \in \mathcal{H}$. Nonetheless, cursory simulations with small hypergraphs confirm that the probability of success very quickly approaches one once the number of samples $N$ surpasses a small fraction of the deterministic sample complexity. 

\begin{proof}[Proof of Thm.~\ref{SLCopt}]
We bound the number of $k$-sparse $\mathbf{\overline x}_i$ from below and then apply Thm.~\ref{DeterministicUniquenessCorollary}. 
%We now apply this fact to bound the number of $k$-sparse $\mathbf{\overline x}_i$. 
Let $n_p$ be the number of $\mathbf{\overline x}_i$ with $\|\mathbf{\overline x}_i\|_0 = p$.
Since the $\mathbf{x}_i$ are all $k$-sparse, by \eqref{minsum} we have:
\begin{align*}
%\mbox{$k \sum_{p = 0}^{\overline m} n_p \geq \sum_{i=0}^N \|\mathbf{x}_i\|_0 \geq \sum_{i=0}^N \|\mathbf{\overline x}_i\|_0 = \sum_{p=0}^{\overline m} p n_p.$}
\sum_{p=0}^{\overline m} p n_p =  \sum_{i=0}^N \|\mathbf{\overline x}_i\|_0
\leq \sum_{i=0}^N \|\mathbf{x}_i\|_0 
\leq kN
\end{align*}
Since $N = \sum_{p = 0}^{\overline m} n_p$, we then have $\sum_{p = 0}^{\overline m} (p-k) n_p \leq 0$. Splitting the sum yields:
\begin{align}\label{eqn}
\sum_{p = k+1}^{\overline m} n_p \leq \sum_{p = k+1}^{\overline m} (p-k) n_p \leq \sum_{p = 0}^k (k-p)n_p \leq k \sum_{p = 0}^{k-1} n_p,
\end{align}
%
demonstrating that the number of vectors $\mathbf{\overline x}_i$ that are \emph{not} $k$-sparse is bounded above by how many are $(k-1)$-sparse. 

We note that no more than $(k-1)|\mathcal{H}|$ of the $\mathbf{\overline x}_i$ share a support $\overline S$ of size less than $k$. Otherwise, by the pigeonhole principle, there is some $S \in \mathcal{H}$ and a set of $k$ indices $K \subseteq I(S)$ for which all $\mathbf{x}_i$ with $i \in K$ are supported in $S$; as argued previously, \eqref{rhs222} follows. It is simple to show that $L_2(\mathbf{A}) \leq \max_j\|\mathbf{A}_j\|_2$, and since $0 \leq \xi < 1$, the right-hand side of \eqref{rhs222} is less than one for $\varepsilon < L_2(\mathbf{A})/C_1$. Thus, by \eqref{dimLem} we would have the contradiction $k = \dim(\bm{\mathcal{A}}_S) \leq \dim(\bm{\mathcal{B}}_{\overline S}) \leq |\overline S| < k.$ 

The total number of $(k-1)$-sparse vectors $\mathbf{\overline x}_i$ thus cannot exceed $|\mathcal{H}|(k-1){ \overline m \choose k-1}$. By \eqref{eqn}, no more than $|\mathcal{H}|k(k-1){ \overline m \choose k-1}$ vectors $\mathbf{\overline x}_i$ are not $k$-sparse. Since for every $S \in \mathcal{H}$ there are over $(k-1)\left[ {\overline m \choose k} + |\mathcal{H}|k{ \overline m \choose k-1} \right]$ vectors $\mathbf{x}_i$ supported there, it must be that more than $(k-1){\overline m \choose k}$ of them have corresponding $\mathbf{\overline x}_i$ that are $k$-sparse. The result now follows from Thm.~\ref{DeterministicUniquenessCorollary}, noting by the triangle inequality that $\|\mathbf{A}\mathbf{x}_i - \mathbf{B}\mathbf{\overline x}_i\| \leq 2\eta$ for $i = 1, \ldots, N$.
\end{proof}

%\begin{proof}[Proof (sketch) of Thm.~\ref{robustPolythm}]
%Let $\textbf{M}$ be the matrix with columns $\mathbf{A}\mathbf{x}_i$, $i \in [N]$.  Consider the polynomial $\prod_{S \in {[N] \choose 2k}} \sum_{S' \in {[n] \choose 2k}} (\det \textbf{M}_{S',S})^2$ in the indeterminate entries of $\mathbf{A}$ and $\mathbf{x}_i$, with notation as in Sec.~\ref{Results}.  
%It can be checked that when this polynomial is nonzero for a substitution of real numbers for the indeterminates, all of the genericity requirements on $\mathbf{A}$ and $\mathbf{x}_i$ in our proofs of stability in Thm.~\ref{DeterministicUniquenessTheorem} are satisfied (in particular, the spark condition \eqref{SparkCondition} on $\mathbf{A}$).
%\end{proof}

%\begin{proof}[Proof (sketch) of Cor.~\ref{ProbabilisticCor}]
%First, note that if a set of measure spaces $\{(X_{\ell}, \Sigma_{\ell}, \nu_{\ell})\}_{\ell=1}^p$ has that $\nu_{\ell}$ is absolutely continuous with respect to $\mu$ for all $\ell \in [p]$, where $\mu$ is the standard Borel measure on $\mathbb{R}$, then the product measure $\prod_{\ell=1}^p \nu_{\ell}$ is absolutely continuous with respect to the standard Borel product measure on $\mathbb{R}^p$. % for space:  (e.g.,  \cite{folland2013real}). 
%By Thm.~\ref{robustPolythm}, there is a polynomial that is nonzero when $Y$ has a stable $k$-sparse representation in $\mathbb{R}^m$; in particular, stability holds almost surely.
%\end{proof}

\section{Discussion}\label{Discussion}

A motivation for this work was the emergence of seemingly characteristic representations from sparse coding models fit to natural data, despite the varied assumptions underlying the many algorithms in current use. To this end, we have taken an important step toward unifying the many publications on the topic by demonstrating general sufficient, deterministic conditions under which identification of parameters in this model is not only possible but also robust to uncertainty in measurement and model choice.

We have shown that, given sufficient data, the problem of seeking a dictionary and sparse codes with minimal average support size (Prob.~\ref{OptimizationProblem}) reduces to an instance of Prob.~\ref{InverseProblem}, to which our main result (Thm.~\ref{DeterministicUniquenessTheorem}) applies: every dictionary and sequence of sparse codes consistent with the data are equivalent up to inherent relabeling/scaling ambiguities and a discrepancy (error) that scales linearly with the measurement noise or modeling inaccuracy. The constants we provide are explicit and computable; as such, there is an effective procedure that sufficiently affirms if a proposed solution to these problems is indeed unique up to noise and inherent ambiguities, although this computation is not efficient in general.

Our theoretical work mathematically justifies one of the few hypothesized theories of bottleneck communication in the brain \cite{Isely10}: that sparse neural population activity is recoverable from its noisy linear compression through a randomly constructed (but unknown) wiring bottleneck by any biologically plausible unsupervised sparse coding method that solves Prob.~\ref{DeterministicUniquenessTheorem} or \ref{SLCopt} (e.g., \cite{rehnsommer2007, rozell2007neurally, pehlevan2015normative}).\footnote{We refer the reader to \cite{ganguli2012compressed} for more on interactions between dictionary learning and neuroscience.}


%A compelling feature of this model is its simple instantiation of the principle of Occam's razor: a natural video, for instance, is modeled as a linear combination of a small number of spatiotemporal building blocks, each representing an archetypical feature latent in the data. 

%For theoretical neuroscience in particular, dictionary learning and related methods have recovered characteristic components of natural images \cite{Olshausen96, hurri1996image, bell1997independent, van1998independent} and sounds \cite{bellsejnowski1996, smithlewicki2006} that reproduce response properties of certain cortical neurons.  

%to the recovery of ``mouse neuronal activity representing location on a track \cite{agarwal2014spatially}

%Moreover, we show that even if the meta-parameter for the number of dictionary elements is overestimated, a subset of parameters may still be identifiable up to noise. 

Beyond an original extension of existing noiseless guarantees \cite{Hillar15} to the noisy regime and their application to Prob.~\ref{OptimizationProblem}, a major innovation in our work is a theory of combinatorial designs for support sets key to the identification of the dictionary. We incorporate this idea into a fundamental lemma in matrix theory (Lem.~\ref{MainLemma}) that draws upon the definition of a new matrix lower bound induced by a hypergraph. Novel insights enabled by our combinatorial approach include: 1) a subset of dictionary elements is recoverable even if dictionary size is overestimated, 2) data require only a polynomial number of distinct sparse supports, and 3) recoverable dictionaries need not necessarily satisfy the spark condition. 

The absence of any assumption at all about dictionaries that solve Prob.~\ref{InverseProblem} was a major technical difficulty in proving Thm.~\ref{DeterministicUniquenessTheorem}. We sought such a general guarantee because of the practical difficulty of ensuring that an algorithm maintain a dictionary satisfying the spark condition \eqref{SparkCondition} at each iteration, an implicit requirement of all previous works except \cite{Hillar15}; indeed, even certifying a dictionary has this property is NP-hard \cite{tillmann2014computational}.

In fact, uniqueness guarantees with minimal assumptions apply to all areas of data science and engineering that utilize learned sparse structure. For example, several groups have applied compressed sensing to signal processing tasks: MRI analysis \cite{lustig2008compressed}, image compression \cite{Duarte08}, and, more recently, the design of an ultrafast camera \cite{Gao14}. Given such effective uses of compressed sensing, it is only a matter of time before these systems incorporate dictionary learning to encode and decode signals (e.g., in a device that learns structure from motion \cite{kong2016prior}), just as scientists have used it to make sense of their data \cite{jung2001imaging, Agarwal14, lee2016sparse, wu2016stability}. Assurances such as those offered by our theorems certify that different devices (with different initialization, etc.) will learn equivalent representations given enough data from statistically identical systems.\footnote{To contrast with the current hot topic of ``Deep Learning'', there are few such uniqueness guarantees for these models of data; moreover, even small noise can dramatically alter their output \cite{goodfellow2014explaining}.} Indeed, it seems a main reason for the sustained interest in dictionary learning as an unsupervised method for data analysis is the assumed well-posedness of parameter identification in the sparse coding model, confirmation of which forms the core of our theoretical findings.

\section{EXTRA}
An additional implication of Thm.~\ref{DeterministicUniquenessTheorem} of potential relevance to Prob.~\ref{OptimizationProblem} is the implied lower bound on $L_{2k}(\mathbf{B})$ in the case where $\mathbf{A}$ satisfies \eqref{SparkCondition}. Due to complexity issues, Prob.~\ref{SLCopt} is often solved by first replacing the $\|\cdot\|_0$-norm with the $\|\cdot\|_1$-norm. The compressive sensing literature informs us that sparse codes are in fact stably recoverable via minimization of the $\|\cdot\|_1$-norm provided $L_{2k}(\mathbf{A})$ is large enough. Thm.~\ref{DeterministicUniquenessTheorem} guarantees that for large enough $L_{2k}(\mathbf{A})$, all valid solutions also have the property that their sparse codes can be inferred by minimization of the $\|\cdot\|_1$ norm. Does this matter..?

\textbf{Acknowledgement.} We thank F. Sommer and D. Rhea for early thoughts, and I. Morris for posting \eqref{eqdim} online.

%Since its inception some twenty years ago, sparse coding has become a standard tool in signal analysis, yielding myriad insights into the structure of natural signals across a variety of scientific domains. In this work, 

%We make a final remark about the tightness of our results and the computability of our derived constants.
%We remark that our constants have been derived for deterministic ``worst-case'' noise, whereas the ``effective'' noise might be smaller when sampled from a distribution.

%Our results suggest that this correspondence could be due to the ``universality'' of sparse representations in natural data, an early idea in neural theory \cite{pitts1947}. 

% \vspace{-.2 cm}

%\showacknow % Display the acknowledgments section

\bibliographystyle{IEEEtran}
\bibliography{chazthm_ieee_trans_sig}

% \vspace{-.2 cm}

\section{Appendix}\label{proofs}

In this section, we prove Lem.~\ref{MainLemma} after stating some required auxiliary lemmas and their proofs.  % stating and proving some auxiliary lemmas. 
%First, note that since $\|\mathbf{A}\mathbf{x}\|_2 \leq \max_j\|\mathbf{A}_j\|_2\|\mathbf{x}\|_1$ and $\|\mathbf{x}\|_1 \leq \sqrt{k} \|\mathbf{x}\|_2$ for $k$-sparse $\mathbf{x}$, by \eqref{Ldef} we have the following frequently applied inequality:
%\begin{align}\label{delrho}
%L_{k'}(\mathbf{A}) \leq  \max_j\|\mathbf{A}_j\|_2 \ \ \text{for all $k' \leq 2k$}.
%\end{align}

% and then sketch the proofs of Thm.~\ref{robustPolythm} and Cor.~\ref{ProbabilisticCor}.

%\begin{lemma}\label{spanIntersectionLemma}
%Let $\mathbf{M} \in \mathbb{R}^{n \times m}$. If every $2k$ columns of $\mathbf{M}$ are linearly independent, then for any $F \subseteq {[m] \choose k}$:
%\begin{align*}
%\text{\rm span}\{\mathbf{M}_{\cap F}\}  = \bigcap_{S \in F} \text{\rm span}\{\mathbf{M}_S\}.
%\end{align*}
%\end{lemma}
%\begin{proof}
%Using induction on $|F|$, it is enough to prove the lemma when $|F| = 2$; but this case follows directly from the assumption.
%\end{proof}

\begin{lemma}\label{spanIntersectionLemma}
If $f: V \to W$ is an injective function, then $f\left(\cap_{i=1}^\ell V_i \right) =  \cap_{i=1}^\ell f\left(V_i\right)$ for any $V_1, \ldots, V_\ell \subseteq V$. ($f(\emptyset):=\emptyset$.)
%In particular, if for some $E \in 2^{[m]}$ the map $M \in \mathbb{R}^{n \times m}$ is injective on $\cup_{S \in \mathcal{H}} \text{span}\{e_i\}_{i \in S}$ then $\text{span}\{ M_{\cap_{S \in \mathcal{H}} S} \} = \cap_{S \in \mathcal{H}} \text{span}\{M_S\}$.
\end{lemma}
\begin{proof}
By induction, it is enough to prove the case $\ell = 2$. Clearly, for any map $f$, if $w \in f(U \cap V)$ then $w \in f(U)$ and $w \in f(V)$, hence $w \in f(U) \cap f(V)$. If $w \in f(U) \cap f(V)$ then $w \in f(U)$ and $w \in f(V)$, hence $w = f(u) = f(v)$ for some $u \in U$ and $v \in V$, implying $u = v$ by injectivity of $f$. Hence $u \in U \cap V$, and $w \in f(U \cap V)$.
\end{proof}
In particular, if a matrix $\mathbf{A}$ satisfies $L_{2\mathcal{H}}(\mathbf{A}) > 0$, then taking $V$ to be the union of subspaces consisting of vectors with supports in $2\mathcal{H}$, we have $\bm{\mathcal{A}}_{\cap \mathcal{G}} = \cap \bm{\mathcal{A}}_\mathcal{G}$ for all $\mathcal{G} \subseteq \mathcal{H}$.
% \vspace{-.4 cm}

\begin{lemma}\label{DistanceToIntersectionLemma}
Let $\mathcal{V} = \{V_i\}_{i=1}^k$ be a set of two or more subspaces of $\mathbb{R}^m$ and let $V = \cap \mathcal{V} $. For  $\mathbf{u} \in \mathbb{R}^m$, we have (recall Defs.~\ref{dDef}~\&~\ref{FriedrichsDefinition}):
\begin{align}\label{DTILeq}
\text{\rm dist}(\mathbf{u}, V) \leq \frac{1}{1 - \xi(\mathcal{V})} \sum_{i=1}^k \text{\rm dist}(\mathbf{u}, V_i).
\end{align}
\end{lemma}
\begin{proof} 
% When $V = \{\mathbf{0}\}$, the result is trivial, so suppose otherwise.  
Recall the projection onto the subspace $V \subseteq \mathbb{R}^m$ is the mapping $\Pi_V: \mathbb{R}^m \to V$ that associates with each $\mathbf{u}$ its unique nearest point in $V$; i.e., $\|\mathbf{u} - \Pi_V\mathbf{u}\|_2 = \text{\rm dist}(\mathbf{u}, V)$.
By repeated triangle inequality,
\begin{align}\label{f}
\|\mathbf{u} - &\Pi_V\mathbf{u}\|_2 
\leq \|\mathbf{u} - \Pi_{V_k} \mathbf{u}\|_2 + \|\Pi_{V_k}  \mathbf{u} - \Pi_{V_k}\Pi_{V_{k-1}}\mathbf{u}\|_2 \nonumber \\
&\ \ \ \ \ \ \ \ \ \ \ + \cdots + \|\Pi_{V_k} \Pi_{V_{k-1}}\cdots \Pi_{V_1} \mathbf{u} - \Pi_V \mathbf{u}\|_2 \nonumber \\
&\leq  \sum_{\ell=1}^k \|\mathbf{u} - \Pi_{V_{\ell}} \mathbf{u}\|_2 
+ \|(\Pi_{V_k}\cdots\Pi_{V_{1}} - \Pi_V) \mathbf{u}\|_2,
\end{align}
where we have also used that the spectral norm of the orthogonal projections $\Pi_{V_{\ell}}$ satisfies $\|\Pi_{V_{\ell}}\|_2 \leq 1$ for all $\ell$. It remains to bound the second term in \eqref{f} by $\xi(\mathcal{V}) \|\mathbf{u} - \Pi_V\mathbf{u}\|_2$. First, note that $\Pi_{V_\ell} \Pi_V = \Pi_V$ and $\Pi_V^2 = \Pi_V$, so we have $\|(\Pi_{V_k} \cdots \Pi_{V_1} - \Pi_V) \mathbf{u} \|_2 
= \| ( \Pi_{V_k} \cdots\Pi_{V_1} - \Pi_V ) (\mathbf{u} - \Pi_V\mathbf{u})\|_2$. % \leq z\|\mathbf{u} - \Pi_V\mathbf{u}\|_2$.
The result \eqref{DTILeq} then follows from the following fact \cite[Thm.~9.33]{Deutsch12}:
\begin{align}
\|\Pi_{V_k}\Pi_{V_{k-1}}\cdots\Pi_{V_1} \mathbf{x} - \Pi_V\mathbf{x}\|_2 \leq z \|\mathbf{x}\|_2 \ \ \text{for all } \mathbf{x},
\end{align}
with \mbox{$z^2= 1 - \prod_{\ell =1}^{k-1}(1-z_{\ell}^2)$} and \mbox{$z_{\ell} = \cos\theta\left(V_{\ell}, \cap_{s=\ell+1}^k V_s\right)$} (recall $\theta$ from Def.~\ref{FriedrichsDefinition}), after substituting $\xi(\mathcal{V})$ for $z$ and rearranging terms.
\end{proof}
\begin{lemma}\label{NonEmptyLemma} 
Fix an $r$-regular hypergraph $\mathcal{H} \subseteq 2^{[m]}$ satisfying the SIP. If the map $\pi: \mathcal{H} \to 2^{[\overline m]}$ has $\sum_{S \in \mathcal{H}} |\pi(S)| \geq \sum_{S \in \mathcal{H}} |S|$ and:
\begin{align}\label{cond}
	|\cap \pi(\mathcal{G})| \leq |\cap \mathcal{G} |,\ \ \   \text{for } \mathcal{G} \in {\mathcal{H} \choose r} \cup {\mathcal{H} \choose r+1},
\end{align}
%
then $\overline m \geq m$, and if $\overline m  (r-1) < mr$ then the map $i \mapsto \cap \pi(\sigma(i))$ is an injective map to $[\overline m]$ from some $J \subseteq [m]$ of size $m - (r-1)(\overline m - m)$.\footnote{Recall $\sigma$ from Def.~\ref{sip}.}  %In particular, if $\overline m = m$ then $\pi$ is induced by a permutation on $[m]$.
\end{lemma}

\begin{proof}
Consider the set: $T_1 := \{(i, S): i \in \pi(S), S \in \mathcal{H}\}$, which numbers $|T_1| = \sum_{S \in \mathcal{H}} |\pi(S)| \geq \sum_{S \in \mathcal{H}} |S| = \sum_{i \in [m]} \deg_\mathcal{H}(i) = mr$ by $r$-regularity of $\mathcal{H}$. Note that $|T_1| \leq \overline m r$; otherwise, pigeonholing the tuples of $T_1$ with respect to their $\overline m$ possible first elements would imply that more than $r$ of the tuples in $T_1$ share the same first element. This cannot be the case, however, since then some $\mathcal{G} \in {\mathcal{H} \choose r+1}$ formed from any $r+1$ of their second elements would satisfy $\cap \pi(\mathcal{G}) \neq 0$, hence $|\cap \mathcal{G}| \neq 0$ by \eqref{cond}, contradicting $r$-regularity of $\mathcal{H}$. Thus, $\overline m \geq m$.

Suppose now that $\overline m (r-1) < mr$, so that $p := mr - \overline m (r-1)$ is positive and $|T_1| \geq \overline m (r - 1) + p$. Pigeonholing $T_1$ into $[\overline m]$ again, there must be at least $r$ tuples in $T_1$ that share some first element; that is, for some $\mathcal{G}_1 \subseteq \mathcal{H}$ of size $|\mathcal{G}_1| \geq r$ we have $|\cap \pi(\mathcal{G}_1)| \geq 1$ and (by \eqref{cond}) $|\cap \mathcal{G}_1| \geq 1$. Since no more than $r$ tuples of $T_1$ can share the same first element, we in fact have $|\mathcal{G}_1| = r$. It follows by $r$-regularity that $\mathcal{G}_1$ is a star of $\mathcal{H}$; hence, $|\cap \mathcal{G}_1| = 1$ by the SIP and $|\cap \pi(\mathcal{G}_1)|  = 1$ by \eqref{cond}.

If $p=1$ then we are done. Otherwise, define $T_2 := T_1 \setminus \{(i,S) \in T_1: i = \cap \pi(\mathcal{G}_1)\}$, which contains $|T_2| = |T_1| - r \geq (\overline m - 1)(r-1) + (p-1)$ ordered pairs having $\overline m - 1$ distinct first indices. Pigeonholing $T_2$ into $[\overline m - 1]$ and repeating the above arguments produces the star $\mathcal{G}_2 \in {\mathcal{H} \choose r}$ with intersect $\cap \mathcal{G}_2$ necessarily distinct (by $r$-regularity) from $\cap \mathcal{G}_1$. Iterating this procedure $p$ times in total yields the stars $\mathcal{G}_i$ for which $\cap\mathcal{G}_i \mapsto \cap \pi(\mathcal{G}_i)$ defines an injective map to $[\overline m]$ from $J = \{\cap \mathcal{G}_1, \ldots, \cap \mathcal{G}_p\} \subseteq [m]$.
\end{proof}

\begin{proof}[Proof of Lem.~\ref{MainLemma}]
%We will show that the bound \eqref{GapUpperBound} trickles through the intersection semi-lattices of $\{\bm{\mathcal{A}}_S\}_{S \in \mathcal{H}}$ and $\{\bm{\mathcal{B}}_{\overline S}\}_{\overline S \in \pi(\mathcal{H})}$ to yield \eqref{MainLemmaBPD} by virtue of the SIP.  
We begin by showing that $\dim(\bm{\mathcal{B}}_{\pi(S)}) = \dim(\bm{\mathcal{A}}_S)$ for all $S \in \mathcal{H}$. Note that since $\|\mathbf{A}\mathbf{x}\|_2 \leq \max_j\|\mathbf{A}_j\|_2\|\mathbf{x}\|_1$ and $\|\mathbf{x}\|_1 \leq \sqrt{k} \|\mathbf{x}\|_2$ for all $k$-sparse $\mathbf{x}$, by \eqref{Ldef} we have $L_2(\mathbf{A}) \leq \max_j\|\mathbf{A}_j\|_2$ and therefore (as $0 \leq \xi < 1$), the right-hand side of \eqref{GapUpperBound} is less than one. From \eqref{dimLem}, we have $|\pi(S)| \geq \dim(\bm{\mathcal{B}}_{\pi(S)}) \geq \dim(\bm{\mathcal{A}}_S) = |S|$, the final equality by injectivity of $\mathbf{A}_S$. As $|\pi(S)| = |S|$, the claim follows. Note, therefore, that $\mathbf{B}_{\pi(S)}$ has full-column rank for all $S \in \mathcal{H}$.

We next demonstrate that \eqref{cond} holds. Fixing $\mathcal{G} \in {\mathcal{H} \choose r} \cup {\mathcal{H} \choose r+1}$, it suffices to show that $d(\bm{\mathcal{B}}_{\cap \pi(\mathcal{G})}, \bm{\mathcal{A}}_{\cap \mathcal{G}} ) < 1$, since by \eqref{dimLem} we then have $|\cap \pi(\mathcal{G})| = \dim(\bm{\mathcal{B}}_{\cap \pi(\mathcal{G})}) \leq \dim(\bm{\mathcal{A}}_{\cap \mathcal{G}}) = |\cap \mathcal{G}|$, with equalities by the full column-ranks of $\mathbf{A}_{S}$ and $\mathbf{B}_{\pi(S)}$ for all $S \in \mathcal{H}$.\footnote{Note that if ever $\bm{\mathcal{B}}_{\cap \pi(\mathcal{G})} \neq \bf 0$ while $\cap \mathcal{G} = \emptyset$, we would have $d(\bm{\mathcal{B}}_{\cap \pi(\mathcal{G})}, \bm 0 ) = 1$. This cannot be the case, however, since the deduction that follows would then lead to a contradiction.} Observe that $d(\bm{\mathcal{B}}_{\cap \pi(\mathcal{G})}, \bm{\mathcal{A}}_{\cap \mathcal{G}}  ) 
\leq d\left( \cap \bm{\mathcal{B}}_{\pi(\mathcal{G})}, \cap \bm{\mathcal{A}}_\mathcal{G} \right)$, by \eqref{UsubU} since trivially $\bm{\mathcal{B}}_{\cap \pi(\mathcal{G})} \subseteq \cap \bm{\mathcal{B}}_{\pi(\mathcal{G})}$ and since $\bm{\mathcal{A}}_{\cap \mathcal{G}} = \cap \bm{\mathcal{A}}_\mathcal{G}$ by Lem.~\ref{spanIntersectionLemma}. Recalling Def.~\ref{dDef} and applying Lem.~\ref{DistanceToIntersectionLemma} yields:
\begin{align}
d\left( \cap \bm{\mathcal{B}}_{\pi(\mathcal{G})}, \cap \bm{\mathcal{A}}_\mathcal{G} \right)
&\leq \max_{\mathbf{u} \in \cap \bm{\mathcal{B}}_{\pi(\mathcal{G})}, \ \|\mathbf{u}\|_2 \leq 1} \sum_{S \in \mathcal{G}} \frac{ \text{\rm dist}\left( \mathbf{u},\bm{\mathcal{A}}_{S} \right) }{ 1 - \xi( \bm{\mathcal{A}}_\mathcal{G} ) } \nonumber \\
&= \sum_{S \in \mathcal{G}} \frac{ d\left( \cap \bm{\mathcal{B}}_{\pi(\mathcal{G})},\bm{\mathcal{A}}_{S} \right) }{ 1 - \xi( \bm{\mathcal{A}}_\mathcal{G} ) }, \nonumber
\end{align}
passing the maximum through the sum.
Since $\cap \bm{\mathcal{B}}_{\pi(\mathcal{G})} \subseteq \bm{\mathcal{B}}_{\pi(S)}$ for all $S \in \mathcal{G}$, by \eqref{UsubU} the numerator of each term in the sum above is bounded by \mbox{$d\left( \bm{\mathcal{B}}_{\pi(S)},\bm{\mathcal{A}}_{S} \right) = d\left(\bm{\mathcal{A}}_{S}, \bm{\mathcal{B}}_{\pi(S)} \right) \leq \varepsilon$}, with equality by \eqref{eqdim} since $\dim(\bm{\mathcal{B}}_{\pi(S)}) = \dim(\bm{\mathcal{A}}_S)$. Thus, altogether:
\begin{align}\label{last}
d(\bm{\mathcal{B}}_{\cap \pi(\mathcal{G})}, \bm{\mathcal{A}}_{\cap \mathcal{G}} )
\leq \frac{|\mathcal{G}| \varepsilon}{1 - \xi( \bm{\mathcal{A}}_\mathcal{G} )}
\leq \frac{C_2 \varepsilon}{\max_j\|\mathbf{A}_j\|_2},
\end{align}
recalling the definition of $C_2$ in \eqref{Cdef2}. Lastly, since $C_2 \varepsilon < L_2(\mathbf{A}) \leq \max_j\|\mathbf{A}_j\|_2$, we have $d(\bm{\mathcal{B}}_{\cap \pi(\mathcal{G})}, \bm{\mathcal{A}}_{\cap \mathcal{G}} ) \leq 1$ and therefore \eqref{cond} holds.

%Keeping in mind that $d(U',V) \leq d(U,V)$ whenever $U' \subseteq U$ and applying (in order) Lem.~\ref{spanIntersectionLemma}, Lem.~\ref{DistanceToIntersectionLemma}, \eqref{eqdim}, \eqref{GapUpperBound}, and \eqref{Cdef2} gives:
%\begin{align}\label{randoml}
%d(&\bm{\mathcal{B}}_{\cap \pi(\mathcal{G})}, \bm{\mathcal{A}}_{\cap \mathcal{G}}  ) 
%\leq d\left( \cap \bm{\mathcal{B}}_{\pi(\mathcal{G})}, \cap \bm{\mathcal{A}}_\mathcal{G} \right)
%\leq \sum_{T \in \mathcal{G}} \frac{ d\left( \cap \bm{\mathcal{B}}_{\pi(\mathcal{G})},\bm{\mathcal{A}}_{T} \right) }{ 1 - \xi( \bm{\mathcal{A}}_\mathcal{G} ) } \nonumber \\
%&\leq \sum_{T \in \mathcal{G}} \frac{ d\left( \bm{\mathcal{B}}_{\pi(T)},\bm{\mathbf{A}}_{T} \right) }{ 1 - \xi( \bm{\mathcal{A}}_\mathcal{G} ) }
%\leq \frac{|\mathcal{G}| \varepsilon}{1 - \xi( \bm{\mathcal{A}}_\mathcal{G} )} 
%\leq \frac{C_2 \varepsilon}{\max_j\|\mathbf{A}_j\|_2},
%\end{align}
%
%which is less than one, since $C_2 \varepsilon < L_2(\mathbf{A}) \leq \max_j\|\mathbf{A}_j\|_2$.

By Lem.~\ref{NonEmptyLemma}, the association $i \mapsto \cap \pi(\sigma(i))$ is an injective map $\overline \pi: J \to [\overline m]$ for some $J \subseteq [m]$ of size $m - (r-1)(\overline m - m)$, and $\mathbf{B}_{\overline \pi(i)} \neq \mathbf{0}$ for all $i \in J$ since we have shown the columns of $\mathbf{B}_{\pi(S)}$ to be linearly independent for all $S \in \mathcal{H}$. Letting $\overline \varepsilon := C_2 \varepsilon / \max_i \|\mathbf{A}_i\|_2$, it follows from \eqref{eqdim} and \eqref{last} that $d\left( \bm{\mathcal{A}}_i, \bm{\mathcal{B}}_{\overline \pi(i)} \right) = d\left(\bm{\mathcal{B}}_{\overline \pi(i)},  \bm{\mathcal{A}}_i \right)  \leq \overline \varepsilon$ for all $i \in J$. %Fixing $\overline \varepsilon = C_2\varepsilon$ and letting $c_i = \|\mathbf{A}_i\|_2^{-1}$, we thus have that for each $\mathbf{e}_i \in \mathbb{R}^m$ with $i \in J$ there exists some $\overline{c}_i \in \mathbb{R}$ such that $\|c_i\mathbf{A}\mathbf{e}_i - \overline{c}_i \mathbf{B}\mathbf{e}_{\overline \pi(i)}\|_2 \leq \overline \varepsilon < L_2(\mathbf{A}) \min_{i\in J} |c_i|$. But this is exactly the supposition in \eqref{1D}, and the result follows from the case $k=1$ in Sec.~\ref{DUT} applied to the submatrix $\mathbf{A}_J$. 
Setting $c_i := \|\mathbf{A}_i\|_2^{-1}$ so that $\|c_i\mathbf{Ae}_i\|_2 = 1$, by Def.~\ref{dDef} we have for all $i \in J$:
\begin{align*}
\min_{\overline c_i \in \mathbb{R}} \|c_i\mathbf{Ae}_i - \overline c_i \mathbf{Be}_{\overline \pi(i)} \|_2
\leq d\left( \bm{\mathcal{A}}_i, \bm{\mathcal{B}}_{\overline \pi(i)} \right)
\leq \overline \varepsilon,
%&= \max_{\bm{u} \in \bm{\mathcal{A}}_i, \|\bm{u}\|_2 \leq 1} \text{dist}\left(\bm{u}, \bm{\mathcal{B}}_{\overline \pi(i)} \right) \\
%&\geq  \text{dist}\left(c_i\bm{Ae}_i, \mathbf{Be}_{\overline \pi(i)} \right) \\
%&= \min_{\overline c_i \in \mathbb{R}} \|c_i\bm{Ae}_i - \overline c_i \mathbf{Be}_{\overline \pi(i)} \|_2
\end{align*}
%
for $\overline \varepsilon < L_2(\mathbf{A})\min_{i \in [m]}|c_i|$. But this is exactly the supposition in \eqref{1D}, with $J$ and $\overline \varepsilon$ in place of $[m]$ and $\varepsilon$, respectively. The same arguments of the case $k=1$ in Sec.~\ref{DUT} can then be made to show that for any $\overline m \times \overline m$ permutation and invertible diagonal matrices $\mathbf{P}$ and $\mathbf{D}$ with, respectively, columns $\mathbf{e}_{\pi(i)}$ and $\frac{\overline{c}_i}{c_i}\mathbf{e}_i$ for $i \in J$ (and, say, $\mathbf{e}_i$ otherwise) we have $\|\mathbf{A}_i - \mathbf{B}\mathbf{PD}_i \|_2 \leq \overline  \varepsilon / |c_i|  \leq C_2 \varepsilon$ for all $i \in J$.
\end{proof}

%\section{Proofs of Thm.~\ref{robustPolythm} and Cor.~\ref{ProbabilisticCor}}

%\clearpage

%\section{$\ell_1$-norm Extension}
%
%\begin{problem}\label{OptimizationProblemL1}
%Find a matrix $\mathbf{B}$ and vectors \mbox{$\mathbf{\overline x}_1, \ldots, \mathbf{\overline x}_N \in \mathbb{R}^m$} that solve:
%\begin{align}\label{minsum}
%\min \sum_{i = 1}^N \|\mathbf{\overline x}_i\|_1 \ \ 
%\text{subject to} \ \ \|\mathbf{z}_i - \mathbf{B}\mathbf{\overline x}_i\|_2 \leq \eta, \ \text{for all $i$}.
%\end{align}
%\end{problem}

%\begin{corollary}\label{SLCopt}
%Fix $\alpha > 0$ and suppose all of the assumptions of Thm.\ref{DeterministicUniquenessTheorem} hold with the following modifications:
%\begin{enumerate}
%\item The coefficients of all $k$-sparse $\mathbf{x}_i$ are drawn from $[-\beta, \beta]$, where $\beta < \frac{\alpha(k+1)}{k}$.
%\item More than $(k-1){\overline m \choose k} + q$ vectors $\mathbf{x}_i$ are supported in each $S \in \mathcal{H}$, where
%\[ q = \left[\frac{\alpha}{\beta}(k+1) - k\right]^{-1}k|\mathcal{H}|(k-1){\overline m \choose k-1}. \]
%\end{enumerate}
%Then all solutions to Prob.~\ref{OptimizationProblemL1} for which the nonzero elements of all $\mathbf{\overline x}_i$ lie outside $[-\alpha, \alpha]$ necessarily satisfy the implications \eqref{Cstable} and \eqref{b-PDa} of Thm.~\ref{DeterministicUniquenessTheorem}.
%\end{corollary}

%\begin{proof}[Proof of Cor.~\ref{SLCopt}]
%We bound the number of $k$-sparse $\mathbf{\overline x}_i$ and then apply Thm.~\ref{DeterministicUniquenessCorollary}. First, observe that no more than $(k-1)|\mathcal{H}|$ of the $\mathbf{\overline x}_i$ share a support $\overline S$ of size less than $k$; otherwise, by the pigeonhole principle, at least $k$ of these indices $i$ belong to the same $K \subseteq I(S)$ for some $S \in \mathcal{H}$ and (as argued previously) \eqref{rhs222} follows. Since the right-hand side of \eqref{rhs222} is less than one, by \eqref{dimLem} we have the contradiction $k = \dim(\bm{\mathcal{A}}_S) \leq \dim(\bm{\mathcal{B}}_{\overline S}) \leq |\overline S|.$ The total number of $(k-1)$-sparse vectors $\mathbf{\overline x}_i$ can thus not exceed $|\mathcal{H}|(k-1){ \overline m \choose k-1}$. 

%Next, observe that the assumptions of the lemma imply:
%\begin{align*}
%\alpha \sum_i \|\mathbf{\overline x}_i\|_0 \leq \sum_i \|\mathbf{\overline x}_i\|_1 \leq \sum_i \|\mathbf{x}_i\|_1 \leq \beta \sum_i \|\mathbf{x}_i\|_0.
%\end{align*}

%We now apply these facts to bound the number of $k$-sparse $\mathbf{\overline x}_i$. Let $n_p$ be the number of $\mathbf{\overline x}_i$ with $\|\mathbf{\overline x}_i\|_0 = p$. Since the $\mathbf{x}_i$ are all $k$-sparse, we have:
%\begin{align*}
%k \sum_{p = 0}^{\overline m} n_p = k N \geq \sum_{i=0}^N \|\mathbf{x}_i\|_0 \geq \frac{\alpha}{\beta} \sum_{i=0}^N \|\mathbf{\overline x}_i\|_0 = \frac{\alpha}{\beta} \sum_{p=0}^{\overline m} p n_p.
%\end{align*}
%Hence, 
%\begin{align*}
%\left[ \frac{\alpha(k+1)}{\beta} - k\right] \sum_{p = k+1}^{\overline m} n_p 
%&\leq \sum_{p = k+1}^{\overline m} (\frac{\alpha}{\beta} p-k) n_p
%\leq \sum_{p = 0}^k (k - \frac{\alpha}{\beta} p)n_p \\
%&\leq k \sum_{p = 0}^k n_p 
%\leq k(1-\frac{\alpha}{\beta})n_k + k|\mathcal{H}|(k-1){ \overline m \choose k-1}
%\end{align*}

%[Shit, we need $\alpha = \beta$.] Therefore, no more than $q$ vectors $\mathbf{\overline x}_i$ are \emph{not} $k$-sparse. Since for every $S \in \mathcal{H}$ there are over $(k-1){\overline m \choose k} + q$ vectors $\mathbf{x}_i$ supported there, it follows that more than $(k-1){\overline m \choose k}$ of them have corresponding $\mathbf{\overline x}_i$ that are $k$-sparse. The result now follows directly from Thm.~\ref{DeterministicUniquenessCorollary}.
%\end{proof}

%\pagebreak

%\begin{corollary}\label{SLCoptL1}
%If all of the assumptions of Thm.\ref{DeterministicUniquenessTheorem} hold, only now with more than $(k-1)\left[ {\overline m \choose k} + |\mathcal{H}|k{\overline m \choose k-1}\right]$ vectors $\mathbf{x}_i$ supported in each $S \in \mathcal{H}$, then every solution to Prob.~\ref{OptimizationProblemL1} for which the average absolute nonzero coefficient of the $\mathbf{\overline x}_i$ is bounded below by that of the $\mathbf{x}_i$ necessarily satisfies the implications \eqref{Cstable} and \eqref{b-PDa} of Thm.~\ref{DeterministicUniquenessTheorem}.
%\end{corollary}

%\begin{proof}[Proof of Cor.~\ref{SLCoptL1}]
%Since the average absolute nonzero coefficient of the $\mathbf{x}_i$ is given by $\alpha = \sum_i \|\mathbf{x}_i\|_1 / \sum_i \|\mathbf{x}_i\|_0$, we have:
%\begin{align*}
%\alpha \sum_i \|\mathbf{\overline x}_i\|_0 \leq \sum_i \|\mathbf{\overline x}_i\|_1 \leq \sum_i \|\mathbf{x}_i\|_1 = \alpha \sum_i \|\mathbf{x}_i\|_0.
%\end{align*}
%
%and the result follows by the same arguments as in the proof of Cor.~\ref{SLCopt}.
%\end{proof}

\end{document}
