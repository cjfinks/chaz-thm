\begin{abstract}
Learning optimal dictionaries for sparse representation modeling has led to the discovery of characteristic sparse features in many natural signals. 
However, universal guarantees of the uniqueness and stability of such features in the presence of noise are lacking. 
This work presents very general conditions guaranteeing when dictionaries yielding the sparsest encodings of a dataset are unique and stable with respect to measurement or modeling error. 
The stability constants are explicit and computable; as such, there is an effective procedure sufficient to affirm if a proposed solution to the dictionary learning problem is unique within bounds commensurate with the noise. 

Two formulations of the dictionary learning problem are considered. The first seeks a dictionary admitting a sparse representation of bounded support size for each point in a dataset. The second seeks a dictionary which minimizes the average induced support size over the dataset. In the first case, beyond the original extension of existing guarantees to the noisy regime, a theory of combinatorial designs for sparse supports is introduced to demonstrate that in almost all cases some or all dictionary elements are recoverable up to an error commensurate with the noise even if the dictionary fails to satisfy the spark condition, its size is overestimated, or data are distributed over only a polynomial number of subspaces spanned by the dictionary. These guarantees are then extended to the second case, becoming the first such guarantees in both in the noiseless and noisy regimes, by demonstrating that the second problem reduces to an instance of the first, given sufficient data. Importantly, in both cases the guarantees apply without imposing any assumptions at all on learned dictionaries beyond a natural upper bound on their size. 

This work serves to justify, in principle, the application of dictionary learning to discover latent sparse structure in data; though much work remains to be done deriving practical criteria for use in applications. 
%The work closes with some open questions and directions for future research, seeded in part by the results of more practically-minded simulations. 
\end{abstract}