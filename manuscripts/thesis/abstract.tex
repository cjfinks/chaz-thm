\begin{abstract}
Learning optimal dictionaries for sparse representation modeling has led to the discovery of characteristic features in many natural signals. 
However, universal guarantees of the uniqueness and stability of such features in the presence of noise are lacking. 
This work presents very general conditions guaranteeing when dictionaries yielding the sparsest encodings of a dataset are unique and stable with respect to measurement or modeling error. 
The stability constants are explicit and computable; as such, there is an effective procedure sufficient to affirm if a proposed solution to the dictionary learning problem is unique within bounds commensurate with the noise. 

The first formulation of the dictionary learning problem to be considered concerns itself with the maximal support size over a dataset.
Beyond the original extension of existing guarantees to the noisy regime, a theory of combinatorial designs for sparse supports is introduced to demonstrate that some or all dictionary elements are recoverable from noisy data even if the dictionary fails to satisfy the spark condition, its size is overestimated, or only a polynomial number of distinct supports appear in the encoded data. 

The second formulation of the dictionary learning problem to be considered concerns itself with the average support size over a dataset. Guarantees are extended to this problem, becoming the first of their kind in both the noise-free and noisy domains, by showing that this problem in fact an instance of the first problem, given sufficient data. Importantly, in all cases the guarantees apply without imposing any assumptions at all on learned dictionaries beyond a natural upper bound on their size. 

This work serves to justify, in principle, the application of dictionary learning to discover latent sparse structure in data, though much work remains to be done deriving criteria that can be used in practice. 
%The work closes with some open questions and directions for future research, seeded in part by the results of more practically-minded simulations. 
\end{abstract}