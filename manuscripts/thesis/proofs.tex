\chapter{Proofs}\label{DUT} % of Theorems~\ref{DeterministicUniquenessTheorem} and ~\ref{SLCopt}}\label{DUT}

\section{Introduction}

%It has been shown here that, given sufficient data, this problem reduces to an instance of Prob.~\ref{InverseProblem} to which the main result (Thm.~\ref{DeterministicUniquenessTheorem}) then applies.


%As mentioned in the previous section, Thm.~\ref{DeterministicUniquenessTheorem} is a particular case of a more general result, which requires a looser set of constraints on the hypergraph $\mathcal{H}$ and which applies to $n \times \overline m$ matrices $\mathbf{B}$ (and $\overline m$-dimensional codes $\mathbf{\overline x}_i$) with $\overline m \neq m$:

%Fix $\overline m$ and suppose the assumptions of Thm.~\ref{DeterministicUniquenessTheorem} hold, only now with the constraints on $\mathcal{H} \subseteq {[m] \choose k}$ being just that $|\cap H(i)| = 1$ for all $i \leq q$ (assuming w.l.o.g. that the nodes of $\mathcal{H}$ are labeled in some order of non-increasing degree), and with more than $(k-1){\overline m \choose k}$ vectors $\mathbf{x}_i$ supported in g.l.p. on each $S \in \mathcal{H}$. Then we must have $\overline m \geq k|\mathcal{H}| / \deg(1)$ and, provided $\overline m < k|\mathcal{H}| / (\deg(1) - 1)$, the guarantee \eqref{Cstable} holds for a submatrix $\mathbf{A}_J$, where $J \subseteq[m]$ is nonempty and of a size equal to the largest number $p$ satisfying:
%\begin{align}\label{pcond}
%\sum_{i=\ell}^{m} \deg(i) > (\overline m + 1 - \ell) (\deg(\ell) - 1) \ \ \text{for all } \ell \leq p \leq q.
%\end{align}
%Specifically, $J$ consists of the union of the set of all nodes of degree exceeding $\deg(p)$ and some subset of those nodes with degrees equal to $\deg(p)$. 

%For the benefit of the reader, we prove below the case where we forgo only the constraint that $\overline m = m$. This yields the implication $\overline m \geq m$ and \eqref{pcond} reduces to $|J| = \overline m - r(\overline m - m)$. The extension to the general result above can be seen by examining how exactly Lemma \ref{NonEmptyLemma} is incorporated into the overall proof. 

% ======== b - PDa =========

% WITH EXPLICIT J SUBSCRIPT
%We begin our proof of Thm.~\ref{DeterministicUniquenessTheorem} by showing how dictionary recovery \eqref{Cstable} already implies sparse code recovery \eqref{b-PDa} when $\mathbf{A}_J$ satisfies \eqref{SparkCondition} and \mbox{$\varepsilon < L_{2k}(\mathbf{A}_J) / C_1$}. %, assuming that $\overline m = m$ without loss of generality. 
%Consider any $2k$-sparse $\mathbf{x} \in \mathbb{R}^m$ and note that $\|\mathbf{x}\|_1 \leq \sqrt{2k}  \|\mathbf{x}\|_2$. By definition of $L_{2k}$ in \eqref{Ldef}, we have:
%\begin{align}\label{stuff}
%\|\mathbf{x}_J - &( \mathbf{D}^{-1}\mathbf{P}^{\top}\mathbf{\overline x} )_J\|_1 \nonumber
%\leq \frac{\|(\mathbf{BPD})_J(\mathbf{x}_J - (\mathbf{D}^{-1}\mathbf{P}^{\top}\mathbf{\overline x})_J)\|_2}{L_{2k}((\mathbf{BPD})_J)} \\ \nonumber
%&\leq \frac{\|( (\mathbf{BPD})_J - \mathbf{A}_J) \mathbf{x}_J\|_2 + \|\mathbf{A}_J\mathbf{x}_J - \mathbf{B}_J\mathbf{\overline x}_J\|_2}{L_{2k}((\mathbf{BPD})_J)} \\
%&\leq \frac{C_1\varepsilon \|\mathbf{x}_J\|_1 + \varepsilon}{L_{2k}((\mathbf{BPD})_J)},
%\end{align}
%where the term $C_1\varepsilon \|\mathbf{x}_J\|_1$ in the numerator above is a consequence of the triangle inequality applied to \eqref{Cstable}.

%It remains for us to bound the denominator.  By the reverse triangle inequality:
%\begin{align*}
%\|(\mathbf{BPD})_J\mathbf{x}_J\|_2 
%&\geq | \|\mathbf{A}_J\mathbf{x}_J\|_2 - \|(\mathbf{A}_J - (\mathbf{BPD})_J)\mathbf{x}_J\|_2 | \\
%&\geq \sqrt{2k} (L_{2k}(\mathbf{A}_J) -  C_1\varepsilon) \|\mathbf{x}_J\|_2,
%\end{align*}
%
%wherein removal of the absolute value is justified since $C_1\varepsilon < L_{2k}(\mathbf{A}_J)$. We therefore have that $L_{2k}((\mathbf{BPD})_J) \geq L_{2k}(\mathbf{A}_J) - C_1\varepsilon  > 0$, and \eqref{b-PDa} then follows from \eqref{stuff}.

% WITHOUT J SUBSCRIPT

It is instructive to begin the proof of Thm.~\ref{DeterministicUniquenessTheorem} by showing how dictionary recovery \eqref{Cstable} already implies sparse code recovery \eqref{b-PDa} when $\mathbf{A}$ satisfies \eqref{SparkCondition} and \mbox{$\varepsilon < L_{2k}(\mathbf{A}) / C_1$}. We shall temporarily assume (without loss of generality) that $\overline m = m$, so as to omit an otherwise requisite subscript $(\cdot)_J$ around certain matrices and vectors. By definition of $L_{2k}$ in \eqref{Ldef}, and noting that $\sqrt{k} \|\mathbf{v}\|_2 \geq \|\mathbf{v}\|_1$ for $k$-sparse $\mathbf{v}$, we have for all $i \in [N[$:
\begin{align}\label{stuff}
\|\mathbf{x}_i - \mathbf{D}^{-1}\mathbf{P}^{\top}\mathbf{\overline x}_i \|_1 \nonumber
&\leq \frac{\|\mathbf{BPD}(\mathbf{x}_i - \mathbf{D}^{-1}\mathbf{P}^{\top}\mathbf{\overline x}_i)\|_2}{L_{2k}(\mathbf{BPD})} \\ \nonumber
&\leq \frac{\|( \mathbf{BPD} - \mathbf{A}) \mathbf{x}_i\|_2 + \|\mathbf{A}\mathbf{x}_i - \mathbf{B}\mathbf{\overline x}_i\|_2}{L_{2k}(\mathbf{BPD})} \\
&\leq \frac{C_1\varepsilon \|\mathbf{x}_i\|_1 + \varepsilon}{L_{2k}(\mathbf{BPD})},
\end{align}
%\begin{align}\label{stuff}
%\frac{ \|\mathbf{x} - \mathbf{D}^{-1}\mathbf{P}^{\top}\mathbf{\overline x} \|_1 } { \|\mathbf{D}^{-1}\|_1 }  \
%&\leq \|\mathbf{Dx} - \mathbf{P}^{\top}\mathbf{\overline x} \|_1 \\ \nonumber
%&\leq \frac{\|\mathbf{BP}(\mathbf{D}\mathbf{x} - \mathbf{P}^{\top}\mathbf{\overline x})\|_2}{L_{2k}(\mathbf{BP})} \\ \nonumber
%&\leq \frac{\|( \mathbf{BPD} - \mathbf{A}) \mathbf{x}\|_2 + \|\mathbf{A}\mathbf{x} - \mathbf{B}\mathbf{\overline x}\|_2}{L_{2k}(\mathbf{BP})} \\
%&\leq \frac{C_1\varepsilon \|\mathbf{x}\|_1 + \varepsilon}{L_{2k}(\mathbf{BP})},
%\end{align}
where the first term in the numerator above follows from the triangle inequality and \eqref{Cstable}.

It remains for us to bound the denominator. For any $2k$-sparse $\mathbf{x}$, we have by the triangle inequality:
\begin{align*}
\|\mathbf{BPD}\mathbf{x}\|_2 
&\geq \|\mathbf{A}\mathbf{x}\|_2 - \|(\mathbf{A} - \mathbf{BPD})\mathbf{x}\|_2 \\
&\geq \sqrt{2k} (L_{2k}(\mathbf{A}) -  C_1\varepsilon) \|\mathbf{x}\|_2,
\end{align*}

%\begin{align*}
%\|\mathbf{BP}\mathbf{x}\|_2 
%&\geq \|\mathbf{A}\mathbf{D}^{-1}\mathbf{x}\|_2 - \|(\mathbf{A} - \mathbf{BPD})\mathbf{D}^{-1}\mathbf{x}\|_2 \\
%&\geq \sqrt{2k} (L_{2k}(\mathbf{A}) -  C_1\varepsilon) \|\mathbf{D}^{-1}\mathbf{x}\|_2, \\ 
%&\geq \sqrt{2k} (L_{2k}(\mathbf{A}) -  C_1\varepsilon) \|\mathbf{D}^{-1}\|_{?} \| \mathbf{x}\|_2,
%\end{align*}
%
We therefore have that $L_{2k}(\mathbf{BPD}) \geq L_{2k}(\mathbf{A}) - C_1\varepsilon  > 0$, and \eqref{b-PDa} then follows from \eqref{stuff}. The reader may also verify that $L_{2k}(\mathbf{BP}) \geq L_{2k}(\mathbf{BPD}) / \|\mathbf{D}\|_1$.

The heart of the matter is therefore \eqref{Cstable}, which we shall now establish beginning with the important special case of $k = 1$. 

\section{Proving the case $k=1$}

Since the only 1-uniform hypergraph with the SIP is $[m]$, which is obviously regular, we require only $\mathbf{x}_i = c_i \mathbf{e}_i$ for $i \in [m]$, with $c_i \neq 0$ to guarantee  linear independence. While we have yet to define $C_1$ generally, in this case we may set $C_1 = 1/ \min_{\ell \in [m]} |c_{\ell}|$ so that $\varepsilon < L_2(\mathbf{A})  \min_{\ell \in [m]} |c_{\ell}|$. 

\begin{proof}[Proof of Thm.~\ref{DeterministicUniquenessCorollary} for $k=1$]

Fix $\mathbf{A} \in \mathbb{R}^{n \times m}$ satisfying $L_2(\mathbf{A}) > 0$, since here we have $2\mathcal{H} = {[m] \choose 2}$, and suppose some $\mathbf{B}$ and $1$-sparse $\mathbf{\overline x}_i \in \mathbb{R}^{\overline m}$ have  $\|\mathbf{A}\mathbf{x}_i - \mathbf{B}\mathbf{\overline x}_i\|_2 \leq \varepsilon < L_2(\mathbf{A}) / C_1$ for all $i$. Then, there exist $\overline{c}_1, \ldots, \overline{c}_m \in \mathbb{R}$ and a map $\pi: [m] \to [\overline m]$ such that:
\begin{align}\label{1D}
\|c_i\mathbf{A}_i - \overline{c}_i\mathbf{B}_{\pi(i)}\|_2 \leq \varepsilon,\ \ \text{for $i \in [m]$}.
\end{align} 
Note that $\overline{c}_i \neq 0$, since otherwise we would reach the following contradiction: $\|\mathbf{A}_i \|_2 \leq C_1 |c_i| \|\mathbf{A}_i \|_2  \leq C_1\varepsilon < L_2(\mathbf{A}) \leq L_1(\mathbf{A}) = \min_{i \in [m]} \|\mathbf{A}_{i}\|_2$. %implies by \eqref{delrho} the contradiction $|c_i| < \min_{\ell \in [m]} | c_\ell |$.

Let us now show that $\pi$ is injective (in particular, a permutation if $\overline m = m$). Suppose that $\pi(i) = \pi(j) = \ell$ for some $i \neq j$ and $\ell$. Then, $\|c_{i}\mathbf{A}_{i} - \overline{c}_{i} \mathbf{B}_{\ell}\|_2  \leq \varepsilon$ and $\|c_{j}\mathbf{A}_{j} - \overline{c}_{j}\mathbf{B}_{\ell}\|_2 \leq \varepsilon$, and we have: %Scaling and summing these inequalities by $|\overline{c}_{i}|$ and $|\overline{c}_{j}|$, respectively, and applying the triangle inequality to annihilate the terms in $\mathbf{B}_\ell$, we obtain:
\begin{align*}
(|\overline{c}_{i}| + |\overline{c}_{j}|) \varepsilon
&\geq |\overline{c}_{i}| \|c_{j}\mathbf{A}_{j} - \overline{c}_{j}\mathbf{B}_{\ell}\|_2  + |\overline{c}_{j}| \|c_{i}\mathbf{A}_{i} - \overline{c}_{i} \mathbf{B}_{\ell}\|_2 \nonumber \\
&\geq \|\mathbf{A}(\overline{c}_{i}c_{j} \mathbf{e}_{j} - \overline{c}_{j}c_{i}\mathbf{e}_{i})\|_2 \nonumber \\ 
&\geq \sqrt{2}  L_2(\mathbf{A}) \|\overline{c}_{i}c_{j} \mathbf{e}_{j} - \overline{c}_{j}c_{i}\mathbf{e}_{i}\|_2 \nonumber \\
&\geq  L_2(\mathbf{A}) \left( |\overline{c}_{i}| + |\overline{c}_{j}| \right) \min_{\ell \in [m]} |c_\ell |,
\end{align*}
contradicting our assumed upper bound on $\varepsilon$. Hence, the map $\pi$ is injective and so $\overline m \geq m$. %Setting $\overline J = \pi([m])$ and 

Letting $\mathbf{P}$ and $\mathbf{D}$ be the $\overline m \times \overline m$ permutation and invertible diagonal matrices with, respectively, columns $\mathbf{e}_{\pi(i)}$ and $\frac{\overline{c}_i}{c_i}\mathbf{e}_i$ for $i \in [m]$ (otherwise, $\mathbf{e}_{i}$ for $i \in [\overline{m}] \setminus [m]$), we may rewrite \eqref{1D} to see that for all $i \in [m]$:
\begin{align*}
\|\mathbf{A}_i - (\mathbf{BPD})_i\|_2 
= \|\mathbf{A}_i - \frac{\overline{c}_i}{c_i}\mathbf{B}_{\pi(i)}\|_2 
\leq \frac{\varepsilon}{|c_i|} 
\leq C_1\varepsilon.
\end{align*}
%\vspace{-.2 cm}
\end{proof}

\section{The main lemma}

An extension of the proof to the general case $k < m$ requires some additional tools to derive the general expression \eqref{Cdef1} for $C_1$. These include a generalized notion of distance (Def.~\ref{dDef}) and angle (Def.~\ref{FriedrichsDefinition}) between subspaces as well as a stability result in combinatorial matrix analysis (Lem.~\ref{MainLemma}), which is the main lemma doing most of the heavy lifting in the proof of Thm.~\ref{DeterministicUniquenessTheorem}. 

\begin{definition}\label{dDef}
For $\mathbf{u} \in \mathbb R^m$ and vector spaces $U,V \subseteq \mathbb{R}^m$, let $\text{\rmfamily dist}(\mathbf{u}, V) := \min \{\| \mathbf{u}-\mathbf{v} \|_2: \mathbf{v} \in V\}$ and define:
\begin{align}
d(U,V) := \max_{\mathbf{u} \in U, \ \|\mathbf{u}\|_2 \leq 1} \text{\rmfamily dist}(\mathbf{u},V).
\end{align}
\end{definition}

Note the following facts about $d$. Clearly, 
\begin{align}\label{UsubU}
U' \subseteq U \implies d(U',V) \leq d(U,V).
\end{align}
From \cite[Ch.~4 Cor.~2.6]{Kato2013}, we also have: %Kato p.223
\begin{align}\label{dimLem}
d(U,V) < 1 \implies \dim(U) \leq \dim(V),
\end{align}
and from \cite[Lem.~3.2]{Morris10}:
\begin{align}\label{eqdim}
\dim(U) = \dim(V) \implies d(U,V) = d(V,U).
\end{align}

The required stability result in combinatorial matrix analysis is the following. 

\begin{lemma}\label{MainLemma}
If an $n \times m$ matrix $\mathbf{A}$ has $L_{2\mathcal{H}}(\mathbf{A}) > 0$ for some $r$-regular $\mathcal{H} \subseteq {[m] \choose k}$ with the SIP, then the following holds for $C_2 > 0$ given by \eqref{Cdef2}:

Fix $\varepsilon < L_2(\mathbf{A}) / C_2$. If for some $n \times \overline m$ matrix $\mathbf{B}$ and map $\pi: \mathcal{H} \mapsto {[\overline m] \choose k}$,
\begin{align}\label{GapUpperBound}
d(\bm{\mathcal{A}}_S, \bm{\mathcal{B}}_{\pi(S)}) \leq \varepsilon, \ \  \text{for $S \in \mathcal{H}$},
\end{align}
then $\overline m \geq m$, and provided $\overline m (r-1) < mr$, there is a permutation matrix $\mathbf{P}$ and invertible diagonal $\mathbf{D}$ such that:
\begin{align}\label{MainLemmaBPD}
\|\mathbf{A}_i - (\mathbf{B}\mathbf{PD})_i\|_2 \leq C_2 \varepsilon, \ \  \text{for } i \in J,
\end{align}
for some $J \subseteq [m]$ of size \mbox{$m - (r-1)(\overline m - m)$}.
\end{lemma}

%\begin{lemma}\label{MainLemma}
%Fix $\mathbf{A} \in \mathbb{R}^{n \times m}$ with $L_\mathcal{H}(\mathbf{A}) > 0$ for some regular $\mathcal{H} \subseteq {[m] \choose k}$ with the SIP. There exists a constant $C_2 > 0$ for which the following holds for all $\varepsilon < L_2(\mathbf{A}) / C_2$:

%If a matrix $\mathbf{B} \in \mathbb{R}^{n \times m}$ and map $\pi: E \mapsto {m \choose k}$ satisfy:
%\begin{align}\label{GapUpperBound}
%d(\text{\rmfamily span}\{\mathbf{A}_{S}\}, \bm{\mathcal{B}}_{\pi(S)}) \leq \varepsilon, \ \ \text{for $S \in \mathcal{H}$},
%\end{align}
%then there exist a permutation matrix $\mathbf{P}$ and invertible diagonal matrix $\mathbf{D}$ such that:
%\begin{align}\label{MainLemmaBPD}
%\|\mathbf{A}_j - \mathbf{BPD}_j\|_2 \leq C_2 \varepsilon, \ \  \text{for } j \in [m],
%\end{align}
%\end{lemma}

The constant $C_2$ (a function of $\mathbf{A}$ and $\mathcal{H}$) will be presented relative to a quantity used in \cite{Deutsch12} to analyze the convergence of the ``alternating projections" algorithm for projecting a point onto the intersection of subspaces. This quantity is incorporated into the following definition, which we shall refer to in the proof of Lem.~\ref{DistanceToIntersectionLemma} in the Appendix; specifically, it will be used to bound the distance between a point and the intersection of subspaces given an upper bound on its distance from each subspace.

\begin{definition}\label{FriedrichsDefinition}
For a collection of real subspaces $\mathcal{V} = \{V_i\}_{i=1}^\ell$, define $\xi(\mathcal{V}) := 0$ when $|\mathcal{V}| = 1$, and otherwise:
\begin{align}\label{xi}
\xi^2(\mathcal{V}) := 1 -  \max \prod_{i=1}^{\ell-1} \sin^2  \theta \left(V_i, \cap_{j>i} V_j \right) ,
\end{align} 
%
where the maximum is taken over all ways of ordering 
%\footnote{We modify the quantity in \cite{Deutsch12} in this way since the subspace ordering is irrelevant to our purpose.} 
the $V_i$ and the angle $\theta \in (0,\frac{\pi}{2}]$ is defined implicitly as \cite[Def.~9.4]{Deutsch12}:
\begin{align*}
\cos{\theta(U,W)} := \max\left\{ |\langle \mathbf{u}, \mathbf{w} \rangle|: \substack{ \mathbf{u} \in U \cap (U \cap W)^\perp, \ \|\mathbf{u}\|_2 \leq 1 \\ \mathbf{w} \in W \cap (U \cap W)^\perp, \  \|\mathbf{w}\|_2 \leq 1 } \right\}.
\end{align*}
\end{definition}
Note that $\theta \in (0,\frac{\pi}{2}]$ implies $0 \leq \xi < 1$, and that $\xi(\mathcal{V}') \leq \xi(\mathcal{V})$ when $\mathcal{V}' \subseteq \mathcal{V}$.\footnote{We acknowledge the counter-intuitive property: $\theta =  \pi/2$ when $U \subseteq W$.}  

The constant $C_2 > 0$ of Lem.~\ref{MainLemma} can now be expressed as:  
\begin{align}\label{Cdef2}
	C_2(\mathbf{A}, \mathcal{H}) := \frac{ (r+1) \max_{j \in [m]} \|\mathbf{A}_j\|_2}{ 1- \max_{\mathcal{G} \in {\mathcal{H} \choose r+1}} \xi( \bm{\mathcal{A}}_\mathcal{G} ) }.
\end{align}

\section{Proving the general case $k<m$}

We may now define the constant $C_1 > 0$ of Thm.~\ref{DeterministicUniquenessTheorem} in terms of $C_2$. Given vectors $\mathbf{x}_1, \ldots, \mathbf{x}_N \in \mathbb{R}^m$, let $\mathbf{X}$ denote the $m \times N$ matrix with columns $\mathbf{x}_i$ and let $I(S)$ denote the set of indices $i$ for which $\mathbf{x}_i$ is supported in $S$. Define:
\begin{align}\label{Cdef1}
C_1(\mathbf{A}, \mathcal{H}, \{\mathbf{x}_i\}_{i=1}^N) := \frac{ C_2(\mathbf{A}, \mathcal{H}) } { \min_{S \in \mathcal{H}} L_k(\mathbf{AX}_{I(S)}) }.
\end{align}
Given the assumptions of Thm.~\ref{DeterministicUniquenessTheorem} on $\mathbf{A}$ and the $\mathbf{x}_i$, this expression for $C_1$ is well-defined\footnote{\label{note1}To see this, fix $S \in \mathcal{H}$ and $k$-sparse $\mathbf{c}$. Using the definitions, we have $\|\mathbf{AX}_{I(S)}\mathbf{c}\|_2 \geq \sqrt{k} L_\mathcal{H}(\mathbf{A})\|\mathbf{X}_{I(S)}\mathbf{c}\|_2 \geq k L_\mathcal{H}(\mathbf{A}) L_k(\mathbf{X}_{I(S)})\|\mathbf{c}\|_2$. Thus, $L_k(\mathbf{AX}_{I(S)}) \geq \sqrt{k} L_\mathcal{H}(\mathbf{A}) L_k(\mathbf{X}_{I(S)}) > 0$, since $L_{\mathcal{H}}(\mathbf{A}) \geq L_{2\mathcal{H}}(\mathbf{A})> 0$ and $L_k(\mathbf{X}_{I(S)}) > 0$ by general linear position of the $\mathbf{x}_i$.} and yields an upper bound on $\varepsilon$ consistent with that proven sufficient in the case $k=1$ considered at the beginning of this chapter.\footnote{When $\mathbf{x}_i = c_i\mathbf{e}_i$, we have $C_2 \geq 2\|\mathbf{A}_i\|_2$ and the denominator in \eqref{Cdef1} becomes $\min_{i \in [m]} |c_i| \|\mathbf{A}_i\|_2$; hence, $C_1 \geq 2 / \min_{i \in [m]} |c_i|$.}
% C_2(\mathbf{A}, \mathcal{H}) := \frac{ (r+1) \max_{j \in [m]} \|\mathbf{A}_j\|_2}{ \min_{\mathcal{G} \in {\mathcal{H} \choose r} \cup {\mathcal{H} \choose r+1}} \xi( \{ \bm{\mathcal{A}}_S \}

The practically-minded reader should note that the explicit constants $C_1$ and $C_2$ are effectively computable: the denominator of $C_1$ involves a quantity $L_k$ that may be calculated as the smallest singular value of a certain matrix, while computing the quantity $\xi$ in the denominator of $C_2$ involves computing ``canonical angles'' between subspaces, which reduces again to an efficient singular value decomposition.
There is no known fast computation of $L_k$ in general, however, since even $L_{k} > 0$ is NP-hard \cite{tillmann2014computational}, although 
efficiently computable bounds have been proposed (e.g., via the ``mutual coherence" of a matrix \cite{donoho2003optimally}); alternatively, 
fixing $k$ yields polynomial complexity. Moreover, calculating $C_2$ requires an exponential number of queries to $\xi$ unless $r$ is held fixed, too (e.g., the ``cyclic order'' hypergraphs described above have $r=k$).  Thus, as presented, $C_1$ and $C_2$ are not efficiently computable in general. % , but efficient subcases do exist and may be of practical relevance.

\begin{proof}[Proof of Thm.~\ref{DeterministicUniquenessCorollary} for $k < m$] 
We shall find a map $\pi: \mathcal{H} \to {[m] \choose k}$ for which the distance $d(\bm{\mathcal{A}}_S, \bm{\mathcal{B}}_{\pi(S)})$ is controlled by $\varepsilon$ for all $S \in \mathcal{H}$. Applying Lem.~\ref{MainLemma} then completes the proof.
%We shall show that for every $S \in \mathcal{H}$ there is some $\overline S \in {[\overline m] \choose k}$ for which the distance $d(\bm{\mathcal{A}}_S, \bm{\mathcal{B}}_{\overline S})$ is controlled by $\varepsilon$. Applying Lem.~\ref{MainLemma} with the map $\pi$ defined by $S \mapsto \overline S$ then completes the proof.

Fix $S \in \mathcal{H}$. Since there are more than $(k-1){\overline m \choose k}$ vectors $\mathbf{x}_i$ supported in $S$, by the pigeonhole principle there must be some $\overline S \in {[\overline m] \choose k}$ and a set of $k$ indices $K \subseteq I(S)$ for which all $\mathbf{\overline x}_i$ with $i \in K$ are supported in $\overline S$.
It also follows\footnote{See footnote \ref{note1}.} from $L_{2\mathcal{H}}(\mathbf{A}) > 0$ and the general linear position of the $\mathbf{x}_i$ that $L_k(\mathbf{AX}_{K}) > 0$; that is, the columns of the $n \times k$ matrix $\mathbf{AX}_K$ form a basis for $\bm{\mathcal{A}}_S$. 

Fixing $\mathbf{y} \in \bm{\mathcal{A}}_S \setminus \{\mathbf{0}\}$, there then exists $\mathbf{c} = (c_1, \ldots, c_k) \in \mathbb{R}^k \setminus \{\mathbf{0}\}$ such that $\mathbf{y} = \mathbf{AX}_K\mathbf{c}$. Setting \mbox{$\mathbf{\overline{y}} = \mathbf{B\overline{X}}_K\mathbf{c}$, which is in $\bm{\mathcal{B}}_{\overline S}$}, we have by triangle inequality:
\begin{align*}
\|\mathbf{y} - \mathbf{\overline{y}}\|_2 
%&= \|\sum_{i=1}^k c_i(\mathbf{AX}_K - \mathbf{B\overline{X}}_K)_i\|_2
&= \|(\mathbf{AX}_K - \mathbf{B\overline{X}}_K)\mathbf{c}\|_2
%\leq \varepsilon \sum_{i=1}^k |c_i| \\
\leq \varepsilon \|\mathbf{c}\|_1
\leq \varepsilon \sqrt{k}  \|\mathbf{c}\|_2  \\
&\leq \frac{\varepsilon}{L_k(\mathbf{AX}_K)} \|\mathbf{y}\|_2,
\end{align*}
where the last inequality uses \eqref{Ldef}. From Def.~\ref{dDef}:
\begin{align}\label{rhs222}
d(\bm{\mathcal{A}}_S, \bm{\mathcal{B}}_{\overline S}) 
\leq \frac{\varepsilon}{  L_k(\mathbf{AX}_{K}) } 
\leq \frac{\varepsilon}{  L_k(\mathbf{AX}_{I(S)}) } 
\leq \varepsilon \frac{C_1}{C_2}.
\end{align}
%
Finally, apply Lem.~\ref{MainLemma} with $\varepsilon < L_2(\mathbf{A})/C_1$ and $\pi(S) := \overline S$. % L_2(A) > 0 by def of L, since L_2H(A) > 0 and H covers [m]
\end{proof}

Before moving on to the proof of Thm.~\ref{SLCopt}, let us briefly revisit the discussion on sample complexity from the end of the previous chapter. While an exponential number of samples may very well prove to be necessary in the deterministic or almost-certain case, our proof of Thm.~\ref{DeterministicUniquenessTheorem} can be extended to hold with some probability for \emph{any} number of samples by alternative appeal to a probabilistic pigeonholing at the point early in the proof where the (deterministic) pigeonhole principle is applied to show that for every $S \in \mathcal{H}$, there exist $k$ vectors $\mathbf{x}_i$ supported on $S$ whose corresponding $\mathbf{\overline x}_i$ all share the same support.\footnote{A famous example of such an argument is the counter-intuitive ``birthday paradox", which demonstrates that the probability of two people having the same birthday in a room of twenty-three is greater than 50\%.} 
Given insufficient samples, this argument has some less-than-certain probability of being valid for each $S \in \mathcal{H}$. Nonetheless, simulations with small hypergraphs demonstrate that successful recovery is nearly certainly even when $N$ is only a fraction of the deterministic sample complexity (see Fig. \ref{probpigeon}). 

\begin{proof}[Proof of Thm.~\ref{SLCopt}]
We shall bound the number of $k$-sparse $\mathbf{\overline x}_i$ from below and then apply Thm.~\ref{DeterministicUniquenessCorollary}. 
%We now apply this fact to bound the number of $k$-sparse $\mathbf{\overline x}_i$. 
Let $n_p$ be the number of $\mathbf{\overline x}_i$ with $\|\mathbf{\overline x}_i\|_0 = p$.
Since the $\mathbf{x}_i$ are all $k$-sparse, by \eqref{minsum} we have:
\begin{align*}
%\mbox{$k \sum_{p = 0}^{\overline m} n_p \geq \sum_{i=0}^N \|\mathbf{x}_i\|_0 \geq \sum_{i=0}^N \|\mathbf{\overline x}_i\|_0 = \sum_{p=0}^{\overline m} p n_p.$}
\sum_{p=0}^{\overline m} p n_p =  \sum_{i=0}^N \|\mathbf{\overline x}_i\|_0
\leq \sum_{i=0}^N \|\mathbf{x}_i\|_0 
\leq kN
\end{align*}
Since $N = \sum_{p = 0}^{\overline m} n_p$, we then have $\sum_{p = 0}^{\overline m} (p-k) n_p \leq 0$. Splitting the sum yields:
\begin{align}\label{eqn}
\sum_{p = k+1}^{\overline m} n_p \leq \sum_{p = k+1}^{\overline m} (p-k) n_p \leq \sum_{p = 0}^k (k-p)n_p \leq k \sum_{p = 0}^{k-1} n_p,
\end{align}
%
demonstrating that the number of vectors $\mathbf{\overline x}_i$ that are \emph{not} $k$-sparse is bounded above by how many are $(k-1)$-sparse. 

Next, observe that no more than $(k-1)|\mathcal{H}|$ of the $\mathbf{\overline x}_i$ share a support $\overline S$ of size less than $k$. Otherwise, by the pigeonhole principle, there is some $S \in \mathcal{H}$ and a set of $k$ indices $K \subseteq I(S)$ for which all $\mathbf{x}_i$ with $i \in K$ are supported in $S$; as argued previously, \eqref{rhs222} follows. It is simple to show that $L_2(\mathbf{A}) \leq \max_j\|\mathbf{A}_j\|_2$, and since $0 \leq \xi < 1$, the right-hand side of \eqref{rhs222} is less than one for $\varepsilon < L_2(\mathbf{A})/C_1$. Thus, by \eqref{dimLem} we would have the contradiction $k = \dim(\bm{\mathcal{A}}_S) \leq \dim(\bm{\mathcal{B}}_{\overline S}) \leq |\overline S| < k.$ 

The total number of $(k-1)$-sparse vectors $\mathbf{\overline x}_i$ thus cannot exceed $|\mathcal{H}|(k-1){ \overline m \choose k-1}$. By \eqref{eqn}, no more than $|\mathcal{H}|k(k-1){ \overline m \choose k-1}$ vectors $\mathbf{\overline x}_i$ are not $k$-sparse. Since for every $S \in \mathcal{H}$ there are over $(k-1)\left[ {\overline m \choose k} + |\mathcal{H}|k{ \overline m \choose k-1} \right]$ vectors $\mathbf{x}_i$ supported there, it must be that more than $(k-1){\overline m \choose k}$ of them have corresponding $\mathbf{\overline x}_i$ that are $k$-sparse. The result now follows from Thm.~\ref{DeterministicUniquenessCorollary}, noting by the triangle inequality that $\|\mathbf{A}\mathbf{x}_i - \mathbf{B}\mathbf{\overline x}_i\| \leq 2\eta$ for $i = 1, \ldots, N$.
\end{proof}

%\begin{proof}[Proof (sketch) of Thm.~\ref{robustPolythm}]
%Let $\textbf{M}$ be the matrix with columns $\mathbf{A}\mathbf{x}_i$, $i \in [N]$.  Consider the polynomial $\prod_{S \in {[N] \choose 2k}} \sum_{S' \in {[n] \choose 2k}} (\det \textbf{M}_{S',S})^2$ in the indeterminate entries of $\mathbf{A}$ and $\mathbf{x}_i$, with notation as in Sec.~\ref{Results}.  
%It can be checked that when this polynomial is nonzero for a substitution of real numbers for the indeterminates, all of the genericity requirements on $\mathbf{A}$ and $\mathbf{x}_i$ in our proofs of stability in Thm.~\ref{DeterministicUniquenessTheorem} are satisfied (in particular, the spark condition \eqref{SparkCondition} on $\mathbf{A}$).
%\end{proof}

%\begin{proof}[Proof (sketch) of Cor.~\ref{ProbabilisticCor}]
%First, note that if a set of measure spaces $\{(X_{\ell}, \Sigma_{\ell}, \nu_{\ell})\}_{\ell=1}^p$ has that $\nu_{\ell}$ is absolutely continuous with respect to $\mu$ for all $\ell \in [p]$, where $\mu$ is the standard Borel measure on $\mathbb{R}$, then the product measure $\prod_{\ell=1}^p \nu_{\ell}$ is absolutely continuous with respect to the standard Borel product measure on $\mathbb{R}^p$. % for space:  (e.g.,  \cite{folland2013real}). 
%By Thm.~\ref{robustPolythm}, there is a polynomial that is nonzero when $Y$ has a stable $k$-sparse representation in $\mathbb{R}^m$; in particular, stability holds almost surely.
%\end{proof}

\section{Discussion}

The absence of any assumptions at all about dictionaries that solve Prob.~\ref{InverseProblem} was a major technical hurdle in proving Thm.~\ref{DeterministicUniquenessTheorem}. 
This very general guarantee was sought because of the practical difficulty of ensuring that an algorithm maintain a dictionary satisfying the spark condition \eqref{SparkCondition} at each iteration, which (to my knowledge) has been an explicit or implicit assumption of all previous works except \cite{Hillar15}; indeed, even certifying a dictionary has this property is NP-hard \cite{tillmann2014computational}.

Several results in the literature had to be combined to extend the guarantees derived in \cite{Hillar15} into the noisy regime. The main challenge was to generalize Lem.~\ref{MainLemma} to the case where the $k$-dimensional subspaces spanned by corresponding submatrices $\mathbf{A}_S$ and $\mathbf{B}_{\pi(S)}$ are assumed to be``close" but not identical. Unlike in \cite{Hillar15} where an inductive argument could be applied to the noiseless case, here it had to be explicitly demonstrated that this proximity relation is propagated through iterated intersections right down to the spans of the dictionary elements themselves. Lem.~\ref{DistanceToIntersectionLemma} was designed to encapsulate this fact, proven by appeal to a convergence guarantee for an alternating projections algorithm first proposed by von Neumann. This result, combined with a little known fact \eqref{eqdim} about the distance metric between subspaces, make up the more obscure components of the deduction.

The proof of Lem.~\ref{MainLemma} diverges most significantly from the approach taken in \cite{Hillar15} by way of Lem.~\ref{NonEmptyLemma}, which utilizes a combinatorial design for support sets (the ``singleton intersection property") to reduce the deterministic sample complexity by an exponential factor. This constitutes a significant advance toward legitimizing dictionary learning in practice, since data must otherwise exhibit the exponentially many possible $k$-wise combinations of dictionary elements required by (to my knowledge) all previously published results; although an exponential number of samples per support is still required (unless $k$ is held fixed). The issue is that the map $\pi:\mathcal{H} \to {[m] \choose k}$ is surjective only when $\mathcal{H}$ is taken to be ${[m] \choose k}$, in which case one may proceed by induction as in \cite{Hillar15}, freely choosing supports in the codomain of $\pi$ to intersect over $(k-1)$ indices to then map back to some corresponding set of $(k-1)$ indices at the intersection of supports in the domain. Here, for $\mathcal{H} \subset {[m] \choose k}$, a bijection between indices had to instead be established by pigeonholing the image of $\pi$ under constraints imposed by the SIP, which was formulated specifically for this purpose. It just so happened that this more general argument for a non-surjective $\pi$ constrained by the SIP applied just as well to the situation where the number of dictionary elements $m$ is over-estimated (i.e. $\overline m > m$), in which case a one-to-one correspondence can be guaranteed between a subset of columns of $\mathbf{A}$ and $\mathbf{B}$ of a size simply expressed in terms of the width of each matrix and the regularity of $\mathcal{H}$. 

One of the mathematically significant achievements in \cite{Hillar15} was to break free from the constraint that the recovery matrix $\mathbf{B}$ satisfy the spark condition in addition to the generating dictionary $\mathbf{A}$. %, in contrast to all previously known uniqueness results which (to my knowledge) have all either explicitly or implicitly made this assumption. 
Here, it has been demonstrated that, in fact, even $\mathbf{A}$ need not satisfy the spark condition! Rather, $\mathbf{A}$ need only be injective on the union of subspaces with supports forming a regular $k$-uniform hypergraph satisfying the SIP (a distinguishing example is given in Sec.~\ref{Results}). This relaxation of constraints inspired the definition of the restricted matrix lower bound $L_{\mathcal{H}}$ in \eqref{Ldef}, which generalizes the well-known (see footnote \ref{ripfootnote}) restricted matrix lower bound $L_k$ to be in terms of a hypergraph $\mathcal{H}$, and an interesting object for further study in its own right. %(For instance, how hard is it to check that $L_\mathcal{H}(\mathbf{A}) > 0$?) 
%The absence of any restrictions on dictionaries solving Prob.~\ref{InverseProblem} is a major technical obstruction, but highly desirable because of the practical difficulty in ensuring that an algorithm maintain a dictionary satisfying the spark condition \eqref{SparkCondition} at each iteration; indeed, even certifying a dictionary has this property is NP-hard \cite{tillmann2014computational}. 

To reiterate, the methods applied here to prove Thm.~\ref{DeterministicUniquenessTheorem} yield the following results beyond a straightforward extension of those in \cite{Hillar15} to the noisy case:

\begin{enumerate}
\item A reduction in deterministic sample complexity: To identify the $n \times m$ generating dictionary $\mathbf{A}$, it is required in \cite{Hillar15} that $k{m \choose k}$ data points be sampled from each of the ${m \choose k}$ subspaces spanned by subsets of $k$ columns of $\mathbf{A}$. It is shown here that in fact it suffices to sample from at most $m$ such subspaces (see Cor.~\ref{DeterministicUniquenessCorollary}). %; thus, for fixed $k$, the deterministic sample complexity is polynomial in $m$.  %HS15 was already poly for fixed k
\item An extension of guarantees to the case where the number of dictionary elements is unknown: The results of \cite{Hillar15} only apply to the case where the matrix $\mathbf{B}$ has the same number of columns as $\mathbf{A}$. It is shown here that if $\mathbf{B}$ has at least as many columns as $\mathbf{A}$ then it contains (up to noise) a subset of the columns of $\mathbf{A}$.
\item Relaxed requirements (no spark condition) on the generating matrix $\mathbf{A}$: Rather, $\mathbf{A}$ need only be injective on the union of subspaces with supports that form a regular uniform hypergraph satisfying the SIP. 
\end{enumerate}

As it seemed to benefit a reviewer of this work, a brief discussion on how the deterministic sample complexity $N = |\mathcal{H}\left[|(k-1){m \choose k} + 1\right]$ derived here compares to those listed in the the top two rows listed in Table I of \cite{Hillar15} may be in order. To be clear, the theory developed here is strictly more general, since $\mathcal{H}$ can always be taken to be ${[m] \choose k}$. The point of difference in this comparison is the assumed set of supports for sparse codes, which is always ${[m] \choose k}$ in \cite{Hillar15}, whereas here it can be assumed to be any regular $k$-uniform hypergraph that satisfies the SIP. By row, %, assuming for our results that sufficient data have been sampled from all supports in a hypergraph H satisfying the SIP:
\begin{enumerate}
\item[I.] The result here improves upon the listed $k{m \choose k}^2$ by an exponential factor, since for every $k < m$ there exists a regular $k$-uniform hypergraph $\mathcal{H}$ with $|\mathcal{H}| = m$ satisfying the SIP. 
\item[II.] The authors in \cite{Hillar15} have applied measure-theoretic arguments to achieve $(k+1){m \choose k}$ with almost-certainty (i.e. with probability one), a factor of $m$ reduction over that for which certainty can alternatively be guaranteed here.    
%\item[III.] A direct comparison cannot be made because it is an open question as to the probability that a random set of supports satisfies the SIP. This would be an interesting problem for the community to solve; regardless, the result of \cite{Hillar15} is still implied by the more general theory given here.
%\item[IV.] The result here is $|\mathcal{H}\left[|(k-1){m \choose k} + 1\right]$ with probability one compared to $(k+1){m \choose k}$ with probability one. In this case the sample complexity stated in \cite{Hillar15} is marginally better, but it relies on a noise-free technicality unique to their assumption $\mathcal{H} = {m \choose k}$. %differs in flavor: their $\mathbf{x_i}$ are assumed to be distributed as $(k+1)$ samples per support for every support of size $k \in [m]$, whereas here it is assumed that $(k-1){m \choose k}$ codes $\mathbf{x_i}$ are distributed over each support in some hypergraph $\mathcal{H}$ satisfying the SIP. 
%\item[V.] Again, a direct comparison cannot be made for the reason stated for row III.
\end{enumerate}
