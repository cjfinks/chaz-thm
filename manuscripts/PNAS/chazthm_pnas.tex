\documentclass[9pt,twocolumn]{pnas-new}
% Use the lineno option to display guide line numbers if required.
% Note that the use of elements such as single-column equations
% may affect the guide line number alignment. 

%%% CHAZ TODO %%%

%%% CHRIS TODO %%
% steamroll

% IMPORTED PACKAGES
\usepackage{amsmath, amssymb, amsthm} 
\newtheorem{theorem}{Theorem}
\newtheorem{lemma}{Lemma}
\newtheorem{conjecture}{Conjecture}
\newtheorem{problem}{Problem}
\newtheorem{question}{Question}
\newtheorem{proposition}{Proposition}
\newtheorem{definition}{Definition}
\newtheorem{corollary}{Corollary}
\newtheorem{remark}{Remark}
\newtheorem{example}{Example}

%\usepackage[pdftex]{graphicx}

% *** ALIGNMENT PACKAGES ***
\usepackage{array}
%\usepackage{cite}

\templatetype{pnasresearcharticle} % Choose template 
% {pnasresearcharticle} = Template for a two-column research article
% {pnasmathematics} = Template for a one-column mathematics article
% {pnasinvited} = Template for a PNAS invited submission

\title{On the uniqueness and stability of dictionaries for sparse representation of noisy signals}

% Use letters for affiliations, numbers to show equal authorship (if applicable) and to indicate the corresponding author
\author[a,1]{Charles J. Garfinkle}
\author[a,1]{Christopher J. Hillar} 

\affil[a]{Redwood Center for Theoretical Neuroscience, Berkeley, CA, USA}

% Please give the surname of the lead author for the running footer
\leadauthor{Garfinkle} 

% Please add here a significance statement to explain the relevance of your work
%\significancestatement{Authors must submit a 120-word maximum statement about the significance of their research paper written at a level understandable to an undergraduate educated scientist outside their field of speciality. The primary goal of the Significance Statement is to explain the relevance of the work in broad context to a broad readership. The Significance Statement appears in the paper itself and is required for all research papers.}

\significancestatement{Many naturally occurring signals (visual scenery, speech, electrocardiograms, etc.) can be characterized as parsimonious combinations of elementary waveforms drawn from a large dictionary. Absent analytic descriptions of these signal classes, it falls on algorithms from machine learning to extract such structures given many examples from a class. We give conditions guaranteeing when a dictionary of a given size is uniquely and stably determined by data, regardless of the algorithm used to compute it. These results provide theoretical grounding for  the application of dictionary learning to recovering sparse signals from unknown noisy compressive measurements.}

% Please include corresponding author, author contribution and author declaration information
\authordeclaration{The authors declare no conflict of interest.}
%\equalauthors{\textsuperscript{1}C.G. proved Theorem 1 and C.H. proved Theorem 2.}
\correspondingauthor{\textsuperscript{1}To whom correspondence should be addressed. E-mail: cjg@berkeley.edu, chillar@msri.org}

% Keywords are not mandatory, but authors are strongly encouraged to provide them. If provided, please include two to five keywords, separated by the pipe symbol, e.g:
\keywords{Sparse linear coding, dictionary learning, uniqueness, compressed sensing} 

\begin{abstract}
Dictionary learning for sparse linear coding has exposed underlying structure in many natural signals.
However, there are few universal theorems guaranteeing uniqueness of model estimation independent of implementation.
Here, we prove that for all diverse enough datasets generated from the sparse coding model, latent dictionaries and codes are uniquely determined up to measurement error.  Applications are given to data analysis, neuroscience, and engineering. 
\end{abstract}

% \dates{This manuscript was compiled on \today}
\doi{\url{www.pnas.org/cgi/doi/10.1073/pnas.XXXXXXXXXX}}

\begin{document}

% Optional adjustment to line up main text (after abstract) of first page with line numbers, when using both lineno and twocolumn options.
% You should only change this length when you've finalised the article contents.
\verticaladjustment{-2pt}

\maketitle
\thispagestyle{firststyle}
\ifthenelse{\boolean{shortarticle}}{\ifthenelse{\boolean{singlecolumn}}{\abscontentformatted}{\abscontent}}{}

% If your first paragraph (i.e. with the \dropcap) contains a list environment (quote, quotation, theorem, definition, enumerate, itemize...), the line after the list may have some extra indentation. If this is the case, add \parshape=0 to the end of the list environment.
\dropcap{B}lind source identification is a classical problem in signal processing \cite{sato1975method}.  
One popular modern formulation of the idea is to find a dictionary of $m$ elementary waveforms, at most $k$ of which need be linearly combined to represent each $n$-dimensional signal in a dataset of size $N$, typically where $k < m \ll N$.  Approximating solutions to this problem have
provided insight into the structure of many signals lacking explicit analytic structure (see \cite{Zhang15} for a comprehensive review). 
In particular, it has been shown that response properties of simple-cell neurons in mammalian visual cortex emerge from learning a dictionary for sparse linear coding of natural images \cite{Olshausen96, hurri1996image, bell1997independent, van1998independent}, a major advance in computational neuroscience. A fundamental aspect of this finding is that the latent waveforms (``Gabors'') estimated from data appear to be canonical; i.e., they are found in learned dictionaries independent of algorithmic implementation or natural image training set.

% (e.g., Fourier bases, wavelets, etc.)

Motivated by these discoveries and earlier work in the theory of neural communication \cite{Coulter10, Isely10}, we address when dictionaries and sparse representations are uniquely determined by data.  Answers to this question also have other real-world implications.  For example, a sparse coding analysis of local painting style can be used for forgery detection \cite{hughes2010, Olshausen10}, but only if sparse components of the artwork are independent of implementation idiosyncrasies. Fortunately, algorithms with proven recovery of generating dictionaries under certain 
conditions have recently been proposed (see \cite[Sec.~I-E]{Sun16} for a summary of the state-of-the-art).  However, there are few universal theorems explaining the general phenomenon of uniqueness.  

Here, we prove very generally that uniqueness in sparse linear coding is an expected property of the model.  In other words, whenever enough sparsely represented data generated from a dictionary (satisfying a spark condition) are observed, the original codes and dictionary are uniquely determined up to an error that is commensurate with the measurement noise.  Applications of this result include:  1) a sufficient diversity of sparse codes sent through a random linear compressive channel are unique, 2) the sparse linear coding problem is well-posed in the sense of Hadamard \cite{Hadamard1902}, and 3) an explanation for universal representation of natural signals in neuroscience. 

% It has remained unknown, however, when the dictionary learning problem is well-posed (as per Hadamard \cite{Hadamard1902}) in general -- that is, independent of all methodological idiosyncracies.
%[**TODO** make sure the claim here is right...or would it be punchier to talk about the fact that everyone always gets Gabors on natural images?]
%- algos with proven recovery of generating dictionary
%- analyses of when target dictionary is local optimizer of natural recovery criteria
%- efficient algos that globally solve DR; detailed models differ but all assume each col has bounded sparsity and nonzero coefs have sub-Gaussian magnitudes



More formally, let $\mathbf{A} \in \mathbb R^{n \times m}$ be a matrix with columns $\mathbf{A}_j$ ($j = 1,\ldots,m$) and let dataset $Z$ consist of measurements:
\begin{align}\label{LinearModel}
\mathbf{z}_i = \mathbf{A}\mathbf{x}_i + \mathbf{n}_i,\ \ \  \text{$i=1,\ldots,N$},
\end{align}
for $k$-\emph{sparse} $\mathbf{x}_i \in \mathbb{R}^m$ having at most $k$ nonzero entries and \emph{noise} $\mathbf{n}_i \in \mathbb{R}^n$, with bounded norm $\| \mathbf{n}_i \|_2 \leq  \eta$ representing our combined worst-case uncertainty in  measuring $\mathbf{A}\mathbf{x}_i$.
The precise mathematical problem addressed here is the following.
% We address the uniqueness properties for the following linear coding question in signal processing.

\begin{problem}[Sparse linear coding]\label{InverseProblem}
Find a matrix $\mathbf{B}$ and $k$-sparse $\mathbf{\hat x}_1, \ldots, \mathbf{\hat x}_N \in \mathbb{R}^m$ such that $\|\mathbf{z}_i - \mathbf{B}\mathbf{\hat x}_i\|_2 \leq \eta$ for all $i$.
\end{problem}

Note that any particular solution $(\mathbf{B}, \mathbf{\hat x}_1 \ldots, \mathbf{\hat x}_N)$ to this problem gives rise to an orbit of other solutions $(\mathbf{BPD}, \mathbf{D}^{-1}\mathbf{P}^{\top}\mathbf{\hat x}_1, \ldots, \mathbf{D}^{-1}\mathbf{P}^{\top}\mathbf{\hat x}_N)$, where $\mathbf{P}$ is any permutation matrix and $\mathbf{D}$ any invertible diagonal.  
We therefore consider whether solutions to Prob.~\ref{InverseProblem} are unique only up to this ambiguity.  More specifically, 
we seek conditions for when sparsely coded datasets have a stable unique representation.


% More specifically,  we seek sufficient conditions on when sparsely coded datasets have a stable unique representation.


% We therefore consider only whether solutions to Problem~\ref{InverseProblem} are unique up to this equivalence \cite{Li15}


% Note that by formulating the problem in this way we have introduced the following ambiguity. Any particular solution represents . Alternatively, one may seek a unique solution (in the usual sense) in the quotient space induced by the action of the transformation group of matrices $PD$.


%  noiseless case $\eta = 0$ 
 
 
%  We introduce the following terminology to handle $\eta > 0$. 
\begin{definition}\label{maindef}
Fix $Y = \{ \mathbf{y}_1, \ldots, \mathbf{y}_N\} \subset \mathbb{R}^n$. We say $Y$ has a \textbf{$k$-sparse linear representation in $m$ dimensions} if there exists a matrix $\mathbf{A}$ and $k$-sparse $\mathbf{x}_1, \ldots, \mathbf{x}_N \in \mathbb{R}^m$ such that $\mathbf{y}_i = \mathbf{A}\mathbf{x}_i$ for all $i$. %Equivalently, $\mathbf{Y} = \mathbf{A}\mathbf{X}$ for $\mathbf{Y}$ and $\mathbf{X}$ having the $\mathbf{y}_i$ and $\mathbf{x}_i$ as their columns, respectively. 
This representation is \textbf{stable} if for every $\delta_1, \delta_2 \geq 0$, there exists $\varepsilon(\delta_1, \delta_2) \geq 0$ (with $\varepsilon > 0$ when  $\delta_1, \delta_2 > 0$) such that if $\mathbf{B}$ and $k$-sparse $\mathbf{\hat x}_1, \ldots, \mathbf{\hat x}_N \in \mathbb{R}^m$ satisfy:
\begin{align*}
\|\mathbf{A}\mathbf{x}_i - \mathbf{B}\mathbf{\hat x}_i\|_2 \leq \varepsilon,\ \ \   \text{for all $i$},
\end{align*}
%
then there is some permutation matrix $\mathbf{P}$ and invertible diagonal matrix $\mathbf{D}$ such that for all $i, j$:
\begin{align}\label{def1}
\|\mathbf{A}_j - (\mathbf{BPD})_j\|_2 \leq \delta_1 \ \ \text{and} \ \ \|\mathbf{x}_i - \mathbf{D}^{-1}\mathbf{P}^{\top}\mathbf{\hat x}_i\|_1 \leq \delta_2.
\end{align}
\end{definition}
% Kato mentions a distance between n-tuples that minimizes the maximum difference over all possible orderings of one of the n-tuples.
% Make just 'sparse representation in m-dimensions? (i.e. with k unspecified?)

To see how Def. \ref{maindef} directly relates to Prob. \ref{InverseProblem}, suppose that $Y$
% \mbox{$Y = \{\mathbf{A} \mathbf{x}_1, \ldots, \mathbf{A}\mathbf{x}_N\}$} 
has a stable $k$-sparse linear representation and fix $\delta_1, \delta_2$ to be the desired recovery accuracy in \eqref{def1}. Consider now any dataset $Z$ generated as in \eqref{LinearModel} with $\eta \leq \frac{1}{2} \varepsilon(\delta_1, \delta_2)$. Then from the triangle inequality, it follows that any matrix $\mathbf{B}$ and $k$-sparse $\mathbf{\hat x}_1, \ldots, \mathbf{\hat x}_N$ solving Prob.~\ref{InverseProblem} are necessarily close to the original dictionary $\mathbf{A}$ and codes $\mathbf{x}_i$. 

%Our results, outlined in Sec.~\ref{Results}, include an explicit form for $\varepsilon(\delta_1, \delta_2)$ in Thm.~\ref{DeterministicUniquenessTheorem}. 


% We ask here: \emph{When does $Y$ have a stable $k$-sparse representation?} 

Previous theoretical work on Prob.~\ref{InverseProblem} for the noiseless case $\eta =0$ (e.g., \cite{li2004analysis, Georgiev05, Aharon06, Hillar15}) has shown that a solution (when it exists) is indeed unique provided the $\mathbf{x}_i$ are sufficiently diverse and the generating matrix $\mathbf{A}$ satisfies the \textit{spark condition} from compressed sensing (CS):
\begin{align}\label{SparkCondition}
\mathbf{A}\mathbf{x}_1 = \mathbf{A}\mathbf{x}_2 \implies \mathbf{x}_1 = \mathbf{x}_2, \ \ \ \text{for all $k$-sparse } \mathbf{x}_1, \mathbf{x}_2,
\end{align}
%
which would be, in any case, necessary for uniqueness.

Our main technical finding is that a sufficient diversity of measurements generated from any dictionary satisfying the spark condition 
has a stable sparse representation.  More specifically, such dictionaries are uniquely identifiable from as few as \mbox{$N = m(k-1){m \choose k} + m$} sparse linear combinations of their columns up to an error that is linear in the measurement noise (Thm.~\ref{DeterministicUniquenessTheorem}). In fact, provided $(n, m, k)$ satisfy CS inequality \eqref{CScondition}, in almost all cases the dictionary learning problem is well-posed given enough data (Cor.~\ref{ProbabilisticCor}). Importantly, these guarantees hold without assuming the recovered matrix satisfies the spark condition. Our results  also apply when only an upper bound on the number of dictionary elements is known. The explicit, algorithm-independent criteria we provide should be a useful tool in the theory of sparse dictionary learning.  % (Thm.~\ref{DeterministicUniquenessTheorem2})

In the next section, we give precise statements of our main results, which include an explicit form for $\varepsilon(\delta_1, \delta_2)$. We then prove our main theorem (Thm.~\ref{DeterministicUniquenessTheorem}) in Sec.~\ref{DUT} after stating some additional definitions and lemmas required for the proof, including our main tool from combinatorial matrix analysis (Lem.~\ref{MainLemma}). All other proofs are relegated to the Supplemental. 
%Our approach is a refinement of the one found in \cite{Hillar15} to handle noise and to reduce the sufficient number of samples.
%for all $k < m$ from $N=k{m \choose k}^2$ to $N = m + m(k-1){m \choose k}$. 
Finally, we present several applications in Discussion Sec.~\ref{Discussion}.

\section{Results}

In preparation for stating our main results, we first identify the combinatorial criteria on support sets of generating codes allowing for stable sparse 
representations.  Let $\{1, \ldots, m\}$ be denoted $[m]$, its power set $2^{[m]}$, and ${[m] \choose k}$ the set of subsets of $[m]$ of size $k$. 
% Let $\text{\rm Span}\{\mathbf{v}_1, \ldots, \mathbf{v}_\ell\}$ be the $\mathbb{R}$-linear span of vectors $\mathbf{v}_1, \ldots, \mathbf{v}_\ell$. 
A vector $\mathbf{x}$ is said to be \emph{supported} on $S \subseteq [m]$ %, written supp$(\mathbf{x}) = S$, 
when $\mathbf{x} \in \text{\rm Span} \{\mathbf{e}_j\}_{j\in S}$, where $\mathbf{e}_j$ are the standard basis in $\mathbb R^m$.  Let $\mathbf{M}_j$ be the $j$th column of a matrix $\mathbf{M}$ and similarly let $\mathbf{M}_S$ be the submatrix formed by the columns of $\mathbf{M}$ indexed by $S$ (we set $\text{\rm Span}\{\mathbf{M}_\emptyset\} := \{\textbf{0}\}$).  

We say a hypergraph $E \subseteq 2^{[m]}$ on vertices $[m]$ is \textit{$k$-uniform} when every $S \in E$ has cardinality $k$. We also say $E$ is \emph{regular} when every element of $[m]$ is contained in exactly $\ell$ elements of $E$ for some $\ell > 0$ (for given $\ell$, we say $E$ is \textit{$\ell$-regular}).

\begin{definition}
Given $E \subseteq 2^{[m]}$, the \textbf{star} $F(i)$ at $i$ are those $S \in E$ with $i \in S$. We say $E$ has the \textbf{singleton intersection property} (\textbf{SIP}) when $\cap_{S \in F(i)} S = \{i\}$ for all $i \in [m]$. 
\end{definition}

It is easy to verify that for every $k < m$, there is a regular $k$-uniform hypergraph that satisfies the SIP; for instance, consecutive intervals of length $k$ in some cyclic order on $[m]$.
%\begin{proposition}
%For every $k < m$, there is a regular $k$-uniform hypergraph that satisfies the singleton intersection property.
%\end{proposition}

%The second ingredient necessary for the statement of Thm.~\ref{DeterministicUniquenessTheorem} 
Next, we describe a quantitative version of the spark condition.  % We first explain how the spark condition \eqref{SparkCondition} relates to t
The \emph{lower bound} \cite{Grcar10} of a matrix $\mathbf{A} \in \mathbb R^{n \times m}$ is the largest number $\alpha$ such that \mbox{$\|\mathbf{A}\mathbf{x}\|_2 \geq \alpha\|\mathbf{x}\|_2$} for all $\mathbf{x} \in \mathbb{R}^m$. By compactness of the unit sphere, every injective linear map has a nonzero lower bound; hence, if $\mathbf{A}$ satisfies \eqref{SparkCondition}, then each submatrix formed from $2k$ of its columns or less has a nonzero lower bound. We therefore define the following domain-restricted lower bound:
\begin{align*}
L_k(\mathbf{A}) := \frac{1}{\sqrt{k}}\max \{ \alpha : \|\mathbf{A}\mathbf{x}\|_2 \geq \alpha\|\mathbf{x}\|_2 \text{ for all $k$-sparse } \mathbf{x}\}.
\end{align*} 

Clearly, $\sqrt{k} L_k(\mathbf{A}) \geq \sqrt{k'}L_{k'}(\mathbf{A})$ whenever $k < k'$, and for any $\mathbf{A}$ satisfying \eqref{SparkCondition}, we have $L_{k'}(\mathbf{A}) > 0$ for  $k' \leq 2k$. The quantity $1 - \sqrt{k} L_k(\mathbf{A})$ is also known in the CS literature as the (asymmetric) lower restricted isometry constant \cite{Blanchard2011, Foucart2009}.

We also say that a set of $k$-sparse vectors is in \emph{general linear position} when any $k$ of them are linearly independent.
The following is the precise statement of our main result. 

\begin{theorem}\label{DeterministicUniquenessTheorem}
Fix integers $n, \ell$, and $k < m$. Suppose $\mathbf{A}~\in~\mathbb{R}^{n \times m}$ satisfies the spark condition \eqref{SparkCondition} and that $k$-sparse \mbox{$\mathbf{x}_1, \ldots, \mathbf{x}_N \in \mathbb{R}^m$} contain at least \mbox{$(k-1){m \choose k}+1$} vectors supported in general linear position on each set in some $\ell$-regular $E \subseteq {[m] \choose k}$ with the SIP.  There exists a constant $C_1 > 0$ for which the following holds for all $\varepsilon < L_2(\mathbf{A}) / C_1$.

Every matrix $\mathbf{B} \in \mathbb{R}^{n \times m}$ and $k$-sparse $\mathbf{\hat x}_1, \ldots, \mathbf{\hat x}_N \in \mathbb{R}^{m}$ with  \mbox{$\|\mathbf{A}\mathbf{x}_i - \mathbf{B}\mathbf{\hat x}_i\|_2 \leq \varepsilon$} for all $i$ necessarily satisfies:
\begin{align}\label{Cstable}
\|\mathbf{A}_j - (\mathbf{B}\mathbf{PD})_j\|_2 \leq C_1 \varepsilon,\ \ \   \text{for all $j \in [m]$},
\end{align}
for some permutation matrix $\mathbf{P}$ and invertible diagonal $\mathbf{D}$.

Furthermore, if $\varepsilon < L_{2k}(\mathbf{A}) / C_1$, then $\mathbf{B}$ also satisfies \eqref{SparkCondition} with $L_{2k}(\mathbf{B}\mathbf{PD}) \geq L_{2k}(\mathbf{A}) - C_1 \varepsilon$, and:
\begin{align}\label{b-PDa}
\|\mathbf{x}_i - \mathbf{D}^{-1}\mathbf{P}^{\top}\mathbf{\hat x}_i\|_1 &\leq  \left( \frac{ 1+C_1 \|\mathbf{x}_i\|_1 }{ L_{2k}(\mathbf{A}) -  C_1\varepsilon } \right) \varepsilon,\ \ \   \text{for all $i \in [N]$}.
\end{align}
\end{theorem}

%\begin{theorem}\label{DeterministicUniquenessTheorem}
%Fix integers $n, \ell$, and $k < m \leq \hat m$. Suppose $\mathbf{A}~\in~\mathbb{R}^{n \times m}$ satisfies the spark condition \eqref{SparkCondition} and that $k$-sparse \mbox{$\mathbf{x}_1, \ldots, \mathbf{x}_N \in \mathbb{R}^m$} contain at least \mbox{$(k-1){\hat m \choose k}+1$} vectors supported in general linear position on each set in some $\ell$-regular $E \subseteq {[m] \choose k}$ satisfying the SIP.  There exists a constant $C_1 > 0$ for which the following holds for all $\varepsilon < L_2(\mathbf{A}) / C_1$.
%
%Every matrix $\mathbf{B} \in \mathbb{R}^{n \times \hat m}$ and $k$-sparse $\mathbf{\hat x}_1, \ldots, \mathbf{\hat x}_N \in \mathbb{R}^{\hat m}$ with  \mbox{$\|\mathbf{A}\mathbf{x}_i - \mathbf{B}\mathbf{\hat x}_i\|_2 \leq \varepsilon$} for all $i$ necessarily satisfies:
%\begin{align}\label{Cstable}
%\|\mathbf{A}_{\phi(j)} - (\mathbf{B}_{J}\mathbf{PD})_j\|_2 \leq C_1 \varepsilon,\ \ \   \text{for $j = 1, \ldots, |J|$},
%\end{align}
%%
%for some $J \subseteq [\hat m]$ of size \mbox{$\lfloor \hat m - \ell(\hat m - m) \rfloor$}, injective map $\phi$, permutation matrix $\mathbf{P}$, and invertible diagonal matrix $\mathbf{D}$.
%%\footnote{We note that the condition $\varepsilon < L_2(\mathbf{A})$ above is necessary. When \mbox{$\mathbf{A}$ = $I$} and $\mathbf{x}_i = \mathbf{e}_i$, it turns out that $C_1 = 1$ and there is a $\mathbf{B}$ and $1$-sparse $\mathbf{\hat x}_i$ with $\|\mathbf{A}\mathbf{x}_i - \mathbf{B}\mathbf{\hat x}_i\|_2 \leq \varepsilon$ violating \eqref{Cstable}.}
%
%Furthermore, if $\varepsilon < L_{2k}(\mathbf{A}) / C_1$, then $\mathbf{B}_J$ also satisfies \eqref{SparkCondition} with $L_{2k}(\mathbf{B}_J\mathbf{PD}) \geq L_{2k}(\mathbf{A}) - C_1 \varepsilon$, and:
%\begin{align}\label{b-PDa}
%\|\mathbf{x}_i - \mathbf{D}^{-1}\mathbf{P}^{\top}\mathbf{\hat x}_i\|_1 &\leq  \left( \frac{ 1+C_1 \|\mathbf{x}_i\|_1 }{ L_{2k}(\mathbf{A}) -  C_1\varepsilon } \right) \varepsilon,\ \ \   \text{for all $i \in [N]$}.
%\end{align}
%\end{theorem}

%In other words, the smaller the difference $\hat m - m$, the more columns and coefficients of the original $n \times m$ dictionary $\mathbf{A}$ and $m$-dimensional codes $\mathbf{x}_i$ contained (up to noise) in the appropriately scaled $n \times \hat m$ dictionary $\mathbf{B}$ and $\hat m$-dimensional codes $\mathbf{\hat x}_i$. In particular, when $\hat m = m$ all columns of $\mathbf{A}$ and coefficients of each $\mathbf{x}_i$ recoverable, as per Def.~\ref{maindef}. 

For expositional clarity, we delay defining the (explicit) constant $C_1 = C_1(\mathbf{A}, \{\mathbf{x}_i\}_{i=1}^N, E)$ until Section \ref{DUT} (\eqref{Cdef1}).  % , since it requires additional definitions not critical to the statement of our results.

An important consequence of Thm.~\ref{DeterministicUniquenessTheorem} is that \eqref{def1} is guaranteed provided $\varepsilon$ does not exceed: % [*** TODO: verify this is still correct ***]
\begin{align}\label{epsdel}
\varepsilon(\delta_1, \delta_2) := \min \left\{ \frac{\delta_1}{ C_1 }, \frac{ \delta_2 (L_{2k}(\mathbf{A}) -  C_1\varepsilon)}{ 1 + C_1 \left( \max_{i \in [N]} \|\mathbf{x}_i\|_1  + \delta_2 \right) } \right\}.
\end{align}

% Note also that Thm~\ref{DeterminsiticUniquenessTheorem} gives conditions for when a single solution to Problem~\ref{InverseProblem} in fact represents \emph{all} possible solutions, related by transformations of the form $PD$.

\begin{corollary}\label{DeterministicUniquenessCorollary}
Given $n, m$, $k < m$, and a regular hypergraph $E \subseteq {[m] \choose k}$ with the SIP, there are $N =  |E| \left[ (k-1){m \choose k} + 1  \right]$ vectors \mbox{$\mathbf{x}_1, \ldots, \mathbf{x}_N \in \mathbb{R}^m$} such that every matrix $\mathbf{A} \in \mathbb{R}^{n \times m}$ satisfying \eqref{SparkCondition} generates a dataset $Y = \{\mathbf{A}\mathbf{x}_1, \ldots, \mathbf{A}\mathbf{x}_N\}$ with a stable $k$-sparse representation in $m$ dimensions.
\end{corollary}
\begin{proof}
Given Thm.\ref{DeterministicUniquenessTheorem}, one needs only to produce sparse codes in general linear position.  However, this is straightforward with a ``Vandermonde'' matrix construction (e.g., see \cite{Hillar15}).
\end{proof}

%\begin{proposition}
%If $E$ is an $\ell$-regular hypergraph of rank $k$ satisfying the singleton intersection property then $|E| \geq \ell m/k$. 
%\end{proposition}
%<<<<<<< HEAD
%As mentioned above, it is easy to construct such $E$ with $|E| = m$, implying the lower bound for sample size $N$ from the introduction.
%A straightforward calculation verifies that if $E$ is an $\ell$-regular hypergraph of rank $k$ satisfying the SIP, then $|E| \geq \ell m/k$.
%%Note that if $E$ is an $\ell$-regular hypergraph of rank $k$ satisfying the SIP, then $|E| \geq \ell m/k$: \[|E|k \geq \sum_{S \in E}|S| = \sum_{i \in [m]} \deg(i) = \ell m.\] 
%In certain cases, there are simple constructions achieving this minimum; for example, when $k = \sqrt{m}$ one can take $E$ to be the rows and columns formed by arranging $[m]$ in a square grid.  In light of Cor.~\ref{DeterministicUniquenessCorollary}, it would be interesting to determine other minimal such $E$.
%=======
%Note that every $\ell$-regular hypergraph $E \subseteq {[m] \choose k}$ satisfies: \[|E|k = \sum_{S \in E}|S| = \sum_{i \in [m]} \deg(i) = \ell m.\] 
As mentioned above, it is easy to construct a regular hypergraph $E \subseteq {[m] \choose k}$ with the SIP having size $|E| = m$, implying the lower bound for sample size $N$ from the introduction. In certain cases, the SIP can be achieved with $|E| < m$; for example, when $k = \sqrt{m}$ one can take $E$ to be the rows and columns formed by arranging $[m]$ in a square grid.  In light of Cor.~\ref{DeterministicUniquenessCorollary}, it would be interesting to determine other such $E$. %[ ** TODO ** can we show for every $k, m$ that there is an $E$ satisfying the SIP with $|E| = \ell m/k$ for every $\ell$ such that $\ell m/k$ is an integer? Then lower bound on $N$ comes from the smallest such $\ell$.]

There are other less direct consequences of Thm.~\ref{DeterministicUniquenessTheorem} to stability in sparse linear coding.  The following implication relates the result to more common optimization formulations.

\begin{corollary}
Fix $n, \ell$, and $k < m$. Suppose $\mathbf{A}~\in~\mathbb{R}^{n \times m}$ satisfies the spark condition \eqref{SparkCondition} and that $k$-sparse \mbox{$\mathbf{x}_1, \ldots, \mathbf{x}_N \in \mathbb{R}^m$} have at least \mbox{$\left[ (k-1){ m \choose k} + 1 \right] + (k-1)^2{m \choose k-1}$} vectors supported in general linear position on each set in an $\ell$-regular $E \subseteq {[m] \choose k}$ with the SIP.  Given any $\delta_1, \delta_2 \geq 0$ and $\varepsilon \leq \varepsilon(\delta_1, \delta_2)$, the following holds.  % (\eqref{epsdel}).
% $\varepsilon < L_2(\mathbf{A}) / C_1$.
%There exists a constant $C_1 > 0$ for which the following holds for all $\varepsilon < L_2(\mathbf{A}) / C_1$.
Every solution $\mathbf{B}, \mathbf{\hat x}_i, \ldots, \mathbf{\hat x}_N$ to the constrained $\ell_0$ optimization problem:
\begin{align}\label{minsum}
\arg \min \sum_{i = 1}^N \|\mathbf{\hat x}_i\|_0 \ \ \text{s.t.} \ \ \|\mathbf{A}\mathbf{x}_i - \mathbf{B}\mathbf{\hat x}_i\|_2 \leq \varepsilon, \ \ \text{$i \in [N]$},
\end{align}
necessarily satisfies the recovery inequalities in \eqref{def1}. % \eqref{Cstable} and \eqref{b-PDa}.
\end{corollary}

\begin{proof}
We derive a lower bound on the number of $k$-sparse $\mathbf{\hat x}_i$ and then apply Thm.~\ref{DeterministicUniquenessCorollary}. 
First, note that $\|\mathbf{\hat x}_i\|_0 \neq 0$ for all $i$, since otherwise we would have $\mathbf{A}\mathbf{x}_i = \mathbf{B}\mathbf{\hat x}_i = \mathbf{0} \implies \mathbf{x}_i = \mathbf{0}$ by the spark condition, contradicting the general linear position of the $\mathbf{x}_i$. Let $n_p$ be the number of $\mathbf{\hat x}_i$ with $\|\mathbf{\hat x}_i\|_0 = p$.  %, so that $\sum_{p = 1}^{m} n_p = N$. 
Since the $\mathbf{x}_i$ are all $k$-sparse, by \eqref{minsum} we have:
\begin{align}
k \sum_{p = 1}^{m} n_p = kN \geq \sum_{i=1}^N \|\mathbf{x}_i\|_0 \geq \sum_{i=1}^N \|\mathbf{\hat x}_i\|_0 = \sum_{p=1}^{m} p n_p 
\end{align}
\[\implies \sum_{p = 1}^k (k-p)n_p \geq \sum_{p = k+1}^{m} (p-k) n_p.\]
Hence,
\begin{align}
\sum_{p = k+1}^m n_p \leq \sum_{p = k+1}^m (p-k) n_p \leq \sum_{p = 1}^k (k-p)n_p \leq (k-1) \sum_{p = 1}^{k-1} n_p.
\end{align}
%\[ \implies \sum_{p = k+1}^m n_p \leq (k-1)^2 {\hat m \choose k-1} \]
By the spark condition and general linear position of the $\mathbf{x}_i$, every $k$ vectors $\mathbf{Ax}_i$ span a $k$-dimensional space; hence, at most $k-1$ vectors $\mathbf{\hat x}_i$ can share a support of size $k-1$ or less, and the number of $(k-1)$-sparse vectors $\mathbf{\hat x}_i$ is then bounded from above by $(k-1) { m \choose k-1}$. It follows that at least $N - (k-1)^2 {m \choose k-1}$ of the vectors $\mathbf{\hat  x}_i$ are $k$-sparse, and that there must then be some $N' \subseteq [N]$ for which for each $S \in E$ there are $(k-1){m \choose k}+1$ of the vectors $\mathbf{x}_i$ with $i \in [N']$ supported on $S$ and for which $\mathbf{\hat x}_i$ is $k$-sparse for all $i \in [N']$.  The result now follows directly from Thm.~\ref{DeterministicUniquenessCorollary}. % \eqref{Cstable} and \eqref{b-PDa} follow.
\end{proof}

Another straightforward application of Thm.~\ref{DeterministicUniquenessTheorem} is a probabilistic extension, which takes advantage of the following well-known application of random matrix theory to compressed sensing.  A random $n \times m$ matrix obeys \eqref{SparkCondition} with probability one (or ``high probability'' for discrete variables) 
provided:
\begin{align}\label{CScondition}
n \geq \gamma k\log\left(\frac{m}{k}\right),
\end{align}
in which $\gamma >0$ is a constant that depends on the particular distribution from which the entries of $\mathbf{A}$ are sampled i.i.d. (many ensembles suffice, e.g. see \cite[Sec.~4]{Baraniuk08}). 

In fact, 
% We motivate our next theorem by the following observation:  
the spark condition can be made explicit.  Let $\mathbf{A}$  be the $n \times m$ matrix of $nm$ indeterminates $A_{ij}$. When real numbers are substituted for $A_{ij}$, the resulting matrix satisfies \eqref{SparkCondition} if and only if the following polynomial is nonzero:
\begin{align*}
f(\mathbf{A}) := \prod_{S \in {[m] \choose k}} \sum_{S' \in {[n] \choose k}} (\det \mathbf{A}_{S',S})^2,
\end{align*}
%
where for any $S' \in {[n] \choose k}$ and $S \in {[m] \choose k}$, the symbol $\mathbf{A}_{S',S}$ denotes the submatrix of entries $A_{ij}$ with $(i,j) \in S' \times S$.   We note that the large number of terms in this product is likely necessary due to the NP-hardness of deciding whether a given matrix $A$ satisfies the spark condition \cite{tillmann2014computational}.

Since $f$ is a real analytic function, having one substitution of real numbers with $f(\mathbf{A}) \neq 0$ implies that its zeroes form a set of (Borel) measure zero. Hence, almost every $n \times m$ real matrix $\mathbf{A}$ satisfies \eqref{SparkCondition} provided \eqref{CScondition} holds for a value of $\gamma$ for some distribution. We remark that the precise relationship between $m$, $n$, and $k$ guaranteeing that $f$ is not identically zero is a challenging problem in real algebraic geometry. In any case, we set $\gamma_0$ to be the smallest known such $\gamma$.

A similar phenomenon applies to sets of vectors with a stable sparse representation. As in \cite[Sec.~IV]{Hillar15}, consider the ``symbolic'' dataset $Y = \{\mathbf{A}\mathbf{x}_1,\ldots,\mathbf{A} \mathbf{x}_N\}$ generated by indeterminate $\mathbf{A}$ and indeterminate $k$-sparse $\mathbf{x}_1, \ldots, \mathbf{x}_N$.  

\begin{theorem}\label{robustPolythm} %label is fucking up formatting..???
Fix $n, m$, and $k < m$. There is a polynomial in the entries of $\mathbf{A}$ and the $\mathbf{x}_i$ with the following property:  if the polynomial evaluates to a nonzero number and at least \mbox{$(k-1){m \choose k}+1$} of the resulting vectors $\mathbf{x}_i$ are supported on each $S \in E$ for some regular $E \subseteq {[m] \choose k}$ with the SIP, then $Y$ has a stable $k$-sparse representation in $\mathbb{R}^m$ (Def.~\ref{maindef}). In particular, either no substitutions impart to $Y$ this property or all but a Borel set of measure zero do. 
\end{theorem}

% this is cool, but let's just save it for later...i can ask around.  -cjh
%An interesting open question in real algebraic geometry is which collections of $(m,n,k)$ determine this last ``or".

\begin{corollary}\label{ProbabilisticCor}
Fix $n, m$, and $k$ satisfying \eqref{CScondition} for $\gamma = \gamma_0$, and let the entries of $\mathbf{A} \in \mathbb{R}^{n \times m}$ and $k$-sparse $\mathbf{x}_1, \ldots, \mathbf{x}_N \in \mathbb{R}^m$ be drawn independently from probability measures absolutely continuous with respect to the standard Borel measure $\mu$. If at least $(k-1){m \choose k} + 1$ of the vectors $\mathbf{x}_i$ are supported on each $S \in E$ for a regular $E \subseteq {[m] \choose k}$ with the SIP, then $Y$ has a stable $k$-sparse representation in $\mathbb{R}^m$ with probability one.
\end{corollary}

% [ ** TODO ** elaborate a little on these? ]

%Next, we address the case when only an upper bound $m'$ on the latent dimension $m$ is known (assuming that $\mathbf{B}$ satisfies \eqref{SparkCondition}).

%\begin{theorem}\label{DeterministicUniquenessTheorem2}
%Fix integers $n$ and $k < m \leq m'$ and matrices $\mathbf{A}~\in~\mathbb{R}^{n \times m}$ and $\mathbf{B} \in \mathbb{R}^{n \times m'}$ both satisfying \eqref{SparkCondition}. Suppose \mbox{$\mathbf{x}_1, \ldots, \mathbf{x}_N \in \mathbb{R}^m$} include at least \mbox{$(k-1){m' \choose k}+1$} $k$-sparse vectors in general linear position supported on each set in some $k$-uniform $E \subseteq 2^{[m]}$ satisfying the singleton intersection property. Then there exists a constant $C_3 > 0$ for which the following holds.

%If for some $k$-sparse $\mathbf{\hat x}_1, \ldots, \mathbf{\hat x}_N \in \mathbb{R}^{m'}$ and $\varepsilon < L_2(\mathbf{A}) / C_3$ we have \mbox{$\|\mathbf{A}\mathbf{x}_i - \mathbf{B}\mathbf{\hat x}_i\|_2 \leq \varepsilon$} for all $i$, then:
%\begin{align}\label{Cstablem'}
%\|\mathbf{A}_j-(\mathbf{B}_J\mathbf{PD})_j\|_2 \leq C_3\varepsilon \ \ \text{for all $j \in [m]$}
%\end{align}
%
%for some $J \in {[m'] \choose m}$, permutation matrix $\mathbf{P}$, and invertible diagonal matrix $\mathbf{D}$.
%\end{theorem}

%In other words, the columns of $B$ contain (up to noise, after appropriate scaling) the columns of the original dictionary $\mathbf{A}$. Similarly, one can show by the same arguments at the beginning of Sec.~\ref{DUT} that the $\mathbf{\hat x}_i$ contain the original codes $\mathbf{x}_i$. The constant $C_3$ here is expression (\ref{Cdef2}) from the proof of Thm.~\ref{DeterministicUniquenessTheorem2}. Note that, in contrast to Thm.~\ref{DeterministicUniquenessTheorem}, this constant is dependent on $\mathbf{B}$; hence, \eqref{Cstablem'} holds for \emph{all} matrices $\mathbf{B} \in \mathbb{R}^{n \times m'}$ satisfying the spark condition only in the case $\varepsilon = 0$. 

\section{Proof of Theorem~\ref{DeterministicUniquenessTheorem}}\label{DUT}

% ======== b - PDa =========
First note that $\|\mathbf{x}\|_1 \leq \sqrt{k} \|\mathbf{x}\|_2$ for $k$-sparse $\mathbf{x} \in \mathbb{R}^m$, which by definition of $L_k(A)$ implies the following often used inequality:
\begin{align}\label{delrho}
L_k(\mathbf{A}) \leq \frac{\|\mathbf{A}\mathbf{x}\|_2}{\sqrt{k} \|\mathbf{x}\|_2} %\leq \frac{\|\mathbf{x}\|_1}{\sqrt{k} \|\mathbf{x}\|_2} \max_{i \in [m]}\|\mathbf{A}_i\|_2 
\leq  \max_{j \in [m]}\|\mathbf{A}_j\|_2.
\end{align}

Our first step is to show how dictionary recovery \eqref{Cstable} already implies sparse code recovery \eqref{b-PDa} when $\varepsilon < L_{2k}(\mathbf{A}) / C_1$. For all $2k$-sparse $\mathbf{x} \in \mathbb{R}^m$, the triangle inequality gives \mbox{$\|(\mathbf{A}-\mathbf{BPD})\mathbf{x}\|_2  \leq C_1\varepsilon \|\mathbf{x}\|_1 \leq C_1 \varepsilon \sqrt{2k}  \|\mathbf{x}\|_2$}. Thus, 
\begin{align*}
\|\mathbf{BPD}\mathbf{x}\|_2 
&\geq | \|\mathbf{A}\mathbf{x}\|_2 - \|(\mathbf{A}-\mathbf{BPD})\mathbf{x}\|_2 | \\
&\geq \sqrt{2k} (L_{2k}(\mathbf{A}) -  C_1\varepsilon) \|\mathbf{x}\|_2,
\end{align*}
%
%where we drop the absolute value due to the upper bound on $\varepsilon$. 
Hence, $L_{2k}(\mathbf{BPD}) \geq L_{2k}(\mathbf{A}) - C_1\varepsilon  > 0$ and \eqref{b-PDa} then follows from:
\begin{align*}
\|\mathbf{x}_i - \mathbf{D}^{-1}\mathbf{P}^{\top}\mathbf{\hat x}_i \|_1
%&\leq \sqrt{2k} |\mathbf{x}_i - D^{-1}P^{\top}\mathbf{\hat x}_i\|_2 \\
&\leq \frac{\|\mathbf{BPD}(\mathbf{x}_i - \mathbf{D}^{-1}\mathbf{P}^{\top}\mathbf{\hat x}_i)\|_2}{L_{2k}(\mathbf{BPD})} \\
&\leq \frac{\|\mathbf{B}\mathbf{\hat x}_i - \mathbf{A}\mathbf{x}_i\|_2 + \|(\mathbf{A} - \mathbf{BPD})\mathbf{x}_i\|_2}{L_{2k}(\mathbf{BPD})} \\
&\leq \frac{\varepsilon (1+C_1 \|\mathbf{x}_i\|_1)}{L_{2k}(\mathbf{BPD})}.
%&\leq \frac{\varepsilon}{\sqrt{2k}} \left( \frac{ 1+C_1 \|\mathbf{x}_i\|_1 }{ L_{2k}(\mathbf{A}) -  C_1\varepsilon } \right).
%\leq \frac{\varepsilon }{\varepsilon_0 - \varepsilon} \left( C_1^{-1}+|\mathbf{x}_i\|_1 \right).
\end{align*}

The heart of the matter is therefore \eqref{Cstable}, which we now finally establish, but first in the important special case $k = 1$.

%In fact, in the case $k=1$ we can relax our assumptions even further and assume that the matrix $\mathbf{B}$ has only at least as many columns as $\mathbf{A}$; as we will see, the conclusions of Thm.~\ref{DeterministicUniquenessTheorem} then hold for some $n \times m$ submatrix of $\mathbf{B}$.  [*** ??? ***]

\begin{proof}[Proof of Thm.~\ref{DeterministicUniquenessCorollary} for $k=1$]
The only 1-uniform hypergraph with the SIP is $[m]$; hence, we have $\mathbf{x}_i = c_i \mathbf{e}_i$ for $c_i \in \mathbb{R} \setminus \{\mathbf{0}\}$, $i \in [m]$. In this case, we may take $C_1 = 1/ \min_{\ell \in [m]} |c_{\ell}|$. 

Fix $\mathbf{A} \in \mathbb{R}^{n \times m}$ satisfying \eqref{SparkCondition} and suppose that for some $\mathbf{B} \in \mathbb{R}^{n \times \hat m}$ ($\hat m \geq m$) and $1$-sparse $\mathbf{\hat x}_i \in \mathbb{R}^{\hat m}$ we have  $\|\mathbf{A}\mathbf{x}_i - \mathbf{B}\mathbf{\hat x}_i\|_2 \leq \varepsilon < L_2(\mathbf{A}) / C_1$ for all $i$. Then, there are $\hat{c}_1, \ldots, \hat{c}_m \in \mathbb{R}$ and a map $\pi: [m] \to [\hat m]$ with:
\begin{align}\label{1D}
\|c_j\mathbf{A}_j - \hat{c}_j\mathbf{B}_{\pi(j)}\|_2 \leq \varepsilon,\ \ \   \text{for all $j$}.
\end{align} 
Note that $\hat{c}_j \neq 0$, since otherwise (by definition of $L_2(\mathbf{A})$) we reach the contradiction $|c_j| < \min_{\ell \in [m]} | c_\ell |$. %by \eqref{delrho} and definition of $C_1$ we would have $\|c_j\mathbf{A}_j\|_2 < \min_{\ell \in [m]}\|c_{\ell}\mathbf{A}_{\ell}\|_2$.  [*** TODO:  This doesn't seem correct ***]
%\begin{align*}
%|c_j| \sqrt{2} L_2(\mathbf{A}) \leq \|\mathbf{A}(c_j\mathbf{e}_j)\|_2 \leq \varepsilon < L_2(\mathbf{A}) \min_{\ell \in [m]} | c_\ell |.
%\end{align*}

We  now show that $\pi$ is injective (and thus is a permutation if $\hat m = m$). Suppose that $\pi(i) = \pi(j) = \ell$ for some $i \neq j$ and $\ell$. Then, $\|c_{j}\mathbf{A}_{j} - \hat{c}_{j}\mathbf{B}_{\ell}\|_2 \leq \varepsilon$ and $\|c_{i}\mathbf{A}_{i} - \hat{c}_{i} \mathbf{B}_{\ell}\|_2  \leq \varepsilon$. Scaling and summing these inequalities by $|\hat{c}_{i}|$ and $|\hat{c}_{j}|$, respectively, and applying the triangle inequality, we obtain:
\begin{align*}%\label{contra}
(|\hat{c}_{i}| + |\hat{c}_{j}|) \varepsilon
&\geq\|\mathbf{A}(\hat{c}_{i}c_{j} \mathbf{e}_{j} - \hat{c}_{j}c_{i}\mathbf{e}_{i})\|_2 \nonumber \\ 
&\geq  \left( |\hat{c}_{i}| + |\hat{c}_{j}| \right) L_2(\mathbf{A}) \min_{\ell \in [m]} |c_\ell |,
\end{align*}
%
which contradicts the bound $\varepsilon < L_2(\mathbf{A})/C_1$. Hence, the map $\pi$ is injective. Setting $J = \pi([m])$ and letting $P = \left( \mathbf{e}_{\pi(1)} \cdots \mathbf{e}_{\pi(m)}\right)$ and $D = \text{diag}(\frac{\hat{c}_1}{c_1},\ldots,\frac{\hat{c}_m}{c_m})$, we see that \eqref{1D} becomes, for all $j \in [m]$:
\begin{align*}%\label{k=1result}
\|\mathbf{A}_j - (\mathbf{B}_J\mathbf{PD})_j\|_2 
= \|\mathbf{A}_j - \frac{\hat{c}_j}{c_j}\mathbf{B}_{\pi(j)}\|_2 
\leq \frac{\varepsilon}{|c_j|} 
\leq C_1\varepsilon.
\end{align*}
\end{proof}

%\begin{remark}
%The above arguments can be easily modified to show that the conclusions of Thm.~\ref{DeterministicUniquenessTheorem} in this case ($k=1$) hold for some $n \times m$ submatrix of $\mathbf{B}$ in the event where only an upper bound on the number of columns in $\mathbf{A}$ is known; i.e., the recovered matrix $\mathbf{B}$ is set to have as many or more columns than $\mathbf{A}$. 
%\end{remark}

%It is easy to see that when $m < m'$, the above result holds for the submatrix of $B$ composed of columns indexed by the image of $\pi$.

We require a few additional tools to extend the proof to the general case $k < m$. These include a general notion of distance (Def.~\ref{dDef}) and angle (Def.~\ref{FriedrichsDefinition}) between subspaces and a stability result in matrix analysis (Lem.~\ref{MainLemma}).

\begin{definition}\label{dDef}
For $\mathbf{u} \in \mathbb R^m$, let $\text{\rm dist}(\mathbf{u}, V) := \inf \{\| \mathbf{u}-\mathbf{v} \|_2: \mathbf{v} \in V\}$, and for subspaces $U,V \subseteq \mathbb{R}^m$, define\footnote{Although identical, note the reference \cite{Morris10} defines the supremum over the unit ball.}: %[Kato p.197]
\begin{align}\label{d}
d(U,V) := \sup_{\mathbf{u} \in U, \ \|\mathbf{u}\|_2 = 1} \text{\rm dist}(\mathbf{u},V).
\end{align}
Note that $d(U,\{\textbf{0}\}) = 1$ for $U \neq \{\mathbf{0}\}$, whereas if $U = \{\textbf{0}\}$ then $d$ has no meaning; in this case, set $d(\{\textbf{0}\},V) = 0$ for any $V$.
\end{definition}
%Note that the supremum is always achievable by a point in $U$ when $U, V \subseteq \mathbb{R}^m$.

We note the following facts about $d$. For subspaces $U,V \subseteq \mathbb{R}^m$, we have \cite[Cor.~2.6]{Kato2013}:
\begin{align}\label{dimLem}
d(U,V) < 1 \implies \dim(U) \leq \dim(V),
\end{align}
%
and \cite[Lem.~3.2]{Morris10}:
\begin{align}\label{eqdim}
\dim(U) = \dim(V) \implies d(U,V) = d(V,U).
\end{align}

%\begin{lemma}\label{dimLem} %Kato p.200, 
%\cite[Cor.~2.6]{Kato2013} For subspaces $U,V \subseteq \mathbb{R}^m$, $d(U,V) < 1$ implies $\dim(U) \leq \dim(V)$. 
%\end{lemma}

%\begin{lemma}\label{eqdim}
%\cite[Lem.~3.2]{Morris10} If $\dim(U) = \dim(V)$ then $d(U,V) = d(V,U)$. 
%\end{lemma}

Our result in combinatorial matrix theory is the following.

%, which we derive by the following arguments (see the Supplemental for the full proof). First, we show that the aforementioned proximity between $k$-dimensional subspaces implies a proximity between smaller subspaces spanned by columns of $\mathbf{A}$ indexed by the intersections of sets in $E$ and those spanned by as many or fewer columns of $\mathbf{B}$. Another pigeonholing argument here combined with our assumptions on $E$ (e.g., the singleton intersection property) reveals that, in fact, each column of $\mathbf{A}$ spans a subspace proximal to that spanned by some column of $\mathbf{B}$. \eqref{Cstable} is a simple consequence of this last fact, which actually constitutes the proof of the theorem for the case $k=1$. % We present this special case now before restating the above arguments in greater detail.

\begin{lemma}\label{MainLemma}
Fix integers $n$ and $k < m$, and suppose $\mathbf{A} \in \mathbb{R}^{n \times m}$ satisfies the spark condition \eqref{SparkCondition}. There exists a constant $C_2 > 0$ for which the following holds for all $\varepsilon < L_2(\mathbf{A}) / C_2$. If for some  $\mathbf{B} \in \mathbb{R}^{n \times m}$ and regular $E \subseteq {[m] \choose k}$ with the SIP there exists a map $\pi: E \mapsto {[m] \choose k}$ satisfying:
\begin{align}\label{GapUpperBound}
d(\text{\rm Span}\{\mathbf{A}_{S}\}, \text{\rm Span}\{\mathbf{B}_{\pi(S)}\}) \leq \varepsilon,\ \ \   \text{for all $S \in E$},
\end{align}
%
then there exists a permutation matrix $\mathbf{P}$ and invertible diagonal matrix $\mathbf{D}$ with:
\begin{align}\label{MainLemmaBPD}
\|\mathbf{A}_j - (\mathbf{BPD})_j\|_2 \leq C_2 \varepsilon, \ \ \  \text{for all } j \in [m].
\end{align}
\end{lemma}

The constant $C_1$ in Thm.~\ref{DeterministicUniquenessTheorem} is then given by\footnote{We can be sure $C_1 > 0$ since $L_k(\mathbf{A}), L_k(\mathbf{X}_{I(S)}) > 0$ by the spark condition and general linear position of the $\mathbf{x}_i$; hence $\|\mathbf{A}_j\|_2 > 0$ for all $j$ and for all $S \in E$ we have $\|\mathbf{AX}_{I(S)}\mathbf{c}\|_2 \geq L_k(\mathbf{A})\|\mathbf{X}_{I(S)}\mathbf{c}\|_2 \geq L_k(\mathbf{A}) L_k(\mathbf{X}_{I(S)})\|\mathbf{c}\|_2$ for all $k$-sparse $\mathbf{c}$, and therefore $L_k(\mathbf{AX}_{I(S)}) \geq L_k(\mathbf{A}) L_k(\mathbf{X}_{I(S)}) > 0$.}:
\begin{align}\label{Cdef1}
C_1(\mathbf{A}, \{\mathbf{x}_i\}_{i=1}^N, E) := \frac{ C_2(\mathbf{A}, E) } { \min_{S \in E} L_k(\mathbf{AX}_{I(S)}) },
\end{align}
%
where $\mathbf{X}$ is the $m \times N$ matrix with columns $\mathbf{x}_i$ and $I(S)$ is defined as those $i \in [N]$ with supports in $S$. The constant $C_2 = C_2(\mathbf{A}, E)$ is defined below in terms of one used in \cite{Deutsch12} to analyze the convergence of the alternating projections algorithm.
% for projecting a point onto the intersection of a set of subspaces. 
We use it to bound the distance between a point and the intersection of a set of subspaces given an upper bound on its distance from each subspace individually.

\begin{definition}\label{SpecialSupportSet}\label{FriedrichsDefinition}
For subspaces $V_1, \ldots, V_\ell \subseteq \mathbb{R}^m$, set $r := 1$ when $\ell = 1$ and define for $\ell \geq 2$:
\begin{align*}
r(\{V_i\}_{i=1}^\ell) := 1 - \left(1 -  \max \prod_{i=1}^{\ell-1} \sin^2  \theta \left(V_i, \cap_{j>i} V_j \right)  \right)^{1/2},
\end{align*} 
%
where the maximum is taken over all orderings\footnote{We modify the quantity in \cite{Deutsch12} in this way since the subspace ordering is irrelevant to our purpose.} of the $V_i$ and the angle $\theta \in (0,\frac{\pi}{2}]$ is defined implicitly as \cite[Def.~9.4]{Deutsch12}:
\begin{align*}
\cos{\theta(U,W)} := \max\left\{ |\langle \mathbf{u}, \mathbf{w} \rangle|: \substack{ \mathbf{u} \in U \cap (U \cap W)^\perp, \ \|\mathbf{u}\|_2 \leq 1 \\ \mathbf{w} \in W \cap (U \cap W)^\perp, \  \|\mathbf{w}\|_2 \leq 1 } \right\}.
\end{align*}
\end{definition}
Note that $\theta \in (0,\frac{\pi}{2}]$ implies $0 < r \leq 1$. (We acknowledge the somewhat counter-intuitive property that $\theta =  \pi/2$ when $U = V$ and when $U \perp V$.)  %\textbf{Note:} we can also define $\theta$ by \cite[Lem.~9.5]{Deutsch12}:
%\begin{align*}
%\cos{\theta(U,W)} := \max\left\{ |\langle \mathbf{u}, \mathbf{w} \rangle|: \mathbf{u} \in U \cap (U \cap W)^\perp, \|\mathbf{u}\|_2 \leq 1, \mathbf{w} \in W \|\mathbf{w}\|_2 \leq 1  \right\}.
%\end{align*}

The constant $C_2$ in Lem.~\ref{MainLemma} can then be expressed as:  % \footnote{Note that $C_2 > 0$ is well-defined since $r > 0$ by definition.}
\begin{align}\label{Cdef2}
C_2(\mathbf{A}, E) := \frac{ 2^{|E|} \max_{j \in [m]} \|\mathbf{A}_j\|_2}{ \min_{F \subseteq E} r( \{ \text{\rm Span}\{\mathbf{A}_{S}\} \}_{S \in F}) }.
\end{align}
%Note that when $k=1$ the constant $C_1$ is consistent with the value in our proof of the case $k=1$.

\begin{proof}[Proof of Thm.~\ref{DeterministicUniquenessCorollary} for $k < m$] 
%We begin by showing that for every $S \in E$ there is some $S' \in {[m] \choose k}$ for which the distance $d(\text{\rm Span}\{\mathbf{A}_S\}, \text{\rm Span}\{\mathbf{B}_S'\})$ is controlled by $\varepsilon$. Fix $S \in E$. Since there are $(k-1){m \choose k}+1$ vectors $\mathbf{x}_i$ supported on $S$, the pigeon-hole principle implies that there exists some $S' \in {[m] \choose k}$ and some set of $k$ indices $J(S)$ such that the supports of all $\mathbf{x}_i$ and $\mathbf{\hat x}_i$ with $i \in J(S)$ are contained in $S$ and $S'$, respectively.
%%% No |S| = k assumption %%%
We shall show that for every $S \in E$ there is some $S' \subseteq [m]$, $|S'| \leq k$, for which the distance $d(\text{\rm Span}\{\mathbf{A}_S\}, \text{\rm Span}\{\mathbf{B}_{S'}\})$ is controlled by $\varepsilon$. Since there are $(k-1){m \choose k}+1$ vectors $\mathbf{x}_i$ supported on $S$, the pigeon-hole principle implies that there exists some $S' \in {[m] \choose k}$ and some set of $k$ indices $I(S)$ such that the supports of all $\mathbf{x}_i$ and $\mathbf{\hat x}_i$ with $i \in I(S)$ are contained in $S$ and $S'$, respectively.

Let $\mathbf{X}$ and $\mathbf{\hat{X}}$ be the $m \times N$ matrices with columns $\mathbf{x}_i$ and $\mathbf{\hat x}_i$, respectively. It follows from the general linear position of the $\mathbf{x}_i$ and the linear independence of every $k$ columns of $\mathbf{A}$ that $L_k(\mathbf{AX}_{I(S)}) > 0$; that is, the columns of the $n \times k$ matrix $\mathbf{AX}_{I(S)}$ form a basis for $\text{\rm Span}\{\mathbf{A}_{S}\}$. Fixing $\mathbf{y} \in \text{\rm Span}\{\mathbf{A}_{S}\}$, there exists a unique $\mathbf{c} = (c_1, \ldots, c_k) \in \mathbb{R}^k$ such that $\mathbf{y} = \mathbf{AX}_{I(S)}\mathbf{c}$. Setting \mbox{$\mathbf{\hat{y}} = \mathbf{B\hat{X}}_{I(S)}\mathbf{c} \in \text{\rm Span}\{\mathbf{B}_{S'}\}$}, we have:
\begin{align*}
\|\mathbf{y} - \mathbf{\hat{y}}\|_2 
&= \|\sum_{i=1}^k c_i(\mathbf{AX}_{I(S)} - \mathbf{B\hat{X}}_{I(S)})_i\|_2
\leq \varepsilon \sum_{i=1}^k |c_i| \\
&\leq \varepsilon \sqrt{k}  \|\mathbf{c}\|_2 
%\leq \frac{\varepsilon \sqrt{k}}{L(\mathbf{A}X_{J(S)})} \|\mathbf{A}X_{J(S)}\mathbf{c}\|_2 
\leq \frac{\varepsilon}{L_k(\mathbf{AX}_{I(S)})} \|\mathbf{y}\|_2,
\end{align*}
where the last inequality is by definition of $L_k$. Also, from Def.~\ref{dDef}, it follows that for all $S \in E$,
\begin{align}\label{rhs222}
d(\text{\rm Span}\{\mathbf{A}_S\}, \text{\rm Span}\{\mathbf{B}_{S'}\}) 
\leq \frac{\varepsilon}{ \min_{S \in E} L_k(\mathbf{AX}_{I(S)}) }.
%\leq \frac{\varepsilon}{L(\mathbf{AX}_{J(S)})}.
%&< \left( \frac{ L_2(\mathbf{A}) }{ 2^{|E|} \max_{j \in [m]} \|\mathbf{A}_j\|_2 } \right) \frac{ \min_{S \in E} L_k(\mathbf{AX}_{I(S)}) } { L(\mathbf{AX}_{J(S)}) } \min_{\substack{F \subseteq E}} r( \{ \text{\rm Span}\{\mathbf{A}_{S}\} \}_{S \in F}) \\
\end{align}
Finally, applying Lem.~\ref{MainLemma} with the map $\pi: E \to {[m] \choose k}$ defined by $S \mapsto S'$ above completes the proof.
%\begin{align}
%\delta= \frac{\varepsilon}{ \min_{S \in E} L_k(\mathbf{AX}_{I(S)}) } = \frac{ C \varepsilon }{ \tilde C } < \frac{L_2(\mathbf{A})}{\tilde C}
%\end{align}
\end{proof}

%We then show that the map over subsets of column indices defined by this association is such that the number of common elements shared by sets in its domain bounds the number of elements common to their images under this map. 
 
\section{Discussion}\label{Discussion}

In this note, we generalize the approach of \cite{Hillar15} for showing uniqueness in Prob.~\ref{InverseProblem} to the case of noisy measurements while also reducing the sufficient number of samples.
% from $N=k{m \choose k}^2$ to $N = m(k-1){m \choose k}+m$. 
% CUT OK? Surprisingly, almost all $n \times m$ dictionaries satisfying the standard assumption \eqref{CScondition} from compressed sensing (CS) are identifiable from enough generic noisy $k$-sparse linear combinations of their elements, up to an error linear in the noise. Moreover, if solutions are constrained to satisfy \eqref{SparkCondition}, then only an upper bound on the number of dictionary elements need be taken as given. 
One interesting aspect of our contribution is a combinatorial criteria (regular hypergraphs satisfying the singular intersection property) for choosing support sets for original sparse codes; fully understanding those combinatorial designs allowing for stable representations of datasets is the subject of future work.
We also remark that our main uniqueness result (Thm.~\ref{DeterministicUniquenessTheorem}) accounts for deterministic ``worst-case'' noise, whereas the ``effective'' noise might be much smaller when it is sampled from a given distribution; in such cases, the constants in our theorems %Thms.~\ref{DeterministicUniquenessTheorem},~\ref{DeterministicUniquenessTheorem2} 
will be different. %We note also that these results extend trivially to cases where point-wise injective nonlinearities are applied to the data. 
We close with several application areas.

% FRITZ: Inverse problems instead of Data Analysis. Don't assume that everyone is assuming they are recovering ground truth. Cite results again and state implications! Focus on them (probabilistic ones too).
\textbf{Inverse Problems}.  
Our results provide theoretical grounding for the use of sparse linear coding in blind source separation, wherein the goal is to infer the generating dictionary and sparse codes from noisy measurements generated as in \eqref{LinearModel} (e.g., recovering a rat's position on a linear track from local field potentials in Hippocampus \cite{Agarwal14}). It would be of practical utility therefore to determine the best possible dependence of $\varepsilon$ on $\delta_1, \delta_2$ (see \eqref{epsdel}) as well as the minimal requirements on the number and diversity of generating codes. %We encourage researchers to extend our results and tighten these parameters.
Phrased in the language of Hadamard \cite{Hadamard1902}),  Mention sparse matrix factorization. [Effective computation of stable datasets, at least in principle.]
[maybe smooth analysis something here]
We hope that researchers in sparse linear coding continue to improve on the bounds and constants here.

\textbf{Theoretical Neuroscience}.
Sparse dictionary learning and related methods have recovered characteristic components of natural images and sounds \cite{bellsejnowski1996, smithlewicki2006, Carlson12}, reproducing response properties of cortical neurons. Our theorems suggest that this correspondence could be due to the uniqueness of sparse representations. Furthermore, our guarantees justify the hypothesis of \cite{Coulter10, Isely10} that sparse codes passed through a communication bottleneck in the brain can be recovered from random projections via (unsupervised) biologically plausible sparse dictionary learning (e.g., \cite{rehnsommer2007, rozell2007neurally, hu2014hebbian}).    Invariance: \cite{pitts1947}.
%TODO: add reproduction of response propeties in olfactory cortex?

% Reiterate probabilistic result in the smooth analysis part?
%\textbf{Smoothed Analysis}.
%The main concept in smoothed analysis \cite{Spielman04} is that certain algorithms having exponential worst-case behavior are, nonetheless, efficient if certain (typically, measure zero in the continuous case and with ``low probability" in the discrete case) pathological input sets are avoided. Our results imply that if there is an efficient ``smoothed" algorithm for solving Prob.~\ref{InverseProblem} given enough samples, then for generic inputs this algorithm determines the unique original solution. We note that avoiding certain pathological sets of inputs is often a necessary technicality for dictionary learning \cite{Razaviyayn15, Tillmann15}.

\textbf{Engineering}.
Several groups utilize compressed sensing for signal processing tasks: MRI analysis \cite{lustig2008compressed},  image compression \cite{Duarte08}, and, more recently, the design of an ultrafast camera \cite{Gao14}. Given such effective uses of CS, it is only a matter of time before these systems incorporate sparse dictionary learning to encode and process data. Guarantees such as those offered by our theorems allow any such device to be equivalent to any other (having different initial parameters and data samples) as long as enough data originate from a statistically identical system.


\acknow{We thank Friedrich Sommer for introducing us to sparse dictionary learning, Darren Rhea for sharing early explorations, and Ian Morris for posting \eqref{eqdim} online.}

\showacknow % Display the acknowledgments section

% \pnasbreak % splits and balances the columns before the references.
% If you see unexpected formatting errors, try commenting out this line
% as it can run into problems with floats and footnotes on the final page.
%\pnasbreak %was causing only first two pages to print..??

% Bibliography
\bibliography{chazthm_pnas}

%% SUPPLEMENTAL INFO? %%%
%\begin{remark}
%Actually, $\mathbf{A}$ need not be injective on all $k$-sparse vectors for Theorem \ref{DeterministicUniquenessTheorem} to hold; rather, it need only be injective the set of all vectors with supports in $E$. [** TODO ** we will probably have to redefine $L_2$ since it is over all $k$-sparse vectors -- may have to instead define $L_E$.] Wait, $E$ need not be regular for $B$ sat. spark cond. Also it need not be $k$-regular!
%\end{remark}

\clearpage

\section{Proof of Main Lemma}

We require some auxiliary lemmas.

\begin{lemma}\label{SpanIntersectionLemma}
Let $\mathbf{M} \in \mathbb{R}^{n \times m}$. If every $2k$ columns of $\mathbf{M}$ are linearly independent, then for any $F \subseteq {[m] \choose k}$:
\begin{align*}
\text{\rm Span}\{\mathbf{M}_{\cap F}\}  = \cap_{S \in F} \text{\rm Span}\{\mathbf{M}_S\}.
\end{align*}
\end{lemma}
\begin{proof}
By induction, it is enough to prove the lemma when $|F| = 2$. The proof now follows directly from the assumption.
\end{proof}

% This can be made tighter by using Pythagoras' Thm for first projection instead of triangle inequality. 
\begin{lemma}\label{DistanceToIntersectionLemma}
Fix $k \geq 2$. Let $V_1, \ldots, V_k$ be subspaces of $\mathbb{R}^m$ and set $V = \cap_{i = 1}^k V_i$. For every $\mathbf{x} \in \mathbb{R}^m$, we have:
\begin{align}\label{DTILeq}
\text{\rm dist}(\mathbf{x}, V) \leq \frac{1}{r(\{V_i\}_{i = 1}^k)} \sum_{i=1}^k \text{\rm dist}(\mathbf{x}, V_i),
\end{align}
where $r$ is given in Def.~\ref{SpecialSupportSet}.
\end{lemma}
\begin{proof} 
Recall that orthogonal projection onto a subspace $V \subseteq \mathbb{R}^m$ is the mapping $\Pi_V: \mathbb{R}^m \to V$ that associates with each $\mathbf{x}$ its unique nearest point in $V$; i.e., $\|\mathbf{x} - \Pi_V\mathbf{x}\|_2 = \text{\rm dist}(\mathbf{x}, V)$.
% Since subspaces of real vector spaces are closed, we can replace inf with min in def. of orthogonal projection

We begin the proof by observing:
%Use Pythagoras' Theorem first?
\begin{align}\label{f}
\|\mathbf{x} - \Pi_V\mathbf{x}\|_2 &\leq \|\mathbf{x} - \Pi_{V_k} \mathbf{x}\|_2 + \|\Pi_{V_k}  \mathbf{x} - \Pi_{V_k}\Pi_{V_{k-1}}\mathbf{x}\|_2 \nonumber \\
&\ \ \ + \cdots + \|\Pi_{V_k} \Pi_{V_{k-1}}\cdots \Pi_{V_1} \mathbf{x} - \Pi_V \mathbf{x}\|_2 \nonumber \\
&\leq \|\Pi_{V_k}\cdots\Pi_{V_{1}} \mathbf{x} - \Pi_V \mathbf{x}\|_2 + \sum_{\ell=1}^k \|\mathbf{x} - \Pi_{V_{\ell}} \mathbf{x}\|_2,
\end{align}
%
using the triangle inequality and that the spectral norm satisfies $\|\Pi_{V_{\ell}}\|_2 \leq 1$ for all $\ell$ (since $\Pi_{V_{\ell}}$ are orthogonal projections).

The desired result \eqref{DTILeq} now follows by bounding the second term on the right-hand side using the following fact \cite[Thm.~9.33]{Deutsch12}:
\begin{align}
\|\Pi_{V_k}\Pi_{V_{k-1}}\cdots\Pi_{V_1} \mathbf{x} - \Pi_V\mathbf{x}\|_2 \leq z \|\mathbf{x}\|_2, %  \ \ \  \text{for } \mathbf{x} \in \mathbb{R}^m,
\end{align}
for \mbox{$z^2= 1 - \prod_{\ell =1}^{k-1}(1-z_{\ell}^2)$} and \mbox{$z_{\ell} = \cos\theta\left(V_{\ell}, \cap_{s=\ell+1}^k V_s\right)$}. Together with $\Pi_{V_\ell} \Pi_V = \Pi_V$ for all $\ell = 1, \ldots, k$ and $\Pi_V^2 = \Pi_V$, this yields:
\begin{align*}
\|\Pi_{V_k} \cdots \Pi_{V_1}\mathbf{x}  - \Pi_V \mathbf{x} \|_2 
&= \|\left( \Pi_{V_k} \cdots\Pi_{V_1} - \Pi_V \right) (\mathbf{x} - \Pi_V\mathbf{x})\|_2 \\
&\leq z\|\mathbf{x} - \Pi_V\mathbf{x}\|_2.
\end{align*}

Substituting this into \eqref{f} and rearranging yields \eqref{DTILeq} after replacing $1 - z$ with $r(\{V_i\}_{i=1}^k)$ (as $z$ may depend on the arbitrary ordering of the $V_i$).
\end{proof}

\begin{lemma}\label{NonEmptyLemma} 
Fix integers $\ell, m$ and $\hat m$. Suppose $E \subseteq 2^{[m]}$ is $\ell$-regular and satisfies the singleton intersection property. If there exists a map $\pi: E \to 2^{[\hat m]}$ such that $\sum_{S \in E} |S| = \sum_{S \in E} |\pi(S)|$ and:
\begin{align}\label{cond}
|\bigcap_{S \in F} \pi(S)| \leq |\bigcap_{S \in F} S |,\ \ \   \text{for all } F \subseteq E,
\end{align}
%
then $\hat m \geq m$ and the association $i \mapsto \cap \pi(F(i))$ defines an injective map from $[J]$ to $[\hat m]$ for some $J \subseteq [m]$ of size $\hat m - \ell(\hat m - m)$. In particular, if $\hat m = m$ then the map $\pi$ is induced by a permutation.
%$\pi(E)$ is $\ell$-regular and satisfies the singleton intersection property. In particular, $|\cap_{S \in F} S| = 1$ if and only if $|\cap_{S \in F} \pi(S)| = 1$. 
\end{lemma}
\begin{proof}
[ ** TODO ** rewrite for $\hat m > m$. ]
We first show that $\hat m \geq m$. Fix $T = \{(j, S): j \in \pi(S), S \in E\}$. Then $|T| = \sum_{S \in E} |\pi(S)| = \sum_{S \in E} |S| = \sum_{i \in [m]} \deg(i) = \ell m$ by regularity of $E$. If $\hat m < m$ then by pigeonholing the $m \ell \geq \ell (\hat m + 1) \geq [(\ell + 1) - 1] \hat m + 1$ elements of $T$ with respect to the set $[\hat m]$ of possible first indices, we see that there must exist at least $\ell + 1$ elements of $T$ having the same first index. But by \eqref{cond} there can be no more than $\ell$ elements of $T$ with a given first index, since $E$ is $\ell$-regular. Hence $\hat m \geq m$. Again, pigeonholing the elements of $T$ with respect to the set $[\hat m]$ of possible first indices, we see that for each $j$ there must be exactly $\ell$ elements of $T$ with $j$ as a first index; hence, $\pi(E)$ is $\ell$-regular. 

Fix $j$ and let $F(j) = \{S \in E: j \in \pi(S)\}$ be the preimage of the star in $\pi(E)$ centered at $j$, which is of size $\ell$ by the above arguments. It follows from \eqref{cond} that $\cap_{S \in F(j)} S$ is nonempty. In fact, we must have $|\cap _{S \in F(j)} S| = 1$, since $E$ is $\ell$-regular and satisfies the singleton intersection property. It follows by \eqref{cond} that $\cap_{S \in F(j)} \pi(S) = \{j\}$. Thus $\pi(E)$ satisfies the singleton intersection property as well, and the preimage of every one of the $m$ stars in $\pi(E)$ is a star in $E$.

It remains to show that every star in $E$ maps through $\pi$ to some star in $\pi(E)$. This follows by pigeonholing the $m$ stars of $\pi(E)$ with respect to their $m$ possible preimages if no two stars in $\pi(E)$ share the same preimage. This indeed must be the case, since $F(i) = F(j)$ implies $\{i\} = \cap_{S \in F(i)} \pi(S) = \cap_{S \in F(j)} \pi(S) = \{j\}$.
\end{proof}

\begin{proof}[Proof of Main Lemma]
We first claim $\dim(\text{\rm Span}\{\mathbf{B}_{\pi(S)}\}) = \dim(\text{\rm Span}\{\mathbf{A}_{S}\})$ for all $S \in E$. To see why, note that the right-hand side of \eqref{GapUpperBound} is then strictly less than one, since $r \leq 1$ and $L_2(\mathbf{A}) \leq 1$ by \eqref{delrho}. By \eqref{dimLem}, it then follows that $|\pi(S)| \geq \dim(\text{\rm Span}\{\mathbf{B}_{\pi(S)}\}) \geq \dim(\text{\rm Span}\{\mathbf{A}_{S}\}) = |S|$, with the equality due to $\mathbf{A}$ satisfying \eqref{SparkCondition}; hence, $\dim(\text{\rm Span}\{\mathbf{B}_{\pi(S)}\}) = \dim(\text{\rm Span}\{\mathbf{A}_{S}\})$ (since $|S| = |\pi(S)|$). It follows that the columns of $\mathbf{B}_{\pi(S)}$ are linearly independent and, by \eqref{eqdim}:% and $\mathbf{B}_i \neq \textbf{0}$ for all $i$.
\begin{align}\label{eq2}
d(\text{\rm Span}\{\mathbf{B}_{\pi(S)}\}, \text{\rm Span}\{\mathbf{A}_S\} ) = d(\text{\rm Span}\{\mathbf{A}_S\}, \text{\rm Span}\{\mathbf{B}_{\pi(S)}\}).
\end{align}
We will now show that \eqref{cond} holds.
%\begin{align}\label{fact2}
%|\bigcap_{S \in F} \pi(S)| \leq |\bigcap_{S \in F} S |,\ \ \   \text{for all } \ F \subseteq E.
%\end{align}

Fix $F \subseteq E$. Since $\text{\rm Span}\{\mathbf{B}_{\cap_{S \in F}\pi(S)}\} \subseteq \cap_{S \in F} \text{\rm Span}\{\mathbf{B}_{\pi(S)}\}$, if $\cap_{S \in F} \text{\rm Span}\{\mathbf{B}_{\pi(S)}\} = \textbf{0}$ then we must have $|\cap_{S \in F} \pi(S)| = 0$ (as $\mathbf{B}_i \neq \textbf{0}$ for all $i$) and \eqref{cond} trivially holds. Suppose then that the intersection is not the zero vector. By Lem.~\ref{SpanIntersectionLemma} and Lem.~\ref{DistanceToIntersectionLemma}, and incorporating \eqref{eq2} and \eqref{GapUpperBound}, we have:
\begin{align}\label{randoml}
d( \text{\rm Span}&\{\mathbf{B}_{\cap_{S \in F}\pi(S)}\}, \text{\rm Span}\{\mathbf{A}_{\cap_{S \in F} S}\}  ) \nonumber \\
&\leq d\left( \cap_{S \in F} \text{\rm Span}\{\mathbf{B}_{\pi(S)}\}, \cap_{S \in F} \text{\rm Span}\{\mathbf{A}_{S}\} \right) \nonumber \\
&\leq \sum_{S \in F} \frac{ d\left( \cap_{T \in F} \text{\rm Span}\{\mathbf{B}_{\pi(T)}\},\text{\rm Span}\{\mathbf{A}_{S}\} \right) }{ r( \{ \text{\rm Span}\{\mathbf{A}_{T}\}_{T \in F}) } \nonumber \\
&\leq \sum_{S \in F} \frac{ d\left( \text{\rm Span}\{\mathbf{B}_{\pi(S)}\},\text{\rm Span}\{\mathbf{A}_{S}\} \right) }{ r( \{ \text{\rm Span}\{\mathbf{A}_{T}\}_{T \in F}) }\nonumber \\
%&\leq \frac{1}{r( \{ \text{\rm Span}\{\mathbf{A}_{S}\}_{S \in F})} \sum_{S \in F} \frac{ \varepsilon }{ L(\mathbf{AX}_{J(S)}) } \nonumber \\
&\leq \frac{|F| \varepsilon}{r( \{ \text{\rm Span}\{\mathbf{A}_{S}\} \}_{S \in F})} 
%&< L_2(\mathbf{A}) \frac{|F| }{ 2^{|E|}} \left( \frac{ \min_{\substack{F \subseteq E}} r( \{ \text{\rm Span}\{\mathbf{A}_{S}\}_{S \in F}) }{ r( \{ \text{\rm Span}\{\mathbf{A}_{S}\}_{S \in F}) } \right) \left( \frac{\min_{S \in E} L_k(\mathbf{AX}_{I(S)}) }{ \min_{T \in F} L(\mathbf{AX}_{J(T)}) } \right) \nonumber \\
\leq C_2 \varepsilon. 
\end{align}
%
Note that since $\varepsilon < L_2(\mathbf{A}) / \tilde C_2$, by \eqref{delrho} the right-hand side in \eqref{randoml} is strictly less than one, so \eqref{dimLem} implies that $\dim(\text{\rm Span}\{\mathbf{B}_{\cap_{S \in F}\pi(S)}\}) \leq \dim(\text{\rm Span}\{\mathbf{A}_{\cap_{S \in F} S}\})$ and \eqref{fact2} follows from the linear independence of the columns of $\mathbf{A}_{S}$ and $\mathbf{B}_{\pi(S)}$ for all $S \in F$.

Now, fix $\ell \in [m]$. Since $E$ satisfies the singleton intersection property, $\{\ell\} = \cap F(\ell)$ for the star $F(\ell) \subseteq E$. By \eqref{cond}, we have that $\cap_{S \in F(\ell)} \pi(S)$ is either empty or it contains a single element. Lem.~\ref{NonEmptyLemma} ensures the latter case is the only possibility. Thus, the association $\ell \mapsto \cap_{S \in F(\ell)} \pi(S)$ defines a map $\hat \pi: [m] \to [m]$ and $\dim(\text{\rm Span}\{\mathbf{B}_{\hat \pi(\ell)}\}) = 1$, since $\mathbf{B}_i \neq \textbf{0}$ for all $i$. By \eqref{eqdim}, it follows from \eqref{randoml} that $d\left( \text{\rm Span}\{\mathbf{A}_\ell\}, \text{\rm Span}\{ \mathbf{B}_{\hat \pi(\ell)} \} \right) \leq C_2 \varepsilon$. Since $\ell$ is arbitrary, fixing $\hat \varepsilon = C_2\delta$ it follows from \eqref{randoml} that for every basis vector $\mathbf{e}_\ell \in \mathbb{R}^m$ there exists some $c'_\ell \in \mathbb{R}$ such that $\|\mathbf{A}\mathbf{e}_\ell - c'_\ell \mathbf{B}\mathbf{e}_{\hat \pi(\ell)}\|_2 \leq \hat \varepsilon < L_2(\mathbf{A})$. This is exactly the supposition in \eqref{1D} (letting $c_i = \|\mathbf{A}_i\|_2^{-1}$ for all $i$) and the result follows from the subsequent arguments for the case $k=1$ in Sec.~\ref{DUT}.
\end{proof}

%The above arguments can be easily modified to prove the following variation of Lem.~\ref{MainLemma}, key to proving Thm.~\ref{DeterministicUniquenessTheorem2}. 

%\begin{lemma}\label{MainLemma2}
%Fix integers $n$ and $k < m \leq \hat m$ and suppose $\mathbf{A}~\in~\mathbb{R}^{n \times m}$ and $\mathbf{B} \in \mathbb{R}^{n \times \hat m}$ both satisfiy \eqref{SparkCondition}. There exists a constant $C_4 > 0$ for which for all $\varepsilon < L_2(\mathbf{A}) / C_4$ the following holds. If for some $E \subseteq2^{[m]}$ satisfying the singleton intersection property there exists a size-preserving map $\pi: E \mapsto 2^{[\hat m]}$ satisfying:
%\begin{align}\label{GapUpperBound}
%d(\text{\rm Span}\{\mathbf{A}_{S}\}, \text{\rm Span}\{\mathbf{B}_{\pi(S)}\}) \leq \varepsilon, \ \ \   \text{for all $S \in E$},
%\end{align}
%
%then for some $J \in {[\hat m] \choose m}$, permutation matrix $\mathbf{P}$ and invertible diagonal matrix $\mathbf{D}$, we have:
%\begin{align}\label{MainLemmaBPD}
%\|\mathbf{A}_j - (\mathbf{BPD})_j\|_2 \leq C_4 \varepsilon, \ \ \  \text{for } j \in [m].
%\end{align}
%The constant $C_4$ is given by:
%\begin{align}\label{Cdefm}
%C_4 := \frac{2^{|E|}}{R} \max_{j \in [m]} \|\mathbf{A}_j\|_2
%\end{align}
%
%with $R$ being the lesser of $\min_{\substack{F \subseteq E}} r( \{ \text{\rm Span}\{\mathbf{A}_{S}\} \}_{S \in F})$ and $\min_{\substack{F \subseteq {[\hat m] \choose k}}} r( \{ \text{\rm Span}\{\mathbf{B}_{S}\} \}_{S \in F})$.
%\end{lemma}

%\begin{proof}
%The proof is very similar to the case $\hat m = m$, the difference being that now we may not invoke Lem.~\ref{NonEmptyLemma} (which requires $\hat m = m$) to infer from \eqref{fact2} that $| \cap_{S \in F(i)} \pi(S) | = 1$ for all stars $F(i)$. We circumvent this issue by instead assuming the spark condition on $\mathbf{B}$, which allows us to swap the roles of $\mathbf{A}$ and $\mathbf{B}$ in the arguments leading to the derivation of \eqref{fact2} to prove the opposite inequality for every star $F(i)$. Of course, this requires that the constant $C_4$  depend on $\mathbf{B}$. By these arguments we establish that $| \cap_{S \in F(i)} \pi(S) | = 1$ for all stars $F(i)$, yielding an injective map $\hat \pi: [m] \to [\hat m]$. The result then follows from the proof of the case $k=1$. 
%\end{proof}

\section{Proofs of Thm.~\ref{robustPolythm} and Cor.~\ref{ProbabilisticCor}}\label{AppendixB} %Cors.~\ref{DeterministicUniquenessCorollary} \& \ref{ProbabilisticCor}}\label{AppendixB}

%\begin{proof}[Proof of Cor.~\ref{DeterministicUniquenessCorollary}]
%We need only demonstrate how to produce $N$ vectors $\mathbf{a}_i$ such that for some $E \subseteq {[m] \choose k}$ satisfying the singleton intersection property there are \mbox{$(k-1){m \choose k}+1$} vectors supported on each $S \in E$ in general linear position. Let $\gamma_1, \ldots, \gamma_N$ be any distinct numbers. Then the columns of the $k \times N$ matrix $V = (\gamma^i_j)^{k,N}_{i,j=1}$ are in general linear position (since the $\gamma_j$ are distinct, any $k \times k$ ``Vandermonde" sub-determinant is nonzero). Next, fix some $E \subseteq {[m] \choose k}$ satisfying the singleton intersection property. Finally, form the $k$-sparse vectors $\mathbf{x}_1, \ldots, \mathbf{x}_N \in \mathbb{R}^m$ with supports $S \in E$ (partitioning the $\mathbf{x}_i$ evenly among these supports so that each contains $(k-1){m \choose k}+1$ vectors $\mathbf{x}_i$) by setting the nonzero values $\mathbf{x}_i$ to be those contained in the $i$th column of $V$.
%\end{proof}
% Actually this is even overkill since 

We now determine classes of datasets $Y$ having a stable sparse representation that are cut out by a single polynomial equation.

\begin{proof}[Proof of Thm.~\ref{robustPolythm}]
We sketch the argument, leaving the details to the reader. 
Let $M$ be the $n \times m$ matrix with columns $\mathbf{A}\mathbf{x}_i$, $i \in [N]$.  Consider the following polynomial \cite[Sec.~IV]{Hillar15} in the entries of $\mathbf{A}$ and the $\mathbf{x}_i$:
\begin{align*}
g(\mathbf{A}, \{\mathbf{x}_i\}_{i=1}^N) = \prod_{S \in {[n] \choose k}} \sum_{S' \in {[N] \choose k}} (\det M_{S',S})^2,
\end{align*}
with notation as in Sec.~\ref{Results}.  

It can be checked that when $g$ is nonzero for a substitution of real numbers for the indeterminates, all of the genericity requirements on $\mathbf{A}$ and $\mathbf{x}_i$ in our proofs of stability in Thm.~\ref{DeterministicUniquenessTheorem} are satisfied (in particular, the spark condition on $\mathbf{A}$). The statement of the theorem now follows directly.
\end{proof}

\begin{proof}[Proof of Cor.~\ref{ProbabilisticCor}]
First, note that if a set of measure spaces $\{(X_{\ell}, \Sigma_{\ell}, \nu_{\ell})\}_{\ell=1}^p$ is such that $\nu_{\ell}$ is absolutely continuous with respect to $\mu$ for all $\ell = 1, \ldots, p$, where $\mu$ is the standard Borel measure on $\mathbb{R}$, then the product measure $\prod_{\ell=1}^p \nu_{\ell}$ is absolutely continuous with respect to the standard Borel product measure on $\mathbb{R}^p$ (e.g.,  \cite{folland2013real}). By Thm.~\ref{robustPolythm}, there is a polynomial that is nonzero whenever $Y$ has a stable $k$-sparse representation in $\mathbb R^m$; in particular, this property (stability) holds with probability one.
\end{proof}

%\begin{proof}[Proof of Thm.~\ref{DeterministicUniquenessTheorem2}]
%The proof is the same as that of the case $\hat m = m$, only now we establish a map $\pi: E \to {[\hat m] \choose k}$ by pigeonholing $(k-1){\hat m \choose k} + 1$ vectors with respect to holes $[\hat m]$ and eventually applying Lem.~\ref{MainLemma2} instead of Lem.~\ref{MainLemma}. 

%In this case the constant $C_3$ is given by:
%\begin{align}\label{Cdefm}
%C_3 := \frac{2^{|E|} \max_{j \in [m]} \|\mathbf{A}_j\|_2}{ R \min_{S \in E} L_k(\mathbf{AX}_{I(S)}) },
%\end{align}
%
%with $R$ being the lesser of $\min_{\substack{F \subseteq E}} r( \{ \text{\rm Span}\{\mathbf{A}_{S}\} \}_{S \in F})$ and $\min_{\substack{F \subseteq {[\hat m] \choose k}}} r( \{ \text{\rm Span}\{\mathbf{B}_{S}\} \}_{S \in F})$.
%\end{proof}

\end{document}