\documentclass[journal, twocolumn]{IEEEtran}

% *** MATH PACKAGES ***
\usepackage{amsmath, amssymb, amsthm} 
\newtheorem{theorem}{Theorem}
\newtheorem{lemma}{Lemma}
\newtheorem{conjecture}{Conjecture}
\newtheorem{problem}{Problem}
\newtheorem{question}{Question}
\newtheorem{proposition}{Proposition}
\newtheorem{definition}{Definition}
\newtheorem{corollary}{Corollary}
\newtheorem{remark}{Remark}
\newtheorem{example}{Example}


\usepackage[pdftex]{graphicx}

% *** ALIGNMENT PACKAGES ***
\usepackage{array}

% correct bad hyphenation here
\hyphenation{op-tical net-works semi-conduc-tor}

\begin{document}

\title{lil' Lemma}

\author{C. Garfinkle}

\maketitle

\begin{lemma}
Suppose $E \subseteq 2^{[m]}$ is a hypergraph with the SIP and let $d_1 \geq d_2 \geq \ldots \geq d_m$ be the degree sequence of nodes in $E$. If for some $\bar m$ the map $\pi: E \to 2^{[\bar m]}$ has $\sum_{S \in E} |\pi(S)| = \sum_{S \in E} |S|$ and:
\begin{align}\label{cond}
|\cap \pi(F)| \leq |\cap F | \text{ for all } F \subseteq E,
\end{align}
then $\bar m \geq\sum_i d_i / d_1$ and the association $i \mapsto \cap \pi(F(i))$ defines an injective map from $J$ to $[\bar m]$ for some $J \in {[m] \choose p}$, where $p$ (provided it exists) is the largest positive integer in $\{1, \ldots, m\}$ satisfying:
\begin{align}\label{pcond}
\sum_{i=\ell}^{m} d_{i} > (\bar m + 1 - \ell) (d_\ell - 1) \ \ \text{for all } \ell \leq p.
\end{align}
If $\bar m < \sum_i d_i / (d_1 - 1)$ then $p \geq 1$. 
In particular, if $\bar m = m$ and $E$ is regular, then $\pi$ is induced by a permutation on $[m]$. 
\end{lemma}

\begin{proof}
Consider the collection of pairs: $T_1 := \{(i, S): i \in \pi(S), S \in E\}$, which number $|T_1| = \sum_{S \in E} |\pi(S)| = \sum_{S \in E} |S| = \sum_{i \in [m]} d_i$. Note that assumption \eqref{cond} implies $\bar m \geq |T_1| / d_1$, since otherwise pigeonholing the elements of $T_1$ with respect to their set of possible first indices $[\bar m]$ would lead us to conclude that there are more than $d_1$ sets in $E$ sharing a common element. %QUESTION: Do we really need to say this..?

By \eqref{pcond} we have $|T_1| >  \bar m (d_1 - 1)$, which implies, again by the pigeonhole principle, that there must be at least $d_1$ elements of $T_1$ sharing the same first index. By \eqref{cond}, the intersection of the set $F_1$ consisting of their second indices is non-empty. As $d_1 \geq d_i$ for all $i$, and $E$ satisfies the SIP, it must be that the sets in $F_1$ intersect at a singleton. Since $\cap \pi(F_1)$ is non-empty, applying \eqref{cond} again implies $\cap \pi(F_1) = \{i_1\}$ for some $i_1 \in [m]$. If $p=1$ then we are done. Otherwise, define $T_2 := T_1 \setminus \{(i_1,S) \in T_1: S \in E\}$, which contains $|T_2| = |T_1| - d_1 = \sum_{i=2}^m d_i$ ordered pairs having $\bar m - 1$ distinct first indices. By \eqref{pcond} we have $|T_2| > (\bar m - 1)(d_2 - 1)$ and reiterating the above arguments produces a (necessarily) distinct index $i_2$. Iterating the arguments $p$ times yields the set of singletons \mbox{$J = \{\cap F_1, \ldots, \cap F_p\} \subseteq [m]$}.
\end{proof}

\begin{remark}[Regular hypergraphs]
Note that if $E$ is $d$-regular then $\sum_{i=\ell}^m d = (m-\ell+1) d$ and we have $\bar m \geq \sum_i d_i / d_1 \geq m$ while \eqref{pcond} becomes $\ell < \bar m - (\bar m - m) d + 1$ for all $\ell \leq p$. Hence, it suffices to know that $p = \bar m - (\bar m - m) d$ is positive to know that \eqref{pcond} is satisfied, which is true if an only if $\bar m < md / (d+1)$. Hence we must have $m \leq \bar m < md / (d-1)$ for the conclusion of the lemma to hold for some non-empty $J \subseteq [m]$. We therefore have $J = [m]$ when $\bar m = m$ and $E$ is regular. 
\end{remark}

NOTE: I think assuming that for some $d$ we have $d_i \in \{d, d+1\}$ for all $i$ may be enough, actually.

\begin{remark}[General hypergraphs]
For what range of values of $\bar m$ can we guarantee there exists some $p \in \{1, \ldots, m\}$ in general? If $\bar m < \sum_i d_i / (d_1-1)$ then we can peel off at least one index for any hypergraph, i.e. $p \geq 1$. Can we find a lower bound on $\bar m$ in general, though (not just for regular hypergraphs)? The difficulty is we have to make sure \eqref{pcond} holds for \emph{all} values of $\ell \leq p$, which was easy for regular hypergraphs (by the resulting transitivity of \eqref{pcond}). My thought is that in general the lower bound $\bar m \geq \sum_i d_i / d_1$ which *has* to hold given \eqref{cond} might not correspond to what condition \eqref{pcond}, which comes from our crappy pigeonholing, gives us as a constraint on $\bar m$ when we make the substitution $p=m$ (as was the case for regular hypergraphs). Could we have $\bar m < m$ but still recover a positive $p$ number of indices?
\end{remark}

\begin{question}
Do we only need $E$ to satisfy the SIP on the $p$ nodes of highest degree, those we end up isolating, for all of Chaz' Thm. to run through? That would be good!
\end{question}

\end{document}