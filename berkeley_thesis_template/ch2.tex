\chapter{Infinite Permutation Ideals in Countable Polynomial Rings are Noetherian}

\section{Finiteness result}

Let $\Omega$ be the set of monomials in indeterminates $p_1, p_2,
\ldots$ (including the ``monomial'' $1$).  Order these monomials
lexicographically with $p_1 < p_2 < \cdots$.  The following fact
is immeadiate.

\begin{lemma}
The ordering $\leq$ is a well-ordering.
\end{lemma}

\begin{proof} Let $S$ be any nonempty set of monomials and $m = p_1^{a_1}\cdots
p_n^{a_n} \in S$.  It suffices to show that $T = \{q \in S: q <
m\}$ has a smallest element. If $q = p_1^{b_1}\cdots p_t^{b_t} \in
T$ with $b_t \neq 0$, then $t \leq n$.  In particular, $q$ only
involves the variables $p_1,\ldots,p_n$.  But it is well-known
that a set of monomials in the variables $p_1,\ldots,p_n$ with
lexicographic order has a least element (``Dickson's Lemma'').
\end{proof}


Consider the $\C$-algebra $R = \C[p_1,p_2,\ldots]$ generated by
$\Omega$ with the natural ring operations.  Also, let $R {\frak
S}_{\infty}$ be the (right) group ring with multiplication $r
\sigma \cdot s \tau = rs (\sigma \tau)$, and view $R$ as an $R
{\frak S}_{\infty}$-module in the standard way.  Suppose one has
an ideal $I$ of $R$ invariant under $R {\frak S}_{\infty}$; that
is, ${\frak S}_{\infty}I \subseteq I$ (so that $R{\frak
S}_{\infty}I \subseteq I$).  A natural question is the following.

\begin{question}
Is $I$ finitely generated as a $R {\frak S}_{\infty}$-module?
\end{question}

Before answering this question in the affirmative, we gather some
preliminary results.  Given the set of monomials $\Omega$, we
define a quasi-ordering as follows:
\begin{definition}
\[P \preceq Q \quad
:\Longleftrightarrow \quad \begin{cases} &\text{\parbox{160pt}{$P
\leq Q$, there exist $\sigma \in {\frak S}_{\infty}$ and a
monomial $m \in \Omega$ with $Q = m \sigma P$, and $U \leq P
\Rightarrow m \sigma U \leq Q$}}\end{cases}\]
\end{definition}

\begin{example}
As an example of this quasi-ordering, consider the following
chain, \[p_1^2 \preceq p_1p_2^2 \preceq p_1^3p_2p_3^2.\]  To
verify the first inequality, notice that $p_1p_2^2 = p_1 \sigma
(p_1^2)$, in which $\sigma$ is the transposition $(12)$.  If $U =
p_1^{u_1}\cdots p_n^{u_n} \leq p_1^2$, then it follows that $n =
1$ and $u_1 \leq 2$.  In particular, $p_1 \sigma U = p_1p_2^{u_1}
\leq p_1p_2^2$.

For the second relationship, we have that $p_1^3p_2p_3^2 = p_1^3
\tau (p_1p_2^2)$, in which $\tau$ is the cycle $(123)$.
Additionally, if $U = p_1^{u_1}\cdots p_n^{u_n} \leq p_1p_2^2$,
then $n = 2$ and $u_2 \leq 2$.  Also, if $u_2 = 2$, then $u_1 \leq
1$.  It follows, therefore, that $p_1^3 \tau U = p_1^3
p_2^{u_1}p_3^{u_2} \leq p_1^3p_2p_3^2$. \qed
\end{example}

Although this definition appears technical, the reason for its
introduction will become clear in the proof of the theorem above.
Since our ordering of monomials is linear ($u \leq v
\Longleftrightarrow uw \leq vw$ for all monomials $u,v,w$), this
last condition may be rewritten as $U \leq P \Rightarrow \sigma U
\leq \sigma P$.  Let us see that it is a quasi-ordering. First
notice that $P \preceq P$ since we may take $m = 1$ and $\sigma$,
the identity permutation.  Next, suppose that $P \preceq Q \preceq
T$. Then, there exist permutations $\sigma$, $\tau$ and monomials
$m_1,m_2$ such that $Q = m_1 \sigma P$, $T = m_2 \tau Q$.  In
particular, $T = m_2 (\tau m_1)(\tau \sigma P)$.  Additionally, if
$U \leq P$, then $m_1 \sigma U \leq Q$ so that $m_2 \tau (m_1
\sigma U) \leq T$.  It follows that $m_2 (\tau m_1)(\tau \sigma U)
\leq T$, which is the requirement for transitivity.

We shall also prove that $\preceq$ is a well-quasi-ordering. We
begin with some preliminary lemmas.  In what follows, it will be
convenient to represent monomials $p_1^{a_1}\cdots p_n^{a_n}$ as
vectors ($a_1,\ldots,a_n,0,\ldots$), and we will move freely from
one representation to the other.

\begin{lemma}\label{oneshiftuplem}
Suppose that $(a_1,\ldots,a_n,0,\ldots) \preceq
(b_1,\ldots,b_t,0,\ldots)$. Then, for any $c \in \N$,
$(a_1,\ldots,a_n,0,\ldots) \preceq (c,b_1,\ldots,b_t,0,\ldots)$.
\end{lemma}

\begin{proof}
First notice that we may assume that $n \leq t$ since
$(a_1,\ldots,a_n,0,\ldots)$ $\leq (b_1,\ldots,b_t,0,\ldots)$.  Let
$(c_1,c_2,\ldots,c_{t+1},0,\ldots)$ =
$(c,b_1,\ldots,b_t,0,\ldots)$, in which $c_1 = c$ and $c_i =
b_{i-1}$ for $i > 1$.  Since $(a_1,\ldots,a_n,0,\ldots) \preceq
(b_1,\ldots,b_t,0,\ldots)$, there exists a permutation $\sigma$
such that $p_{\sigma(1)}^{a_1} \cdots p_{\sigma(n)}^{a_n} \, \mid
\, p_{1}^{b_1} \cdots p_{t}^{b_t}$. Let $\tau$ be the cyclic
permutation $\tau = (123 \cdots (t+1))$.   Then, $\hat{\sigma} =
\tau \sigma$ is such that $p_{\hat{\sigma}(1)}^{a_1} \cdots
p_{\hat{\sigma}(n)}^{a_n} \, \mid \, p_{1}^{c_1} \cdots
p_{t+1}^{c_{t+1}}$.

Next, suppose that $U = p_1^{u_1} \cdots p_n^{u_n} \leq
p_{1}^{a_1} \cdots p_{n}^{a_n} = P$.  By assumption, $\sigma U
\leq \sigma P$, so that \[\hat{\sigma} U = p_{\sigma(1)+1}^{u_1}
\cdots p_{\sigma(n)+1}^{u_n} \leq p_{\sigma(1)+1}^{a_1} \cdots
p_{\sigma(n)+1}^{a_n} = \hat{\sigma} P\] since we are using lex
order.  Finally, the condition $p_{1}^{a_1} \cdots p_{n}^{a_n}
\leq p_{1}^{c_1} \cdots p_{t+1}^{c_{t+1}}$ is obvious, completing
the proof.
\end{proof}

\begin{lemma}\label{twoshiftuplem}
If $(a_1,\ldots,a_n,0,\ldots) \preceq (b_1,\ldots,b_t,0,\ldots)$
and $a,b \in \N$ are such that $a \leq b$, then
$(a,a_1,\ldots,a_n,0,\ldots) \preceq (b,b_1,\ldots,b_t,0,\ldots)$.
\end{lemma}

\begin{proof}
As before, we may assume that $n \leq t$.  Since we are using lex
order, it is clear that $(a, a_1,\ldots,a_n,0,\ldots) \leq
(b,b_1,\ldots,b_t,0,\ldots)$.  Let $\sigma \in {\frak S}_t$ be
such that $p_{\sigma(1)}^{a_1} \cdots p_{\sigma(n)}^{a_n} \, \mid
\, p_{1}^{b_1} \cdots p_{t}^{b_t}$, and let $\tau$ be the cyclic
permutation $\tau = (12 \cdots (t+1))$.  Setting $\hat{\sigma} =
\tau \sigma \tau^{-1}$, we have
\begin{equation*}
\begin{split}
\hat{\sigma} \left(p_1^a p_{2}^{a_1} \cdots p_{n+1}^{a_n} \right)
= \ &  \tau \sigma
\left(p_{t+1}^a p_{1}^{a_1} \cdots p_{n}^{a_n}\right)  \\
= \ & \tau \left(p_{t+1}^a p_{\sigma(1)}^{a_1} \cdots
p_{\sigma(n)}^{a_n}\right) \\
= \ & p_{1}^a p_{\sigma(1)+1}^{a_1} \cdots p_{\sigma(n)+1}^{a_n}.
\\
\end{split}
\end{equation*}
It is easily seen that this last expression divides
$p_{1}^{b}p_{2}^{b_1} \cdots p_{t+1}^{b_t}$.

Finally, suppose that $U = p_1^{u_1} \cdots p_{n+1}^{u_{n+1}} \leq
p_1^{a}p_{2}^{a_1} \cdots p_{n+1}^{a_n} = P$.  Then, since we are
using lex order, it follows that \[p_2^{u_2} \cdots
p_{n}^{u_{n+1}} \leq p_{2}^{a_1} \cdots p_{n+1}^{a_n}.\] By
assumption, however, this implies that $\sigma \tau^{-1} p_2^{u_2}
\cdots p_{n+1}^{u_{n+1}} \leq \sigma \tau^{-1} p_{2}^{a_1} \cdots
p_{n+1}^{a_n}$ and thus $\hat{\sigma} p_2^{u_2} \cdots
p_{n+1}^{u_{n+1}} \leq \hat{\sigma} p_{2}^{a_1} \cdots
p_{n+1}^{a_n}$.  It follows that if $\hat{\sigma} p_2^{u_2} \cdots
p_{n+1}^{u_{n+1}} \neq \hat{\sigma} p_{2}^{a_1} \cdots
p_{n+1}^{a_n}$, we must have $\hat{\sigma} U < \hat{\sigma} P$. On
the other hand, if $p_2^{u_2} \cdots p_{n+1}^{u_{n+1}} =
p_{2}^{a_1} \cdots p_{n+1}^{a_n}$, then $u_1 \leq a$, in which
case we still have $\hat{\sigma} U \leq \hat{\sigma} P$ (since
$\hat{\sigma}p_1 = p_1$).  This completes the proof.
\end{proof}

We finally have enough to prove the well-quasi-ordering of
$\preceq$.

\begin{theorem}
The quasi-order relation $\preceq$ is a well-quasi-ordering.
\end{theorem}

\begin{proof}
The proof uses some ideas from Nash-Williams' proof \cite{NW} of a
result of Higman \cite{Higman}. Assume for the sake of
contradiction that $m^{(1)},m^{(2)},\dots$ is a bad sequence in
$\Omega$, in which $m^{(i)}=(m^{(i)}_1,m^{(i)}_2,\dots)$.  Given a
monomial $m \in \Omega$, we let $j(s)$ denote the smallest index
$j$ such that $m_{j}=m_{j+1}=\cdots$.  We may assume that the bad
sequence is chosen in such a way that for every $i$, $j(m^{(i)})$
is {\it minimal} among the $j(m)$, where $m$ ranges over all
elements of $\Omega$ with the property that
$m^{(1)},m^{(2)},\dots,m^{(i-1)},m$ can be continued to a bad
 sequence in $\Omega$.   Clearly, we have
$j(m^{(i)})>1$ for all $i$.  Let
$t^{(i)}=(m_2^{(i)},m_3^{(i)},\dots) \in \Omega$ so that
$j(t^{(i)})=j(m^{(i)})-1$, for all $i$.  Now consider the sequence
$m^{(1)}_1,m^{(2)}_1,\dots,m^{(i)}_1,\dots$ of elements of $\N$.
Since $\N$ is well-ordered, there is an infinite sequence $1\leq
i_1<i_2<\cdots$ of indices such that $m^{(i_1)}_1 \leq m^{(i_2)}_1
\leq \cdots$. By minimality of $m^{(1)},m^{(2)},\dots$, the
sequence
$m^{(1)},m^{(2)},\dots,m^{(i_1-1)},t^{(i_1)},t^{(i_2)},\dots$ is
good; that is, there exist $j < i_1$ and $k$ with $m^{(j)} \preceq
t^{(i_k)}$, or $k<l$ with $t^{(i_k)} \preceq t^{(i_l)}$. In the
first case we have $m^{(j)}\preceq m^{(i_k)}$ by Lemma
\ref{oneshiftuplem}; and in the second case, $m^{(i_k)}\preceq
m^{(i_l)}$ by Lemma \ref{twoshiftuplem}. This contradicts the
badness of our sequence $m^{(1)},m^{(2)},\dots$.
\end{proof}

If $f \in R$, we define the \textit{leading monomial} of $f$,
lm$(f)$, to be the largest monomial occurring in $f$ with respect
to $\leq$. We now come to our main result.

\begin{theorem}\label{onevarfinitegenthm}
Let $I$ be an ideal of $R = \C[p_1,p_2,\ldots]$ such that ${\frak
S}_{\infty}I \subseteq I$.  Then, $I$ is finitely generated as a
$R {\frak S}_{\infty}$-module.
\end{theorem}


\begin{proof}
Let $I$ be an ideal of $R$ with ${\frak S}_{\infty}I \subseteq I$
that is not finitely generated.  Define a sequence of elements of
$I$ as follows:  $f_1$ is an element of $I$ with minimal leading
monomial;  $f_{i+1}$ is an element of $I \setminus
\<f_1,\ldots,f_i \>_{R{\frak S}_{\infty}}$ with minimal leading
monomial.  Of course, if $f_1 \in \C$, then $I = R$ or $I = (0)$
is finitely generated.

It is clear that lm($f_i$) $\leq$ lm($f_{i+1}$).  If lm($f_i$) $=$
lm($f_{i+1}$), then for a suitable $a \in \C$, $f_{i+1} - af_i \in
I \setminus \<f_1,\ldots,f_i \>_{R{\frak S}_{\infty}}$ has a
smaller leading term, a contradiction.  Thus, lm($f_i$) $<$
lm($f_{i+1}$).

We, therefore, obtain a sequence of (strictly) increasing
monomials $m_i =$ lm($f_i$). By the well-quasi-ordering theorem
above, there exist $i < j$ such that $m_i \preceq m_j$.  Let
$\sigma$ be the permutation inducing $m_i \preceq m_j$.  Then,
$m_j = m \sigma m_i$ for some monomial $m$.  Examine the element
\[g = f_{i+1} - am \sigma f_{i} \in I \setminus \<f_1,\ldots,f_i
\>_{R{\frak S}_{\infty}},\] in which $a \in \C$ is chosen so that
the leading term of $f_{i+1}$ cancels.  The leading monomial of
$g$ cannot come from any of the (necessarily, strictly smaller)
monomials occurring in $f_{i+1}$ by our choice of $f_{i+1}$.

On the other hand, any monomial occurring in $m \sigma f_{i}$ is
of the form $m \sigma u$ for $u$ occurring in $f_{i}$.  Since $u <
$ lm($f_i$) $= m_i$, the relation $m_i \preceq m_j$ implies that
$m \sigma u < m_j$.  In particular, $g$ again has a smaller
leading term than $f_{i+1}$.  This contradiction finishes the
proof.
\end{proof}
